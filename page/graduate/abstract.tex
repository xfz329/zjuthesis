\cleardoublepage
\chapternonum{摘要}
子痫前期(Preeclampsia,PE)是孕妇妊娠期特有的一种多系统进展性疾病,可引起严重的母胎并发症,是孕产妇和围产儿病死率升高的主要原因。原发性高血压与蛋白尿是PE最突出的特点,此外,PE也可
引起全身小静脉痉挛。但现阶段对PE的诊断识别主要依赖于生化标志物有创检查,受操作流程、检测场所及成本等条件制约,缺乏对PE的便捷、快速、无创、准确的临床诊断手段。

光电容积脉搏波(Photoplethysmography, PPG)是心脏周期性搏动的体现,包含了有关人体血液微循环方面的更富细节的生理信息,可以反应血液动力学上的变化。此外,PPG在采集测量过程中定位简单、易于操作,
具有无创、无痛、迅速、便捷等优点,采集得到的信号质量高、稳定性强,
在观察和评估微循环状况上具有天然优势。

本文改进脉搏波的波形检测流程,提高脉搏波波形检测准确率。研发能够表征脉搏波形态学特征的多种新型时域
特征参数,构建通用的脉搏波时域特征描述集合。对脉搏波时域特征集合进行特征筛
选,从中选取与子痫前期相关性高的特征参数,使用多种机器学习算法构建子痫前期的
识别模型,并验证所得模型的有效性与可靠性。

本文的主要内容及创新点如下:

1、基于策略与机制分离思想,提出了脉搏波波形检测的初筛-复核-投票算法

本研究对 PPG 波形的检测过程进行了模式设计,提出了一种新型 SCD 算法,算
法由初筛、复核与决策等三部分组成。在初筛时,通过搜索窗的方法确定波形;在复核时,
通过功率、标准差、波峰相对位置与基线漂移程度等 PPG 形态学特征,区分有效波形与异
常干扰;在决策时,通过加权投票的策略确定 PPG 波形是否为有效波形。SCD 算法可根据
使用场景进行二次开发,支持更改初筛算法、增删复核标准与切换决策策略等设置。经验证,
该算法各模块设计合理,与其他算法对比结果表明,对 PPG 波形的识别准确率高、抗干扰
能力强。SCD 算法为多平台、多研究应用下的 PPG 分析检测的不同场景提供了一定的参考。

2、结合PPG形态特征,从时间、幅值、角度、弧度、面积、斜率等维度设计了多种PPG新型时域特征,同时,也从
描述多个PPG波形间的差异的角度出发,设计了三种描述PPG波形间差异时域特征参数。
在PPG新型时域特征的基础上,构建了PPG多维度时域特征集合;同时,将PPG信号
的原始采样序列的幅值作为特殊的描述特征,构建了PPG采样序列时域特征集合。

3、 利用多种机器学习算法,在PPG多维度时域特征集与PPG采样序列时域特征集的基础上,分别构建了多种PE的识别模型,通过各模型在测试集上的泛化效果评估了各算法的性能,
也对两个PPG特征集本身表征PE的能力进行了评估。

综上,本文提出了针对光电容积脉搏波的一种新型检测算法,提出了多种脉搏波时域描述特征,并建立了三两脉搏波描述特征集合。在此基础上,基于机器学习相关算法,建立了可有效识别子痫前期的分类模型,
对于实现PE的识别判断及保障孕妇及围产儿的生命健康安全,具有重大的临床应用价值。


\textbf{关键词}:子痫前期;识别模型;光电容积脉搏波;波形检测算法;随机森林



\cleardoublepage
\chapternonum{Abstract}

Preeclampsia (PE) is a multisystemic progressive disorder specific to maternal pregnancy that can cause severe maternal-fetal complications and is a major cause of increased maternal and perinatal morbidity and mortality. Primary hypertension and proteinuria are the most prominent features of PE, and in addition, PE can
cause systemic small venous spasms. However, at this stage, the diagnosis of PE mainly relies on invasive biochemical markers, which is constrained by the operation procedure, testing site and cost, and lacks a convenient, rapid, non-invasive and accurate clinical diagnostic tool for PE.
Photoplethysmography (PPG) is a manifestation of the periodic pulsations of the heart, which contains more detailed physiological information about the human blood microcirculation and can reflect hemodynamic changes, and has the advantages of being rapid, convenient, non-invasive, and the acquisition equipment has become popular.

In this thesis, a variety of novel pulse wave time domain description features are proposed from the time domain characteristics of the photoelectric volume pulse wave signal, and three types of pulse wave time domain description features (sets) are constructed. Based on this, various machine learning algorithms are integrated and applied to the study of preeclampsia recognition model.
During this period, we also completed the dimensionality reduction and performance evaluation of the three types of pulse wave time-domain features. Finally, the pre-eclampsia recognition model obtained using random forest algorithm has a good performance.

The main contents and innovations of this paper are as follows:

1. based on the idea of separation of strategy and mechanism, a novel pulse waveform detection algorithm, namely Screen-Check-Vote algorithm, is proposed, which can detect more complex original pulse waveform and effectively improve the anti-interference capability of the waveform detection algorithm.

2. A variety of new pulse wave time domain description features are proposed, and three types of pulse wave time domain description features (sets) are constructed, which can be used as a general pulse wave description set to quantitatively describe the pulse waveform from the time domain perspective.

3. Based on the above pulse wave time-domain description feature set, a subset of time-domain features that can effectively characterize the pre-eclampsia is selected by using various algorithmic models of machine learning. Based on this, various recognition models for preeclampsia were established, among which, the random forest-based
Among them, the random forest-based algorithm model has the best comprehensive recognition performance.

In summary, this paper proposes a novel detection algorithm for photoelectric volume pulse wave, proposes various pulse wave time-domain descriptive features, and establishes a set of three types of pulse wave descriptive features. Based on this, a classification model that can effectively identify preeclampsia is established based on machine learning related algorithms.
It is important for the early identification and detection of preeclampsia, and to ensure the safety of mothers and infants in the perinatal period.


\textbf{Key words}:preeclampsia; recognition model; photoelectric volume pulse wave; waveform detection algorithm; random forest