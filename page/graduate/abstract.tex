\cleardoublepage
\chapternonum{摘要}
子痫前期(Preeclampsia,PE)是孕妇妊娠期特有的一种多系统进展性疾病,可引起严重的母胎并发症,是孕产妇和围产儿病死率升高的主要原因。原发性高血压与蛋白尿是PE最突出的特点,此外,PE也可
引起全身小静脉痉挛。但现阶段对PE的诊断识别主要依赖于生化标志物有创检查,受操作流程、检测场所及成本等条件制约,缺乏对PE的便捷、快速、无创、准确的临床诊断手段。
光电容积脉搏波(Photoplethysmography, PPG)是心脏周期性搏动的体现,包含了有关人体血液微循环方面的更富细节的生理信息,可以反应血液动力学上的变化,具有快速、便捷、无创、采集设备已经普及等优点。

本论文从光电容积脉搏波信号的时域特征出发,提出了多种新型脉搏波时域描述特征,构建了三类脉搏波时域描述特征(集合)。在此基础上,综合应用了多种机器学习算法进行了子痫前期识别模型的研究,
在此期间也完成了对三类脉搏波时域特征的降维处理与性能评估等工作。最终,使用随机森林算法得到的子痫前期识别模型具有较好的性能。

本文的主要内容及创新点如下:

1. 基于策略与机制分离思想,提出了一种新型脉搏波波形检测算法,即初筛-复核-投票(Screen-Check-Vote)算法,可对较为复杂的原始脉搏波波形进行检测,有效提高波形检测算法的抗干扰能力。

2. 提出了多种新型脉搏波时域描述特征,构建了三类脉搏波时域描述特征(集合),可作为通用脉搏波描述集合从时域角度对脉搏波波形进行定量描述。

3. 基于上述脉搏波时域描述特征集合,综合使用机器学习的多种算法模型,从中筛选出于能够有效表征子痫前期的时域特征子集。在此基础上建立了多种子痫前期的识别模型,其中,基于随机森林的
算法模型的综合识别性能最为优秀。

综上,本文提出了针对光电容积脉搏波的一种新型检测算法,提出了多种脉搏波时域描述特征,并建立了三类脉搏波描述特征集合。在此基础上,基于机器学习相关算法,建立了可有效识别子痫前期的分类模型,
对子痫前期的早期识别与检测、保障围产期母婴安全具有重要意义。


\textbf{关键词}:子痫前期;识别模型;光电容积脉搏波;波形检测算法;随机森林



\cleardoublepage
\chapternonum{Abstract}