\cleardoublepage
\chapternonum{摘要}
人体心电信号一直是生物医学工程重要的研究领域。而心电监护在临床上对各类心脏疾病的预防与诊断过程中一直有着不可替代的作用。
近年来随着智能手机的普及与信息通信技术的高速发展,移动医疗的概念逐渐兴起,越来越多的人开始关注通过智能手机进行某些疾病的
辅助诊断的可能。在此背景下,基于智能手机的心电监护系统由于普及率高和便携性好等突出优点,能有效帮助人们及时了解自己的心电
的相关信息。因此,针对智能手机的心电监护系统的研究与开发具有广泛的应用前景。

在前人工作的基础上,本文将心电信号分析和移动终端结合起来,在 Android 平台上开发出一套具有心电信号去噪、显示等功能的心电监护
系统。本文的工作包含理论研究和应用开发。理论上主要是借助 Matlab 研究了心电信号前期处理所涉及的信号处理方法如去噪、滤波、信号
寻峰、周期计算等相关算法的实现与仿真。应用开发的软件方面是在 Eclipse 集成环境下以 Java 为编程语言进行系统开发的。
该系统对 MIT-BIH 提供的心电异常数据进行了智能手机的分析处理,并通过图形动态显示方法直接形象的展现了滤波前后的数据差异。
同时针对在滤波后的数据里进行了特征提取,并将提取结果在图形中做出了标注。程序的最后将提取的相关特征值进行了文本显示。经实验,本系统成
功在真机 Google Nexus 10 上运行,达到了预期效果。

\textbf{关键词:}移动医疗;心电;安卓;去噪;特征提取

\cleardoublepage
\chapternonum{Abstract}
The study of the human ECG signals has been an important research area of biomedical
engineering all the time. The ECG monitoring in clinical medical has played an irreplaceable
role in diagnosing and preventing various types of heart diseases. With the widely popularity
of smart phones and the swift and violent development of information communication
technology in recent years, the concept of Mobile Health gradually rises which draws more
and more attention to the possibility of diagnosing certain diseases through smart phones.
Against the above background,the monitoring system of ECG based on the smartphone can
do great favor to help people understand the relevant information of their own ECG timely
and effectively due to the highlight advantages of the high-penetration rate and easy
portability .Therefore the research and development of the monitoring system of ECG based
on the Smartphone has a broad application prospects.

On the basis of previous work, this thesis combines the analysis of the ECG signal with
the mobile terminals, and develops a set of ECG monitoring system on the Android platform
by which we can denoise the ECG signal, display the result. The work of this thesis includes
theoretical research together with application development. In the theory, the main work is the
study, implementation and simulation of signal processing method involved in the
pre-processing of ECG signal such as noise removal, filtering, signal peak search, cycle
calculation algorithms by means of Matlab. In the software of the system, this thesis chooses
the Eclipse as the integrated environment and Java as the programming language for system
development. The system has analyzed and processed the abnormal examples of MIT-BIH
ECG database on the smartphone,also directly shows the difference of data before and after
filtering by displaying dynamic graphic images. At the same time,the thesis extracts the
characters of the data after filtering which are specially marked in the dynamic graphic
images. Finally, the relevant extracted characteristic values are displayed as text. With the
experiment, the system runs successfully on the Google Nexus 10 tablet computer and
achieves the intended result.

\textbf{Key Words:}Mobile Health;ECG; Android;Denoise;Character extraction