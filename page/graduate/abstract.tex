\cleardoublepage
\chapternonum{摘要}
子痫前期(preeclampsia,PE)是孕妇妊娠期特有的一种多系统进展性疾病,可引起全身小静脉痉挛,导致严重的母胎并发症,
是孕产妇和围产儿病死率升高的主要原因之一。但现阶段对PE的诊断识别主要通过生化标志物检查,该过程受操作流程、检测场所及成本等条件制约。
本论文基于光电容积脉搏波(photoplethysmography, PPG)对PE的识别进行了研究。
PPG是心脏周期性搏动的体现,包含了人体血液微循环方面的丰富细节,具有无创、便捷等优点。

本论文构建了包含PE病发孕妇与正常妊娠孕妇的PPG数据集,设计了PPG多维度时域特征集与PPG采样序列时域特征。
为扩充数据集规模,确定了先通过PPG单个波形、再通过被试的全部PPG波形进行PE识别的研究策略,
使用决策树、K近邻与随机森林三种算法,进行了PE识别模型的研究工作,设计并实现了基于PPG的PE识别分析软件系统。

本论文的主要内容及创新点如下:

一、提出了一种基于初筛-复核-投票的PPG波形检测算法

本研究对 PPG 波形的检测过程进行了模式设计,提出了一种基于初筛—复核—投票(screening-checking-deciding,SCD)的PPG波形检测算法,
算法由初筛、复核与决策等三部分组成。在初筛时,通过搜索窗的方法确定波形;在复核时,
通过功率、标准差、波峰相对位置与基线漂移程度等 PPG 形态学特征,区分有效波形与异
常干扰;在决策时,通过加权投票的策略确定 PPG 波形是否为有效波形。SCD 算法可根据
使用场景进行二次开发,支持更改初筛算法、增删复核标准与切换决策策略等设置。
经临床及公开数据库验证,该算法对PPG波形的检测准确率高、抗干扰能力强,
为多平台、多研究应用下的PPG分析检测的不同场景提供了一定的参考。

二、构建了PPG信号和采样序列的多维度时域特征集,提取描述PPG波形间差异的时域特征,
利用决策树、K 近邻与随机森林三种算法进行子痫前期识别模型的研究

本论文对PPG时域特征的设计过程进行了方法学的归纳,提出了PPG特征描述向量的概念,
提出了多种基于时间、幅值、角度、弧度、面积、斜率等多维度的新型PPG时域特征;
提出了描述PPG波形间差异的新型时域特征。
在上述多维度特征的基础上,通过左视策略、中视策略与分层策略,构建了PPG多维度时域特征集。
经过算法筛选,利用决策树、K 近邻与随机森林三种算法进行子痫前期识别模型的研究。
结果表明,基于PPG信号多维时域特征的随机森林算法的准确率最高(AUC为0.952,准确率93.8\%)。

三、设计并实现了一款可以满足多场景的PE识别分析软件系统,具有从预处理到PE状态识别的完整分析功能

本论文设计并实现了一款可以满足实验室研究、医院监护、社区检查、居家监护等多应用场景的PE识别分析软件系统,
系统由数据预处理、跨平台客户端、PE识别模型训练生成与云服务器程序等四个功能模块组成,具有从数据预处理到PE病发状态的识别预测的完整分析功能。
同时,软件系统也进行了多项兼容性设计,支持对核心处理算法、识别模型等进行更新迭代。
经测试验证,该PE识别分析软件系统的各项功能均能按照设计预期正常工作,能满足多场景下的不同使用需求。
同时,该系统具有良好的拓展性,能根据实际需求,拓展至其他基于PPG的应用中,为类似研究工作的开展提供便利。

综上,本论文提出了一种新型PPG波形检测算法,从时间、幅值、角度、弧度、面积、斜率等多维度描述PPG形态特征的PPG时域特征集合,
利用决策树、K 近邻与随机森林三种算法进行子痫前期识别模型的研究,开发了一款可以满足多场景的PE识别分析软件系统。
这些研究对PE的临床识别诊断具有重要意义。

\vspace{2em}

\textbf{关键词}: 子痫前期; 识别模型; 光电容积脉搏波; 波形检测算法; 随机森林;PE识别软件分析系统


\cleardoublepage
\chapternonum{Abstract}

Preeclampsia (PE) is a multi-systemic progressive disorder specific to maternal pregnancy that causes systemic small venous spasms 
leading to severe maternal-fetal complications. It is one of the major causes of increased maternal and perinatal morbidity and mortality.  
However, at this stage, the diagnostic identification of PE is mainly through biomarkers, a process constrained by operational procedures, 
testing sites, and costs. 
This thesis investigates the identification of PE based on photoplethysmography (PPG). PPG is a manifestation of the periodic pulsations 
of the heart and contains a wealth of details about the microcirculation of human blood, which has the advantages of being non-invasive and convenient.

In this thesis, a PPG dataset comprising pregnant women diagnosed with PE and normal pregnancies was constructed, 
a multidimensional time-domain feature set of PPG and a time-domain feature set of PPG sampling sequence was designed.
To augment the size of the dataset, the strategy of explore PE identification using PPG individual waveforms and subsequently 
by all PPG waveforms from the subjects was determined.
With the help of the three algorithms of decision tree, K-nearest neighbor and random forest, 
the research on the PE recognition model was conducted. Moreover, a PPG-based PE recognition analysis software 
system was designed and implemented.

The main contents and innovations of this thesis are as follows.

I. A PPG waveform detection algorithm based on screening-checking-deciding is proposed and implemented.

In this thesis, we proposed a PPG waveform detection algorithm based on screening-checking-deciding (SCD). 
The algorithm determined the waveform by searching the window in the initial screening stage; in the checking-review stage, 
it distinguished the valid waveform from the abnormal interference by the PPG morphological features such as power, standard deviation, 
relative positions of wave peaks, and baseline drift; in the decision stage, it determined whether the PPG waveform 
is a valid waveform by the weighted voting strategy. 
The SCD algorithm could be developed according to the usage scenarios and supports settings such as 
changing the initial screening algorithm, adding and deleting criteria, 
and switching decision strategies. 
This algorithm is clinically and publicly available database effective in detecting PPG waveforms, 
demonstrating high accuracy and strong anti-interference capability, suitable for multi-platform and multi-study applications.
It offers a useful guide for diverse scenarios of PPG analysis and detection under multi-platform and multi-research applications.

II. The multidimensional time-domain feature sets of PPG signals and sampling sequences was constructed, 
time-domain features describing the differences between PPG waveforms were extracted,
and decision tree, K-nearest neighbor and random forest were utilized for the PE recognition model construction.

This thesis presented a methodical summation of the design process of PPG time-domain features, 
and introduced the concept of PPG feature description vector.
A variety of novel PPG time-domain features based on multi-dimensions such as time, magnitude, angle, radian, area and slope were proposed; 
novel time-domain features depicting the differences between PPG waveforms were also proposed.
On the basis of these multidimensional features, a PPG multidimensional time-domain feature set was constructed 
using the left-view strategy, center-view strategy, and scaled-view strategy.
After the initial screening of algorithms, three algorithms, namely decision tree, K-nearest neighbor and random forest, 
were used as the main research methods for PE recognition models in this thesis.
The results showed that the Random Forest algorithm based on the multidimensional time domain features of PPG signals 
possessed the highest accuracy (AUC of 0.952, accuracy 93.8\%).


III. A PE recognition and analysis software system flexible towards multiple scenarios was designed and implemented, 
with the thorough analysis functions from data preprocessing to PE state recognition

In this thesis, a PE recognition and analysis software system was designed and implemented that could satisfy multi-application scenarios, 
such as laboratory research, hospital monitoring, community examination and home monitoring.
The system consisted of four functional modules, including data preprocessing, cross-platform client, 
PE identification model training generation and cloud server program, 
and possessed the thorough analysis functions from data preprocessing to the recognition and prediction of PE disease state.
Simultaneously, the software system also incorporates a range of compatibility designs, supporting pertinent processing algorithms, 
recognition models and other enhancements and iterations.
After rigorous testing and validation, all functions of the PE identification and analysis software system operated normally per the design expectations, 
and could fulfill the diverse requirements of usage in multiple scenarios.
Additionally, the system demonstrates commendable extensibility and could be expanded to other PPG-based applications based on the actual needs, 
fostering similar research work.

In summation, this thesis advanced a novel PPG waveform detection algorithm incorporating a collection of PPG
time domain features elucidating the morphological attributes of pulse oximeters in multiple dimensions,
such as time, magnitude, angle, radian, area and slope.
Three algorithms, decision tree, K nearest neighbour, and random forest, were utilized for 
the research of pre-eclampsia recognition model, and an application system for PE recognition and analysis 
that could accommodate diverse scenarios was constructed.
These investigations are of paramount significance to the clinical recognition and diagnosis of PE.


\vspace{2em}

\textbf{Keywords}: preeclampsia; recognition model; photoplethysmography; waveform detection algorithm; random forest;
PE recognition software analysis system