\cleardoublepage
\chapternonum{摘要}
子痫前期(Preeclampsia,PE)是孕妇妊娠期特有的一种多系统进展性疾病,可引起严重的母胎并发症,是孕产妇和围产儿病死率升高的主要原因。原发性高血压与蛋白尿是PE最突出的特点,此外,PE也可
引起全身小静脉痉挛。但现阶段对PE的诊断识别主要依赖于生化标志物有创检查,受操作流程、检测场所及成本等条件制约,缺乏对PE的便捷、快速、无创、准确的临床诊断手段。
光电容积脉搏波(Photoplethysmography, PPG)是心脏周期性搏动的体现,包含了有关人体血液微循环方面的更富细节的生理信息,可以反应血液动力学上的变化,具有快速、便捷、无创、采集设备已经普及等优点。

本论文从光电容积脉搏波信号的时域特征出发,提出了多种新型脉搏波时域描述特征,构建了三类脉搏波时域描述特征(集合)。在此基础上,综合应用了多种机器学习算法进行了子痫前期识别模型的研究,
在此期间也完成了对三类脉搏波时域特征的降维处理与性能评估等工作。最终,使用随机森林算法得到的子痫前期识别模型具有较好的性能。

本文的主要内容及创新点如下:

1. 基于策略与机制分离思想,提出了一种新型脉搏波波形检测算法,即初筛-复核-投票(Screen-Check-Vote)算法,可对较为复杂的原始脉搏波波形进行检测,有效提高波形检测算法的抗干扰能力。

2. 提出了多种新型脉搏波时域描述特征,构建了三类脉搏波时域描述特征(集合),可作为通用脉搏波描述集合从时域角度对脉搏波波形进行定量描述。

3. 基于上述脉搏波时域描述特征集合,综合使用机器学习的多种算法模型,从中筛选出于能够有效表征子痫前期的时域特征子集。在此基础上建立了多种子痫前期的识别模型,其中,基于随机森林的
算法模型的综合识别性能最为优秀。

综上,本文提出了针对光电容积脉搏波的一种新型检测算法,提出了多种脉搏波时域描述特征,并建立了三类脉搏波描述特征集合。在此基础上,基于机器学习相关算法,建立了可有效识别子痫前期的分类模型,
对子痫前期的早期识别与检测、保障围产期母婴安全具有重要意义。


\textbf{关键词}:子痫前期;识别模型;光电容积脉搏波;波形检测算法;随机森林



\cleardoublepage
\chapternonum{Abstract}

Preeclampsia (PE) is a multisystemic progressive disorder specific to maternal pregnancy that can cause severe maternal-fetal complications and is a major cause of increased maternal and perinatal morbidity and mortality. Primary hypertension and proteinuria are the most prominent features of PE, and in addition, PE can
cause systemic small venous spasms. However, at this stage, the diagnosis of PE mainly relies on invasive biochemical markers, which is constrained by the operation procedure, testing site and cost, and lacks a convenient, rapid, non-invasive and accurate clinical diagnostic tool for PE.
Photoplethysmography (PPG) is a manifestation of the periodic pulsations of the heart, which contains more detailed physiological information about the human blood microcirculation and can reflect hemodynamic changes, and has the advantages of being rapid, convenient, non-invasive, and the acquisition equipment has become popular.

In this thesis, a variety of novel pulse wave time domain description features are proposed from the time domain characteristics of the photoelectric volume pulse wave signal, and three types of pulse wave time domain description features (sets) are constructed. Based on this, various machine learning algorithms are integrated and applied to the study of preeclampsia recognition model.
During this period, we also completed the dimensionality reduction and performance evaluation of the three types of pulse wave time-domain features. Finally, the pre-eclampsia recognition model obtained using random forest algorithm has a good performance.

The main contents and innovations of this paper are as follows:

1. based on the idea of separation of strategy and mechanism, a novel pulse waveform detection algorithm, namely Screen-Check-Vote algorithm, is proposed, which can detect more complex original pulse waveform and effectively improve the anti-interference capability of the waveform detection algorithm.

2. A variety of new pulse wave time domain description features are proposed, and three types of pulse wave time domain description features (sets) are constructed, which can be used as a general pulse wave description set to quantitatively describe the pulse waveform from the time domain perspective.

3. Based on the above pulse wave time-domain description feature set, a subset of time-domain features that can effectively characterize the pre-eclampsia is selected by using various algorithmic models of machine learning. Based on this, various recognition models for preeclampsia were established, among which, the random forest-based
Among them, the random forest-based algorithm model has the best comprehensive recognition performance.

In summary, this paper proposes a novel detection algorithm for photoelectric volume pulse wave, proposes various pulse wave time-domain descriptive features, and establishes a set of three types of pulse wave descriptive features. Based on this, a classification model that can effectively identify preeclampsia is established based on machine learning related algorithms.
It is important for the early identification and detection of preeclampsia, and to ensure the safety of mothers and infants in the perinatal period.


\textbf{Key words}:preeclampsia; recognition model; photoelectric volume pulse wave; waveform detection algorithm; random forest