\cleardoublepage
\chapternonum{摘要}
子痫前期(Preeclampsia,PE)是孕妇妊娠期特有的一种多系统进展性疾病,可引起严重的母胎并发症,是孕产妇和围产儿病死率升高的主要原因。原发性高血压与蛋白尿是PE最突出的特点,此外,PE也可
引起全身小静脉痉挛。但现阶段对PE的诊断识别主要依赖于生化标志物有创检查,受操作流程、检测场所及成本等条件制约,缺乏对PE的便捷、快速、无创、准确的临床诊断手段。
光电容积脉搏波(Photoplethysmography, PPG)是心脏周期性搏动的体现,包含了有关人体血液微循环方面的更富细节的生理信息,可以反应血液动力学上的变化,具有快速、便捷、无创、采集设备已经普及等优点。
本论文从脉搏波信号的时域特征出发,构建了脉搏波时域描述特征集合,在此基础上进行了子痫前期识别模型的研究。


本文的主要内容及创新点如下:



综上,基于机器学习的PE研究可总结为。具有重要意义。


\textbf{关键词}:子痫前期;识别模型;光电容积脉搏波;波形检测算法;随机森林




\cleardoublepage
\chapternonum{Abstract}