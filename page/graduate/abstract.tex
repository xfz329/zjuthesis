\cleardoublepage
\chapternonum{摘要}
子痫前期(preeclampsia,PE)是孕妇妊娠期特有的一种多系统进展性疾病,可引起全身小静脉痉挛,导致严重的母胎并发症,
是孕产妇和围产儿病死率升高的主要原因之一。但现阶段对PE的诊断识别主要通过生化标志物检查,该过程受操作流程、检测场所及成本等条件制约。
本论文基于光电容积脉搏波(photoplethysmography, PPG)对PE的识别进行了研究。
PPG是心脏周期性搏动的体现,包含了人体血液微循环方面的丰富细节,具有无创、便捷等优点。

本论文构建了包含PE孕妇与正常妊娠孕妇的PPG数据集,设计了PPG多维度时域特征集与PPG采样序列时域特征集,将
被试PPG数据的单个波形与全波波形分别作为最小分析单位,使用决策树、K近邻与随机森林三种算法,进行了PE识别模型的研究工作。

本论文的主要内容及创新点如下:

一、基于策略与机制分离思想,提出并实现了PPG波形检测的初筛—复核—投票算法

本论文对PPG波形的检测过程进行了模式设计,提出了一种基于初筛—复核—投票(screening-checking-deciding,SCD)的PPG波形检测算法。
算法在初筛时,通过搜索窗的方法确定波形;在复核时,通过功率、标准差、波峰相对位置与基线漂移程度等PPG形态学特征,区分有效波形与异常干扰;
在决策时,通过加权投票的策略确定 PPG 波形是否为有效波形。SCD 算法可根据使用场景进行二次开发,支持更改初筛算法、增删复核标准与切换决策策略等设置。
经验证,该算法对PPG波形的识别准确率高、抗干扰能力强,可为多平台、多研究应用下的 PPG 分析检测的不同场景提供参考。

二、设计了多种新型PPG时域形态特征,构建了完整的PPG时域描述特征集合

本论文对PPG时域特征的设计过程进行了方法学的归纳,提出了PPG描述向量的概念,并从时间、幅值、角度、弧度、面积、斜率等维度设计了多种新型PPG时域特征,
通过左视策略、中视策略与分层策略构建了PPG多维度时域特征集。
基于PPG波形的原始采样序列,通过重采样策略、补齐策略,构建了PPG采样序列时域特征集。
此外,也提出了描述PPG波形间差异的新型时域特征。
这些研究工作为PPG的时域特征分析提供了新的思路,也为基于PPG的各类应用研究提供了普适通用的PPG时域特征集合。

三、使用多种机器学习算法构建了子痫前期的识别模型

经算法初筛,本论文将决策树、K近邻与随机森林三种算法作为PE识别模型的主要研究方法。
按照将被试PPG数据的单个波形与全波波形分别作为最小分析单位的两个研究角度,基于上述两个PPG时域特征集,使用决策树、K近邻与随机森林三种算法,进行了PE识别模型的研究。
结果表明,将PPG单个波形作为最小分析单位时,在PPG多维度时域特征集上,由随机森林算法可训练得到最佳模型(AUC为0.99,测试集准确率97.0\%);
在PPG采样序列时域特征集上,由随机森林算法可得到最佳模型(AUC为0.967,测试集准确率92.5\%)。
将被试PPG全部波形作为最小分析单位时,在PPG多维度时域特征集上,由随机森林算法可得到最佳模型(AUC为0.952,测试集准确率93.8\%);
在PPG采样序列时域特征集上,由随机森林算法可得到最佳模型(AUC为0.929,测试集准确率87.5\%)。

以上结果表明,本论文构建的PPG多维度时域特征集与PPG采样序列时域特征集可作为PE识别的数据基础,其中,各算法模型在PPG多维度时域特征集效果更好。
对比各算法模型在测试集上的准确率表明,随机森林算法性能优于决策树与K近邻算法。对两个特征集中有效特征贡献度分析表明,与PPG波形形态的对应关系的研究表明,
两类时域特征集中对随机森林模型贡献度高的特征所对应的PPG形态位置高度相似,且集中出现在PPG主波峰后与下降支末端附近,PPG波形在这些位置的具体形态可能是识别PE的关键。

\vspace{2em}

\textbf{关键词}: 子痫前期; 识别模型; 光电容积脉搏波; 波形检测算法; 随机森林


\cleardoublepage
\chapternonum{Abstract}

Preeclampsia (PE) is a multisystemic progressive disorder specific to maternal pregnancy that causes systemic small venous spasms 
leading to severe maternal-fetal complications. It is one of the major causes of increased maternal and perinatal morbidity and mortality.  
However, at this stage, the diagnostic identification of PE is mainly through biomarkers, a process constrained by operational procedures, 
testing sites, and costs. 
This thesis investigates the identification of PE based on photoplethysmography (PPG). PPG is a manifestation of the periodic pulsations 
of the heart and contains a wealth of details about the microcirculation of human blood, which has the advantages of being non-invasive and convenient.

In this thesis we constructed a PPG dataset containing pregnant women with PE and normal pregnancy, designed a multidimensional time-domain feature set of PPG and 
a time-domain feature set of PPG sampling sequence, and used three algorithms: decision tree (DT), K-nearest neighbor (KNN), and random forest (RF), 
regarded the single waveform and all the waveforms of the subject PPG data as the minimum unit of analysis respectively.

The main contents and innovations of this thesis are as follows.

I. A PPG waveform detection algorithm based on screening-checking-deciding is proposed and implemented, based on the idea of separation of strategy and mechanism.

In this thesis, we propose a PPG waveform detection algorithm based on screening-checking-deciding (SCD). The algorithm determines the waveform by searching the window 
in the initial screening; in the checking-review, it distinguishes the valid waveform from the abnormal interference by the PPG morphological features such as power, standard deviation, 
relative positions of wave peaks, and baseline drift; in the decision, it determines whether the PPG waveform is a valid waveform by the weighted voting strategy. 
The SCD algorithm can be developed according to the usage scenarios and supports settings such as changing the initial screening algorithm, adding and deleting criteria, 
and switching decision strategies. The algorithm is proven to have high accuracy and strong anti-interference capability for PPG waveform identification, 
which can provide a reference for different scenarios of PPG analysis and detection under multi-platform and multi-study applications.

II. A variety of new PPG time-domain morphological features are designed to build two complete sets of PPG time-domain description features.

In this thesis, the design process of PPG time-domain features is summarized methodologically, the concept of PPG description vector is proposed, 
and a variety of novel PPG time-domain features are designed in multiple dimensions such as time, amplitude, angle, radian, area, slope, etc. 
The set of PPG multi-dimensional time-domain features is constructed by left-view strategy, center-view strategy, and scaled-view strategy. 
Based on the original sampling sequence of PPG waveform, the time domain feature set of PPG sampling sequence is constructed by resampling strategy and 
complementary strategy. In addition, novel time-domain features describing the differences between PPG waveforms are also proposed. 
These research works provide new ideas for the anlysis of time-domain features of PPG, and also provide universal PPG time-domain feature sets for various PPG-based application studies.

III. The pre-eclampsia recognition models are constructed using various machine-learning algorithms.

After the initial screening of algorithms, three algorithms, namely DT, KNN, and RF, are used as the main research methods 
for PE recognition models in this thesis. Based on the two PPG time-domain feature sets, decision tree, K-nearest neighbor, and RF algorithms were 
used to study the PE recognition model from the perspective that individual waveforms and full waveforms of the subject PPG data were used as the minimum analysis unit. 
The results show that the best model can be trained by the RF algorithm (AUC of 0.99 and test set accuracy of 97.0\%) on the PPG multidimensional time-domain 
feature set, and the best model can be obtained by the RF algorithm (AUC of 0.967 and test set accuracy of 92.5\%) on the PPG sampling sequence time-domain 
feature set when the single PPG waveform is used as the minimum unit of analysis. The best model can be obtained by the RF algorithm (AUC 0.952, 
test set accuracy 93.8\%) for the PPG multidimensional time-domain feature set, and the best model can be obtained by the RF algorithm (AUC 0.929, 
test set accuracy 87.5\%) for the PPG sampling sequence time-domain feature set when all the PPG waveforms are used as the minimum analysis unit.

The above results show that the PPG multi-dimensional time-domain feature set and the PPG sampling sequence time-domain feature set constructed in this thesis 
can be used as the data basis for PE identification, among which, each algorithmic model works better in the PPG multidimensional time-domain feature set.
Comparing the accuracy of each algorithmic model on the test set, the RF algorithm outperforms 
the decision tree and K-nearest neighbor algorithms. The analysis of the effective feature contributions in both feature sets showed that the correspondence 
with the PPG waveform morphology showed that the locations of the PPG morphology corresponding to the features with high contributions to the RF 
model in both time-domain feature sets were highly similar and concentrated near the end of the descending branch after the peak of the main PPG wave, 
and the specific morphology of the PPG waveform at these locations may be the key to identify PE.

\vspace{2em}

\textbf{Keywords}: preeclampsia; recognition model; photoplethysmography; waveform detection algorithm; random forest