\cleardoublepage
\chapternonum{致谢}

在完成毕业论文之后,终于有心情来整理这一路走来的成长与收获。在此向一路相随相伴的师长、同窗、朋友以及家人们
致以最诚挚的谢意。

感谢母校浙江大学给予的学习机会和资源。年少时的向往终于在研究生期间梦想成真。我会永远记得2014年夏天第一次邂逅玉泉时的
激动与羡慕。非常自豪能在研究生期间成为浙江大学的孩子,希望能在毕业后也能担得起“浙大人”的称呼。

感谢我敬爱的导师陈杭教授在研究生期间给予我的悉心教导与无私关怀!
陈老师治学严谨,对于我在科研学术之路上循循善诱,传我以道、授我以业、解我之惑。
整个研究与论文撰写过程都是在老师耐心的指导下完成,从论文选题、实验数据获取到论文框架确定、遣词造句的雕琢,
老师均事无巨细地反复给予指导。不仅如此,老师对于我在生活中遇到的问题也给及了关心与帮助,
在为人处事、待人纳物等等方面都为我树立了标杆,
老师的敦敦教诲会让自己受益终生。

感谢课题组的合作老师陈新忠教授,陈老师从医学的角度为我提供了许多
研究方向和思路,每次向陈老师请教问题困惑时,都能得到老师的热情教导与
殷切鼓励。陈老师身上体现的修身、齐家、治学、济天下的医者仁心,会一直激励着我一路前行。

感谢生仪学院思政科、研究生科的老师们。

独学则无友,孤陋则寡闻。感谢求学期间遇见、相交、结识的师兄弟、同学及好友们。
感谢蒋凯、黄超、汪洋三位师兄在科研与学术上的指导。特别感谢刘梦星师兄,
在我有任何困惑与需要时给予无私的关心与帮助,刘梦星师兄永远是我的偶像。
感谢实验室李杰、由佳、吴灶全、刘龙、王梦婷、侯冲等师兄师姐;感谢江河、孙玉彤、高欢等同侪同窗;感谢陈婉琳、
王亚露、缪家俊、陈沙利、朱志斌、张梦鸽、冯静雯、徐艺菲等师弟师妹。和大家朝夕相处、共同求学的经历会是我一辈子的美好回忆。
感谢张彪、马济通、林家伟、黄橙、刘晨东、龚晴、陈翔、王福园、陈璟、史理全、魏古月、方正刚、徐星庐等好友,我会珍藏和大家的欢乐时光。


最后感谢我的家人,勤劳而朴实的父母和我最亲爱的姐姐,是你们给予我最最无私的关心与支持。
感谢父母一路心血养育了我、教育了我,为我的成长环境、学习环境
一路保驾护航。感谢姐姐永远在我身后支持着我、理解着我、关爱着我。希望我们一家人永远幸福!

最后感谢所有帮助过我的老师、同学、朋友们。感谢无数前人先贤的智慧。也感谢求学期间给予自己无数启发的未知姓名的网络博主们。
希望本论文也能于人有益。


\rightline{江锋}
\rightline{2022年秋}
\rightline{于玉泉}