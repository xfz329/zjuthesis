\cleardoublepage
\chapternonum{缩写词表(页码序)}
% \footnote{所有缩略词按其英文缩写字母序排列。}
\begin{center}
    % uncomment line 5 if you want to set the fontsize of the table.
    % \zihao{-4}
    % comment line 8 and uncomment line 9 if you need to set the table as mid-aligment.
    % or addjust the paras as your personal need.
    \begin{longtable}{m{2cm}m{7cm}m{5cm}m{1cm}<{\centering}}
    % \begin{longtable}{m{2cm}<{\centering}m{8cm}<{\centering}m{5cm}<{\centering}} 
        \topline
        \textbf{英文缩写}&\textbf{英文全称}&\textbf{中文全称}&\textbf{页码}\\
        \midline
        \endfirsthead
        \topline
        \textbf{英文缩写}&\textbf{英文全称}&\textbf{中文全称}&\textbf{页码}\\
        \midline
        \endhead 
        \bottomline
        \endfoot
        \bottomline
        \endlastfoot
        PE      &       preeclampsia                                    &   子痫前期、先兆子痫      &   1   \\
        HDP     &       hypertension disorders of pregnancy             &   妊娠期高血压疾病        &   1    \\
        BP     &        blood pressure                         &   血压                 &    1   \\
        SBP     &       systolic blood pressure                         &   收缩压                 &    1   \\
        DBP     &       diastolic blood pressure                         &   舒张压                 &    1   \\
        MAP     &       maximum a Posteriori                         &   极大后验                 &    5   \\
        UPTI     &       uterine artery pulsatility index                         &   子宫动脉搏动指数                &    5   \\
        PAPP­-A     &       pregnancy associated plasma protein A                         &   妊娠相关血清蛋白A                 &    5   \\
        PLGF     &       placental growth factor                         &   胎盘生长因子                 &    5   \\
        MAP     &   mean arterial pressure                              & 平均动脉压 & 5 \\
        RI      &   resistance index & 阻力指数 & 6\\
        SDR     &  systolic to diastolic ratio & 收缩压/舒张压比 & 6\\
        UE3     & unconjugated estriol & 游离雌三醇 & 7\\
        AFP     & alpha fetoprotein     & 甲胎蛋白 & 7 \\
        HCG     & human chorionic gonadotropin  & 人绒毛膜促性腺激素    & 7 \\
        sFlt-1  & soluble fms like tyrosine kinase 1    & 可溶性fms样酪氨酸激酶-1 & 7\\
        ML     & machine learning  & 机器学习  & 10    \\
        GP  & genetic programming   & 基因遗传算法  &10 \\
        ROC & Receiver Operating Characteristic & 受试者工作特征 \\
        AUC & area under curve & 曲线下面积 & 10 \\
        NN & neural networks & 神经网络 & 10 \\
        DL & deep learning & 深度学习 & 10 \\
        PSO & particle swarm optimization & 粒子群优化 & 10 \\
        LOO & leave one out & 留一法 & 10 \\
        DR  & dimension reduction & 降维 & 11 \\
        TSS & time-series summary & 时间序列总结 & 11 \\
        EN & elastic net  & 弹性网算法 & 11 \\
        GB &  gradient boosting  & 梯度提升算法 & 11 \\
        PPG & photoplethysmography & 光电容积脉搏波 & 14 \\


        SPV & systolic pressure variation & 收缩压变化参数 & 34 \\
        PPV & pulse pressure variation & 脉压变化参数 & 34 \\
        CA  & correlation analysis & 相关性分析 & 36 \\
        PT  & parametric test & 参数检验 & 37 \\
        NPT & non-parametric test & 非参数检验 & 37 \\
        CF  & cut-off frequency & 截止频率 & 40 \\
        SNR & signal-noise rate & 信噪比 & 40 \\
        RMSE & root mean square error & 均方误差 & 40 \\



        
        RF&random forest&随机森林\\
        SVM&support vector machine&支持向量机\\
        MLP&multilayer perceptron&多层感知机\\
        PPGFV &photoplethysmographic feature vector&  脉搏波特征描述向量 & 65\\
        PPGMTFS &     photoplethysmographic multidimensional time-domain feature set & 脉搏波多维度时域特征集 & 73 \\
        PPGSTFS &     photoplethysmographic sample-value-based time-domain feature set & 脉搏波采样值时域特征集 & 73 \\
        
        LVS & left view strategy & 左视策略 & 75 \\
        CVS & center view strategy & 中视策略 & 75 \\
        SVS & scaled view strategy & 分层策略 & 75 \\
        RS & resampling strategy & 重采样策略 & 78 \\
        CS & complementary strategies & 补齐策略 & 78 \\
        IS & interception strategy & 截取策略 & 78 \\
        PR & pulse-based research & 基于波形的研究 & 80 \\
        SR & subject-based research &基于被试的研究 & 80 \\

        GI & Gini index & 基尼指数 & 87 \\
        IG & information gain & 信息增益 & 87 \\
        IE & information entropy & 信息熵 & 88 \\
        ID3 & iterative dichotomiser 3 & 第三代迭代二分器算法 & 88 \\
        C4.5 & classifier 4.5 & 第4.5代分类器算法 & 88 \\ 
        CART & classification and regression tree & 分类与回归树 & 88 \\
        CM & confusion matrix & 混淆矩阵 & 91 \\
        TP &  true positive & 真阳性 & 91 \\
        FP & false positive & 假阳性 & 91 \\
        TN & true negative & 真阴性 & 91 \\
        FN & false negative & 假阴性 & 91 \\
        TPR &  true positive rate & 真阳性率 & 92 \\
        FPR & false positive rate& 假阳性率 & 92 \\
        YI  & Youden index & 约登指数& 93 \\
        SGD & stochastic gradient descent & 随机梯度下降算法 & 94 \\
        KNN & K-nearest neighbors & K近邻算法 & 94 \\
        GNB & Gaussian plain Bayes & 高斯朴素贝叶斯算法 & 94 \\
        LR  & logistic regression  & 逻辑回归算法 & 94 \\
        LSVM & linear support vector machines & 线性支持向量机算法 & 94 \\
        KSVM & kernel support vector machines & 核支持向量机算法 & 94 \\
        CSVM & C-support vector machines & C-支持向量机 & 94 \\
        MLP & multilayer perceptron & 多层感知机 & 94 \\
	\end{longtable}
\end{center}