\chapter{低耦合、高拓展的软件综合分析系统的设计与实现}
\section{引言}
良好的软件设计应该支持软件自身更新迭代导致的各种变化\cite{Enrich2018}。随着本研究的不断深入,软件涉及的数据管理任务不断增加、需要管理维护的特征等算法不断开发、迭代、对PE进行识别判断的模型不断训练得到,这些内容都属于软件设计中
的变化。对这些内容进行有效管理需要我们设计好可以适应这些变化、具有健壮性的软件分析系统(Software comprehensive analysis system,SCAS)。同时,良好的软件设计也应该使得软件系统对类似的基于脉搏波等生理信号的各项研究有良好的兼容性与可移植性。
因此,本章着重从软件设计的角度阐述项目中涉及的具体问题及解决应对措施上的多种设计。

% 为PE分析研究而生,但不止服务于PE分析课题\cite{Enrich2018}
\section{需求分析}
一、核心分析功能

从原始生理信号中完成数据清洗、特征提取、分类模型训练和生成及最后对新数据进行有效分类是对软件分析系统最基本要求,也是软件系统应具有的核心功能。而在此功能完成的基础上,当软件系统中各个部件、模块因为种种原因
引入某种变化之后,需要根据具体情景针对该特定的变化引入代码的模式设计与重构以增加软件系统的复用性、健壮性。

二、数据源兼容

软件分析系统的使用的原始数据可能来自多种硬件采集设备自行采集或公开数据库。不同硬件设备使用的电子元器件、传感器等在性能参数上的区别可能导致PPG(在内的)数据采集质量、采样率有所差异。此外,不同的硬件厂商的导出数据存储格式也往往不尽相同。
而源自数据库的数据在字段信息上更容易出现差异。保持对不同格式数据读取的兼容性是软件分析系统需要首先解决的问题。

三、信号扩展

目前软件分析系统仅使用了1通道的光电容积脉搏波PPG数据。但由于数据采集设备性能的增强,多通道多参数生理信号的获取不存在任何技术性障碍。因此,软件分析系统需要为以后可能的多通道多生理参数的联合分析预留好设计接口,避免出现大规模的代码重构现象。

四、算法迭代与更新

在软件综合分析系统的各个环节均会使用到相应的处理算法,如数据预处理时的滤波检波算法、特征提取时的各种新型参数的计算算法以及模型训练时用到的多种机器学习模型。当需要对这些算法进行迭代更新时,特别是计算特征时又提出了新的参数,如果没有恰当的软件模式设计
,软件代码将不可避免地出现大量修改,从而影响系统整体的健壮性与稳定性。

五、数据管理

从核心功能分析可以看出,本软件系统不仅需要对原始数据进行管理处理,还需要对基于原始数据的参数特征进行管理,甚至最终生成的模型也是基于数据集及数据集中使用的具体参数的规模。如何对这些数据进行有效管理以及如何在不影响程序功能的情况下,增加对新数据的支持,也是
软件系统需要解决的问题。此过程涉及数据的传递与保存、数据库设计及管理与模型更新与管理等具体问题。

六、多场景多平台

软件分析系统需要考虑到多种潜在的应用场景,满足在实验室研究、医院监护、社区检查、居家监护多种场景下不同的使用需求。同时由于智能化设备的不断普及,软件分析系统在设计时也应该考虑到从PC端软件到移动端应用的移植与扩展的可能。

七、研究课题的扩展性

从核心功能分析可以看出,软件分析系统对PE的分析处理过程——数据清洗、特征提取、分类模型训练和生成——可以进一步抽象成一类基于人体电生理信号的分类分析过程。因此,软件分析系统在设计时应充分考虑使用将特定功能模块化,以方便对类似研究的进行扩展开发。

\section{总体方案设计}
根据上小节软件需求分析,最终SCAS设计了一下模块:

预处理部分使用Java完成所有的功能。
数据以Json的格式上传至服务器进行管理。
由于Python在机器学习领域的广泛普及,机器学习方面所有的编程设计与实现全部基于Python完成。
因此,在服务器端最后选择了Django+Python的形式完成了所有功能。

策略与机制分离
检测算法、特征算法
重构 满足更多需求
软件易于复用

* 实验室已有前期硬件研究基础,家俊研究基础

\subsection{总体方案}
\subsection{功能架构}
\subsection{技术架构}
\subsection{整体设计框图展示}

\section{各模块具体实现}
模块化、低耦合、高拓展

* 硬件采集支持

* 对数据源的拓展——多种数据格式均可兼容

* 对数据预处理的拓展——波形定位算法可拓展、纠错算法可拓展

* 对数据类型的拓展——目前只涉及脉搏波,但可方便拓展ECG等信号

* 对特征集构建拓展——方便随时研发新的特征参数进入特征集

\section{系统验证与测试}
界面
输入输出
\section{小结}