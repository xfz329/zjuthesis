\chapter{低耦合、高拓展的软件综合分析系统的设计与实现}
\section{引言}
随着本研究的不断深入,研究过程中涉及的各项软件及算法不断迭代更新,如何有效的对其进行设计、管理、迭代是一项重要内容。由于本研究涉及的内容从最贴近硬件系统的基本数据
采集到各项基于脉搏波的生理参数计算提取、再到基于各种机器学习算法的模型训练,各环节产生的众多数据管理也需要进行有效管理。同时,良好的软件设计也应使得本软件系统对类似的
基于脉搏波的各项基础研究有较好的兼容性。本章则着重从设计模型及实现角度阐述本研究工作。
\section{背景分析}
目前基于光电容积脉搏波的数据针对子痫前期进行分析和预测的综合分析系统还处于待发展阶段。但随着
* 子痫前期检测的特定场景:孕妇、高龄、周期长、时间

* 实验室已有前期硬件研究基础,家俊研究基础

\section{需求分析:模块化、低耦合、高拓展}
1. 有效管理数据
2. 硬件兼容
3. 特征提取算法计算可迭代
4. 预测模型可迭代更新
4. 多场景
5. 模型训练及更新

\section{设计及实现}
\subsection{总体方案}
\subsection{功能架构}
\subsection{技术架构}
\subsection{整体设计框图展示}
\subsection{接口设计}
\section{各模块具体实现}
* 硬件采集支持

* 对数据源的拓展——多种数据格式均可兼容

* 对数据预处理的拓展——波形定位算法可拓展、纠错算法可拓展

* 对数据类型的拓展——目前只涉及脉搏波,但可方便拓展ECG等信号

* 对特征集构建拓展——方便随时研发新的特征参数进入特征集

\section{系统测试}
界面
输入输出
\section{小结}