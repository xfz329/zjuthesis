\chapter{脉搏波的预处理与特征参数提取}
\section{引言}
本章将从PPG信号的预处理的角度对本研究过程中的相关工作进行介绍。
本章首先介绍本研究使用的所有PPG数据来源自自主试验数据,从试验设备、试验流程与数据导出等方面对数据采集方案进行了介绍。
其次,利用常用的变量分析方法,统计分析数据试验涉及的所有被试的人口统计学特征。
随后,从信号滤波、波形检测、切迹检测、基线漂移处理、信号重采样与数据标准化等过程对PPG的预处理分析流程进行介绍。特别地,提出一种基于初筛-复核-决策的新型PPG波形检测算法。
最后,汇总介绍PPG常见的时域特征参数。

本章研究内容的框架图如\autoref{fig:frameworks3}所示。

\begin{figure}[htbp]
    \centering
    \includegraphics[width=0.7\linewidth]{pulse_preprocess/frameworks3}
    \caption{\label{fig:frameworks3}第三章研究内容框架}
\end{figure}
\section{数据采集方案设计}
本研究于2017年6月至2019年4月期间,于浙江大学医学院附属妇产科医院进行了数据采集试验。其中,患有PE的孕妇作为此次数据试验的试验组,而正常妊娠孕妇作为对照组。
此次数据采集试验经浙江大学医学院附属妇产科医院医学伦理委员会的审查并被批准进行(审查编号:20170131)。所有参与试验的孕妇均知晓本研究目的及具体试验流程的情况,并签署了知情同意书。
需特别引起注意的是,所有实验组孕妇均已确诊PE,出现不同程度的血压升高在内的临床症状。因此,\textbf{实验组所有患有PE的孕妇均已接受过药物治疗在内的临床干预以减轻其症状}。
\subsection{试验设备}
本研究对脉搏波数据采集仪器的专业性、可靠性都有较高要求。在考察试验场地(浙江大学医学院附属妇产科医院麻醉科室)的诸多临床在用设备后,选取了美国GE公司生产的GE Healthcare CARESCAPE B650 麻醉监护仪进行数据采集。
B650监护仪可对包括12导联心电、脑电、心输出量、血氧饱和度、血压、呼出气体中$CO_{2}$与$O_{2}$成份及熵指数等专业参数进行监测,如\autoref{fig:monitor}所示。此外,B650监护仪还提供了基于PPG的表征动脉血压变化的
收缩压变化参数(systolic pressure variation,SPV)和脉压变化参数(pulse pressure variation,PPV)\cite{GE2021,Michard1999}。
\begin{figure}[htbp]
    \centering
    \subfigure[B650监护仪及其配件]{
    \includegraphics[width=7cm]{data_plan/monitor1}
    }
    \quad
    \subfigure[B650监护界面]{
    \includegraphics[width=5.5cm]{data_plan/monitor2}
    }
    \caption{\label{fig:monitor}GE Healthcare CARESCAPE B650监护仪}
\end{figure}

\subsection{试验流程}
被试孕妇到达数据试验采集现场后,会由试验人员登记被试人员的PE风险因子信息,包括姓名、年龄、孕产史等。接着试验人员会告知被试人员本研究的研究背景与试验过程。
在征得被试人员同意后,采集人员会按前文操作规范为被试佩戴血氧采集探头\cite{Chen2021}。与血压采集过程类似\cite{FIGO},被试人员需在数据采集试验全程保持坐姿,
在被试孕妇休息至少5分钟后,PPG数据开始正式记录,单次采集时长不低于1分钟。
通过被试孕妇的左手食指,使用透射式血氧探头进行PPG数据采集。若被试孕妇的左手食指有损伤,则将测量部位替换为左手中指。手指在测量时需放入指套内部,
指甲与传感器表面有指甲标记的部位正对,指尖触及但不超出指套顶端,确保发光管发出的所有光线全部通过被试的组织。使用的血氧采集探头美国泰科公司旗下的Nellcor DS-100A型血氧传感器。 

\subsection{数据导出及复核}
由于GE公司并没有公开B650数据通讯协议,无法直接获取该设备可采集的全部生理参数的原始数据信息。通过产商技术支持提供的第三方软件,可将B650监护仪采集得到的PPG数据以
时间-相对幅值的键值对的形式以按CVS的文件格式导出,导出的PPG信号采样率为100$Hz$。

本研究共完成了80例被试孕妇的PPG原始数据采集工作。经复核校验,有1例正常妊娠孕妇数据由于采集时间过短、信号质量过低等原因被剔除。
剩余79例有效PPG数据作为本文后续所有分析的数据源,其中实验组数据44例,对照组数据35例。
\section{被试孕妇人口统计学特征分析}
本节使用统计分析的方法对上述试验中被试孕妇的人口统计学特征进行了相关性分析,具体包括被试的年龄、孕周、身高、体重、BMI指数、血压及心率等。
\subsection{变量分析方法}
相关性分析时统计分析学科中最常用的工具之一。依据每种方法的适应性不同,本研究选取了其中的非参数方法对上述变量完成最后的分析。

一、相关性分析

相关性分析(correlation analysis,CA)是指对两个具备相关性的变量元素进行分析,衡量两个变量因素的相关密切程度\cite{Zhang2019}。相关性的元素之间需要存在一定的联系或者概率才可以进行相关性分析。

统计学上引入了相关系数$r$以量化表征变量之间的密切程度。$r$的取值范围为[-1,1],其绝对值大小反映了两变量之间相关性的强弱:当$r$>0时,表明两个变量正相关;反之,则两个变量变化趋势相反,呈负相关。
一般可以通过散点图即可直接观察变量之间的联系,如\autoref{fig:relation}所示。\autoref{fig:relation}也在每副子图中标注了对应的$r$值。

\begin{figure}[htbp]
    \centering
    \includegraphics[width=.6\linewidth]{data_plan/relation}
    \caption[二元变量之间常见的相关关系]{\label{fig:relation}二元变量之间常见的相关关系\cite{IXL2022}}
\end{figure}

常用的相关性分析方法有皮尔逊(Pearson)相关性分析法与斯皮尔曼(Spearman)相关性分析法,利用这两种方法计算得到的相关系数也对应被称为皮尔逊相关系数$r_p$与斯皮尔曼相关系数$r_s$:
\begin{equation}
    \label{equ:spearman}
    r_s=1-\frac{6\sum_{i=1}^{n}(x_{i}-y_{i})^2}{n(n^2-1)}
\end{equation}
\begin{equation}
    \label{equ:pearson}
    r_p=\frac{\sum_{i=1}^n{(x_i- \mathop{x} \limits^-)(y_i- \mathop{y} \limits^-)}}{\sqrt{{\sum_{i=1}^n}{{(x_i- \mathop{x} \limits^-)^2\sum_{i=1}^n}{(y_i- \mathop{y} \limits^-)^2}}}}
\end{equation}

一般认为,斯皮尔曼相关性分析适用于对存在单调性关系的变量进行检测,而皮尔逊相关性分析适用于对正态分布的变量进行检测。
由于斯皮尔曼相关系数的计算对变量的分布特性要求并不严格,应用得也更为广泛。
为计算斯皮尔曼相关系数,需要对长度为$n$的待检二元变量$X$与$Y$按升序排列,得到原始数据在排序后的序次$x$、$y$。特别地,若出现多个数据排序相同,则用这些数据的平均序次统一表征后再进行计算,如\autoref{equ:spearman}所示。

二、参数检验与非参数检验

数据的集中趋势、离散程度与分布形态是对一组数据进行描述时最常用的三个角度\cite{Hu2021}。而参数检验(parametric test,PT)通常都是在假设数据总体服从正态分布、样本统计量服从T分布的前提下,对总体分布中的总体均值、方差及样本差等未知参数做出统计推断。
而面对总体分布类型未知或分布类型已知,但不对称或变量无法精准测量的数据,或样本容量小、无法运用中心极限定理进行相关参数检验时,
另一类不以特定的总体分布为前提、不针对总体分布参数做任何推断的分析方法也发展起来,此类分析方法被统一称为非参数检验(non-parametric test,NPT)\cite{Guo2017,Hu2021,Zhang2019}。

一般而言,参数检验的精确度高于非参数检验,因此在条件允许的情况下,应优先采用参数检验。若由于各种原因导致参数检验的条件不满足,可以应用非参数检验方法对数据进行分析。
常见的非参数检验方法及其适用情形如\autoref{tab:nonparametric-test}所示。
\begin{center}
    \zihao{-5}
    \begin{longtable}{m{1.8cm}<{\centering}m{2cm}<{\centering}m{5cm}<{\centering}m{5cm}<{\centering}}
		\caption{常见的非参数检验方法}\\
		\label{tab:nonparametric-test}\\
		\topline
        \colorhead\textbf{样本数目}&\textbf{样本相关性}&\textbf{检验方法}&\textbf{检验方法英文名}\\
        \midline
        \endfirsthead
        \caption[]{(续)}\\
        \topline
        \colorhead\textbf{样本数目}&\textbf{样本相关性}&\textbf{检验方法}&\textbf{检验方法英文名}\\
        \midline
        \endhead 
        \hline
        \endfoot
        \bottomline
        \endlastfoot
        \colorrowa 单样本   & /     & 卡方检验  & Chi-Squared Test \\
        \colorrowc 单样本   & /     & 二项分布检验 & Binomial Test \\
        \colorrowa 单样本   & /     & K-S检验 & Kolmogorov–Smirnov Test \\
        \colorrowc 单样本   & /     & 符号秩检验 & Wilcoxon Signed-Rank Test \\
        \colorrowa 单样本   & /     & 游程检验  & Wald–Wolfowitz runs Test \\
        \colorrowc 两样本   & 独立    & Wilcxon W等级和检验 & Mann-Whitney U Test \\
        \colorrowa 两样本   & 独立    & 摩西极端反映差异检验 & Moses Extreme Reaction Test \\
        \colorrowc 两样本   & 独立    & K-S检验 & Kolmogorov–Smirnov Test \\
        \colorrowa 两样本   & 独立    & 游程检验  & Wald–Wolfowitz runs Test \\
        \colorrowc 两样本   & 相关    & 符号检验  & Sign Test \\
        \colorrowa 两样本   & 相关    & 符号秩检验 & Wilcoxon Signed-Rank Test \\
        \colorrowc 两样本   & 相关    & 变化显著性检验 & McNemar's Test \\
        \colorrowa 两样本   & 相关    & 边缘一致性检验 & Marginal Homogeneity Test \\
        \colorrowc 多样本   & 独立    & K-W平均秩检验 & Kruskal-Wallis H Test \\
        \colorrowa 多样本   & 独立    & 中位数检验 & Median Test \\
        \colorrowc 多样本   & 独立    & 分组分布检验 & Jonckheere-Terpstra Test \\
        \colorrowa 多样本   & 相关    & 双向等级方差分析 & Friedman Test \\
        \colorrowc 多样本   & 相关    & 肯德尔和谐系数检验 & Kendall's W Test \\
        \colorrowa 多样本   & 相关    & 二分变量检验 & Cochran's Q Test \\
    \end{longtable}
\end{center}
\vspace{-0.8cm}

\subsection{分析结果}
由于人口统计学特征、PE风险因子等待检参数存在样本量较小、具体分布未知等客观限制因素,满足U检验的适用条件。
本论文最终选取\autoref{tab:nonparametric-test}中的中的Wilcxon W等级和检验,亦即Mann-Whitney U检验对被试的人口统计学特征、PE风险因子在实验室与对照组上
是否存在分布差异进行相关性分析。

U检验的基本思想是将全部样本混合后一起求秩,然后根据两组样本的秩分情况判断是否存在差异。
被试孕妇的人口统计学特征、PE风险因子经U检验后的统计结果如\autoref{tab:factors_res}所示,其中,所有待分析变量均以平均值±标准差的形式表征,组别之间存在分布差异的变量
按粉红色底色、黑色加粗字体的格式进行了突出显示。

\begin{center}
    \zihao{-5}
    \begin{longtable}{m{4cm}<{\centering}m{4cm}<{\centering}m{4cm}<{\centering}m{2cm}<{\centering}}
		\caption{被试孕妇风险因子统计结果}\\
		\label{tab:factors_res}\\
		\topline
        \colorhead\textbf{检验变量(单位)}      & \textbf{实验组(n=44)} & \textbf{对照组(n=35)} & \textbf{p值} \\
        \midline
        \endfirsthead
        \caption[]{(续)}\\
        \topline
        \colorhead\textbf{检验变量(单位)}      & \textbf{实验组(n=44)} & \textbf{对照组(n=35)} & \textbf{p值} \\
        \midline
        \endhead 
        \hline
        \endfoot
        \bottomline
        \endlastfoot
        \colorrowa 年龄(years) & 32.3±3.6 & 33.8±4.6 & 0.108 \\
        \colorrowc 孕周(weeks) & 32.7±3.8 & 34.3±4.3 & 0.053 \\
        \colorrowa 身高(cm) & 158.1±5.0 & 160.0±3.3 & 0.089 \\
        \colorrowb \textbf{体重(kg)} &  \textbf{75.7±12.9} &  \textbf{67.1±8.2} & \textbf{0.002* }\\
        \colorrowb \textbf{BMI(kg/cm)} &  \textbf{30.2±4.5} &  \textbf{26.2±3.3} & \textbf{<0.001*} \\
        \colorrowb \textbf{收缩压(mmhg)} &  \textbf{160.1±19.5} &  \textbf{111.2±9.8} & \textbf{<0.001*} \\
        \colorrowb \textbf{舒张压(mmhg)} &  \textbf{96.1±14.5} &  \textbf{66.8±10.4} & \textbf{<0.001*} \\
        \colorrowc 心率(bpm) & 87.0±11.9 & 87.5±13.1 & 0.656 \\
    \end{longtable}
\end{center}
\vspace{-1cm}

从\autoref{tab:factors_res}可以发现,实验组与对照组的被试孕妇在年龄、孕周、身高、心率等风险因子
上均无明显统计意义上的差别(p>0.05),但在收缩压与舒张压的数值上有显著差异(p<0.001),这也与被试孕妇的PE病发状态一致,符合预期。

\section{脉搏波的预处理}
不同硬件设备采集的PPG数据在采样率、采样精度及导出的数据格式等方面存在着不小的差异,同时这些设备对数据采集过程中出现的人体肌电、呼吸与体动以及环境工频等噪声干扰的软硬件滤波处理能力也不尽相同。
因此,在具体的研究与应用场景下,在对PPG进行分析前,通常需要进行一定的预处理。
\autoref{fig:samplesignal}展示了此前数据试验采集得到的一段有代表性的数据波形。

\begin{figure}[htbp]
    \centering
    \includegraphics[width=\linewidth]{pulse_preprocess/samplesignal}
    \caption{\label{fig:samplesignal}一名被试实采PPG信号(片段)}
\end{figure}

从\autoref{fig:samplesignal}中可以看到,数据试验经由GE B650监护仪采集得到的PPG信号质量较高,但仍然存在着基线偏移、PPG信号重博波特征不明显等问题。这些问题可能与实验使用的传感器种类、监护仪硬件检测电路及监护仪软件处理算法等因素有关。
除此之外,在\autoref{fig:samplesignal}中,40 s至55 s内存在着一段干扰无效数据,在进行分析前必须设法对其进行剔除。

故本节对PPG信号的分析预处理过程进行介绍,包括滤波处理、波形检测、去除基线漂移、样条插值与数据标准化等,如\autoref{fig:process}所示。同时说明上述过程中涉及的具体算法方案及技术手段。

\begin{figure}[htbp]
    \centering
    \includegraphics[width=\linewidth]{pulse_preprocess/process.pdf}
    \caption{\label{fig:process}信号预处理流程示意}
\end{figure}

\subsection{信号滤波}
数字滤波可从原始信号中特定频段的成分,是信号处理的常见的处理方法。而在人体电生理信号分析领域,数字滤波更是必不可少的处理步骤。针对人体各电生理信号,此前的诸多学者们提出了多种不同的滤波算法。
在本研究中,由于如\autoref{fig:samplesignal}所示的原始PPG信号质量高、干扰噪声较少,仅使用较为简单的滑动平均滤波器对其进行了处理。

平滑滤波器本质上是一个低通滤波器,若以$X$表示原始信号,$Y$表示滤波后信号,$N$表示其滤波阶数,则平滑滤波过程可表示为
\begin{equation}
    \label{equ:filter}
    Y(k)=\frac{1}{N}\sum_{i=0}^{N-1}X(k+i)
\end{equation}
而该滤波器的截止频率(cut-off frequency,CF)与滤波阶数$N$存在以下关系\cite{malp2011,malp2022}
\begin{equation}
    \label{equ:malpf}
    f_{co} \approx 0.443 \cdot \frac{f_s}{N}    
\end{equation}
其中,$f_s$为原始信号采样率。由\autoref{equ:malpf}可知,滤波器的阶数越高,其截止频率越低,滤波后信号越均匀平滑,滤波效果越好。

\begin{figure}[htbp]
    \centering
    \subfigure[原始PPG信号]{
    \includegraphics[width=6.5cm]{pulse_preprocess/before}
    }
    \quad
    \subfigure[平滑滤波后PPG信号]{
    \includegraphics[width=6.5cm]{pulse_preprocess/after}
    }
    \caption{\label{fig:filter}平滑滤波前后PPG信号对比}
\end{figure}

经滑动滤波前后的PPG信号对比如\autoref{fig:filter}所示,其中滤波阶数$N$被设置为5,对应的的截止频率约为89$Hz$。
另一方面,信噪比(signal-noise rate,SNR)与均方误差(root mean square error,RMSE)是最常用的评估滤波效果的两个指标,其定义分别为
\begin{equation}
    \label{equ:snr}
    SNR=10 \cdot \log_{10}\frac{\sum_{i=1}^{n}{(X_i-\mathop{X} \limits^-})^2}{\sum_{i=1}^{n}{(X_i-Y_i})^2}
\end{equation}
\begin{equation}
    \label{equ:rmse}
    RMSE=\sqrt{\frac{\sum_{i=1}^{n}{(X_i-Y_i})^2}{n}}
\end{equation}
通过SNR与RMSE评估\autoref{fig:filter}中的效果,可得$SNR=33.19$,$RMSE=6.27$。
这说明PPG信号波形特征已比较清晰,因此,滤波阶数$N$为5的平滑滤波器可满足后续分析需求。

\subsection{波形检测}
准确地检测PPG波形是分析研究PPG的前提。多数波形检测算法一般会一次性完成PPG波形分析与检测,而对结果的校验需全部由人工完成\cite{Zhang2010,Chen2021,Allen2007,Feng2018,FengJiang2018}。
在处理受到干扰或存在畸变的PPG数据时,此类算法的检测性能通常也会下降。
为改进这些问题,在借鉴计算机科学领域的策略与机制分离思想的基础上\cite{Levin1975},本研究对PPG波形的检测过程进行了模式设计,
提出一种基于初筛—复核—决策(screening-checking-deciding,SCD)的新型算法,其处理流程图如\autoref{fig:detect}所示。

\begin{figure}[htbp]
    \centering
    \includegraphics[width=0.6\linewidth]{pulse_preprocess/scd}
    \caption{\label{fig:detect}SCD算法检测流程示意}
\end{figure}

机制与策略分离是计算机系统领域的一项重要设计原则,被广泛应用于一系列资源分配问题(如CPU调度、内存分配、服务质量)以及软件抽象的设计等问题中\cite{Wulf1974,Levin1975,Brinch2001}。
这里的机制可以理解成决定如何做(how),而策略则决定具体做什么(what)。
SCD算法以“宽进严出”为原则,通过初筛、复核及决策等机制检测有效PPG波形。
其中,初筛阶段可初步确定PPG波形;复核阶段利用多种形态学特征,将初筛结果定性描述为有效波形或异常干扰;决策阶段则是按照一定的策略从上述定性描述中最终确定有效波形。
SCD算法对各机制下使用的具体策略不做过多限制,支持直接更改初筛算法,增删复核标准,切换决策策略等设置,
保证了SCD算法的适用性与调整的灵活性,使高效准确地检测受到干扰或存在畸变的PPG数据成为可能。

另一方面,SCD算法可以视为一个广义上的监督学习模型的训练与应用过程\cite{Zhou2016}。其中,多种形态学特征的复核结果对应着模型的输入数据,决策阶段使用的策略对应着模型的训练算法,
确定有效波形的决策过程则对应着模型在新数据集上的应用过程。而复核与决策阶段涉及的多项具体数值可以视为该模型的超参数。本研究基于自主试验数据确定了这些超参数的具体数值(参见3.2小节)。

以下是对SCD算法的详细介绍。

一、初筛

波峰与波谷是脉搏波的最基本特征点,也是检测波形其他特征点的基础。在初筛阶段,SCD算法通过搜索窗定位PPG的波峰与波谷,再从原始数据中确定波形。

1、波峰定位

波峰是特定PPG波形内的最大值,在其左邻域内PPG幅值单调递增,右邻域内PPG幅值单调递减,则显然波峰必然是原始数据中的局部极大值。

本研究使用的波峰的定位步骤如下:
计算并遍历原始信号的一阶导数$X^{'}$,若出现$X_i^{'}\ge 0$且$X_{i+1}^{'}\le 0$,则说明出现了极大值。此时,定义时长一长一短的两个搜索窗,分别以当前数据点为窗中心,向前后双向检测并返回窗内最大值的位置。
若两个搜索窗的返回结果一致,则说明返回值就是一个波峰点。为防止同一波峰被多次检测,只有与上一成功检测的波峰位置点不同返回值才会被加入波峰缓存数组$Peaks$中。

在实际检测时,常会出现一阶导数在某邻域内多次出现过零点,导致多个局部极大值连续出现。为避免连续调用搜索窗、提高算法效率,本研究针对性进行了以下剪枝优化设计:

1)、采用中心对称搜索窗

窗口进行最大值搜索时是以当前搜索位置中心对称前后双向搜索,当第一个过零点出现时,这一邻域内的所有极值点均已被检索过,得到的局部最值已经是该邻域内最大值。

2)、合理设置搜索窗时长

结合正常波形的形态特点,本研究将搜索窗时长分别设置为0.1s与0.3s。前者可以直接剔除掉时间跨度过小的局部极值,后者是基于此前PPG波峰与其重博峰时间间隔的经验值,可以保证重博峰点不会被误检为整个波形的波峰。

3)、搜索窗步进值策略调整

若长窗在当前位置$C$以中心沿时间轴延伸方向找到了局部极大值的坐标$L$后,下一次的搜索窗位置可直接从该$L$处开始。当搜索完成后,若短窗返回结果$S=L$,则返回值必然是检测得到的波峰位置$P$,下次搜索可直接从波峰后开始;若$S\ne L$,
此时从$C$至$L$再进行搜索已经不可能再更新$L$坐标。而由于长窗搜索必然包含短窗,故下次搜索也可以同样调整检索位置从$L$处开始。

2. 波谷定位

在波峰的位置确定之后,波谷的定位相对简单,一种可行的处理思路是将两个连续波峰之间的最小值定义为PPG波形的波谷。该思路已经默认波形之间有连续性,即任意两个波形之间不存在其他干扰段,
且所有的波峰均已被正确无误地检测出来,因此对复杂信号的处理效果欠佳。

本研究改进上述思路,对波谷的定位按照先寻找定位再二次确认的方法进行。除满足最小值的条件下,波谷需要进一步满足其出现位置必须在这两个波峰之间的后半段。
此步得到的PPG波谷位置也被保存至缓存数组$Troughs$中,这一过程如\autoref{alg:troughs_detect}所示。
\begin{breakablealgorithm}
    \caption{PPG波形波谷定位检测}
    \label{alg:troughs_detect}
    \begin{algorithmic}[1] %每行显示行号
        \Require 待检原始数据数组$Points$,原始数据的一阶差分数组$D$,波峰数组$Peaks$
        \Ensure 正常形态下记录脉搏波波谷位置的数组
        \Function {DetectTroughs}{$Points, D, Peaks$}
            \State 初始化$Troughs$
            \For{$i\gets 0,Peaks.length()-1$}
                \State $leastX \gets (Peaks[i].x + Peaks[i+1].x )/2$
                    \For{$j \gets Peaks[i+1].x)-1, leastX$}
                        \If {$D[j-1]<0 \And D[j]\le 0$}
                            \State $lastT \gets Points[j]$
                            \State \textbf{break}
                        \EndIf
                    \EndFor
                \State $Troughs \gets lastT$
            \EndFor
            \State \Return{$Troughs$}
        \EndFunction
    \end{algorithmic}
\end{breakablealgorithm}

3. 完整波形确认

前两步已经分别得到了原始信号中所有波峰与波谷点,此时仅需按照(波谷,波峰,下一波谷)的规则组合成完整波形即可。
但对某些异常信号,\autoref{alg:troughs_detect}无法有效检出在两波峰的波谷。
因此,有必要组合后的波形再次进行检查校对。

参考以往研究对PPG波形对经验统计,本研究以一定的PPG波形的时间规则完成校对,以确保有效波形的必定在这些时间规则规定的间期内。
这些时间规则包括,波形起点与波形终点间期需要$P_{S2E}$需要大于0.4$s$,波形峰值点与波形终点间期$P_{P2E}$需要大于0.3$s$。

4. 其他特征点定位

初筛阶段不做除波峰波谷外的其他特征点检测。其他特征点的定位依赖于波形的正确检测,这些检测被推迟到SCD算法检测完成后才进行。

二、复核

如绪论中所述,PPG是一种平稳随机信号,短时间内采样得到的数据波形理应具有较高的相似度\cite{Qiu2012}。
而有效PPG波形与干扰、畸变信号在形态特征、统计特征上往往存在着一定差异,存在通过软件算法进行智能识别区分的理论可行性。
因此,SCD算法设计了PPG波形的多种形态学特征,并以这些特征为标准对初筛结果进行复核。

1、功率

功率标准描述了PPG波形的能量均值,也即波形内所有采样点的幅值平方的均值
\begin{equation}
    \label{equ:ppgp}
    P=\frac{\sum_{i=0}^{n-1}{x_i}^2}{n}
\end{equation}

其中,$P$为当前PPG波形功率值,$n$为该波形的采样点数,$x_i$为该波形每个采样点的采样幅值。

2、标准差

标准差标准描述了PPG波形中交流成分的平均幅值[1],是对上述功率标准的补充
\begin{equation}
    \label{equ:ppgstd}
    S=\sqrt{\frac{\sum_{i=0}^{n-1}{(x_i-\mathop{x} \limits^-})^2}{n}}
\end{equation}

其中,$S$为当前PPG波形功率值,$n$为该波形的采样点数,$x_i$为该波形每个采样点的幅值, $\mathop{x} \limits^-$为该波形内所有采样点的幅值均值。

3、波峰相对位置 

由于血管回流作用,PPG波形的下降支会较上升支持续时间更长,波峰必然出现在PPG波形的前半段。而波峰在PPG波形的相对位置可表示为
\begin{equation}
    \label{equ:rpeak}
    R = \frac{n}{n_p}
\end{equation}

其中,$R$为当前PPG波形的波峰相对位置,$n$为该波形的采样点数,$n_p$为该波形波峰所对应的采样点下标值。

4、基线漂移程度 

基线漂移程度衡量了PPG波形起点与终点幅值差异,可量化PPG波形受到干扰的程度,可用下列公式衡量
\begin{equation}
    \label{equ:b1}
    B_1 = |x_s-x_e|
\end{equation}
\begin{equation}
    \label{equ:b2}
    B_2 = \frac{|x_s-x_e|}{|x_p-x_l|}
\end{equation}
\begin{equation}
    \label{equ:b3}
    B_3 = \frac{x_p-x_s}{x_p-x_e}
\end{equation}

其中,$B_1$是对当前PPG波形的基线漂移程度绝对数值衡量,$B_2$与$B_3$是对当前PPG波形的基线漂移两种相对数值衡量。而$x_p$、$x_s$与$x_e$分别为PPG的波峰、起点与终点处幅值,
$x_l$为$x_s$与$x_e$中数值较小者。

5、其他标准

由于不同人群、不同采集设备、不同采集环境下采集得到的PPG信号可能形态上千差万别,与之对应的干扰信号也往往不尽相同。上述标准仍有不足以甄别区分有效信号与无效干扰的可能。
故SCD算法允许根据实际分析需要,自习定义新的复核标准或只使用以上部分复核标准。

另外,时间规则也是PPG波形检测阶段得到应用的另一项重要标准。
如波形检测阶段,搜索窗的窗长已经限定了重博波峰与主波峰间期$T_{P2R}$需要大于0.3$s$;
完整波形在进行确认时也限定了波形周期$T_{S2E}$需要大于0.4$s$、下降支时长$P_{P2E}$需要大于0.3$s$。
若参照心率定义,将一分钟中内有效脉搏波的数量规定为脉率(Pulse Rate,PR)。那么,原则上该算法可对$PR \le 150$的原始脉搏波数据进行波形检测。

6、有效波形判断

复核阶段进行有效波形判断时,使用的是上述PPG的形态学特征基于数值比较的逻辑值,而非其具体数值。
这是由于PPG波形个体差异性较为明显;同时,在上述特征上,同一个体的有效PPG波形与干扰、畸变信号也存在着较为显著的数值差异。
使用具体特征标准进行复核的步骤如下:

1)、计算当前个体的所有PPG波形对应的特征值。

2)、将步骤1中所有数值排序后,选取部分数值计算均值。

3)、依次将所有特征值与该均值进行比较,若特征值在均值的一定区间内,对应的波形才会被判断为有效波形(输出0);否则会被判为异常干扰(输出1)。

复核阶段的超参数数值如\autoref{tab:checkingp}所示。

\begin{center}
    \zihao{-5}
    \begin{longtable}{m{1.8cm}<{\centering}m{4cm}<{\centering}m{4cm}<{\centering}m{3cm}<{\centering}}
		\caption{复核阶段各标准的超参数数值}\\
		\label{tab:checkingp}\\
		\topline
        \colorhead\textbf{复核标准}      & \textbf{个体全部波形参与均值计算百分比区间} & \textbf{判为有效波形的均值相对区间} & \textbf{备注} \\
        \midline
        \endfirsthead
        \caption[]{(续)}\\
        \topline
        \colorhead\textbf{复核标准}      & \textbf{个体全部波形参与均值计算百分比区间} & \textbf{判为有效波形的均值相对区间} & \textbf{备注} \\
        \midline
        \endhead 
        \hline
        \endfoot
        \bottomline
        \endlastfoot
        \colorrowa $P$ & [0\%, 100\%] & [0.4, 1.8] & 无需排序 \\
        \colorrowc $S$ & [0\%, 100\%] & [0.4, 1.8] & 无需排序 \\
        \colorrowa $R$ & [20\%, 80\%] & [0.8, 1.2] & 需按升序排序 \\
        \colorrowc $B_1$ & [0\%, 80\%] & [0, 8] & 需按升序排序 \\
        \colorrowa $B_2$ & [0\%, 80\%] & [0, 10] & 需按升序排序 \\
        \colorrowc $B_3$ & / & [0.5, 2.0] & 无需计算均值 \\
    \end{longtable}
\end{center}
\vspace{-1cm}

三、决策

本研究在决策阶段使用了一定的处理策略,基于复核的逻辑输出,最终确定PPG波形是否为有效波形。由于复核与决策的输出均为逻辑值0或1,可用的处理策略包括基于数值组合直接判断、投票表决及训练监督学习模型等\cite{Zhou2016}。

本研究选择使用加权投票策略进行决策,具体过程如下:

1)、按复核流程计算当前波形在各标准上的逻辑输出。

2)将复核标准$P$与$S$的逻辑与运算结果计为$E$,将复核标准$B_1$、$B_2$与$B_3$的逻辑与计算结果计为$B$。

3)、按照0.3、0.5、0.2的权重对步骤2中得到的$E$、$B$及复核标准$R$进行加权计算,将结果计为$Y$。

4)若$Y$小于0.5,当前波形会被判断为有效波形;否则,会被判为异常干扰。

5)考虑到在进行与运算时,多个复核结果同时为1的特殊情况,若$P$与$S$的数值加法结果为2或$B_1$、$B_2$与$B_3$的数值加法结果超过2,当前波形也会被判断为异常干扰。

四、检测算法性能评估

本研究基于自主试验的PPG数据设计研发了SCD算法,同时利用公开数据集对算法的检测结果与性能进行了验证与评估。此外,本研究也选取了两种有代表性的PPG检测算法与SCD算法进行了横向对比验证。

1、自主数据试验数据

数据试验具体过程参见3.1节。经统计,本次临床数据采集共获取有效数据79条,共有有效PPG波形计7864个。

2、 公开数据集

重症监护医疗信息数据库(medical information mart for intensive care,MIMIC)是生物医学工程领域开源社区PhysioNet上最为知名的一个大型公开数据库\cite{mit2022,Goldberger2000,johnson2018mimic,mimic4}。该数据库记录了2001年
至2019年期间贝斯以色列女狄肯斯医疗中心重症监护病房患者的相关数据,拥有超过40000名患者的医疗健康数据记录\cite{johnson2018mimic}。
无袖带血压估计验证数据集(cuffless blood pressure estimation data set,CBPEDS)是Kachuee Mohamad等\cite{Kachuee2015,ucibp2022}于2015年在MIMIC的基础上,
进行了一定的预处理与数据清洗二次开发而来。CBPEDS目前在美国加州大学欧文分校机器学习数据库上公开,由于其出色的易用性而被广泛使用。
CBPEDS一共包含12000条数据记录,每条记录包含PPG、有创动脉血压与心电等三通道数据,采样率均为125 Hz,单条数据记录时长范围是1至10 min不等。
由于CBPEDS数据记录量较大,本研究仅以其中前50条数据记录为代表进行PPG检波算法的性能评估,这部分数据共有有效PPG波形计17562个。

3、其他PPG检波算法

差分法、阈值法、移动窗法、数字滤波法及由这些方法的组合构成的新方法常用于PPG波形检测过程之中\cite{Chen2019,cwl,Chen2021,ChenH2019,QYY2008,SJ2007,van2019,van20192}。
本研究选取了两种有代表性并提供完整代码的PPG检波算法与SCD算法进行对比分析,包括本研究团队的陈婉琳等\cite{Chen2019,cwl}此前提出的双移动时间窗算法(dual moving time-window, DMTW)
与Paul van Gent等\cite{van2019,van20192}提出的一种基于数字滤波器与离群点检测的PPG检波算法HeartPy。

4、对比与分析

SCD算法较DMTW算法与HeartPy算法的优势主要体现在对包含噪声干扰与畸变的PPG信号的准确识别上。使用这三种算法检测自主试验数据与CBPEDS数据的结果存在一定的差异,如\autoref{fig:scd_detect}所示。
由于HeartPy算法的本身输出限制,\autoref{fig:scd_detect}只标注了HeartPy算法检测出的PPG波形的波峰位置,有效波峰与异常波峰分别用黑色与红色进行了标注。
而SCD算法与DMTW算法检测出的有效PPG波形均在\autoref{fig:scd_detect}中用虚线进行了框选,其中黑色虚线表示两种算法对PPG波形的识别判断完全一致,红色虚线表示两种算法的检测结果出现了分歧。

\begin{figure}[htbp]
    \centering
    \includegraphics[width=\linewidth]{pulse_preprocess/scd3}
    \caption{\label{fig:scd_detect}三种PPG检波算法性能对比}
\end{figure}

从整体检测准确率来看,SCD算法较另外两种算法也具有一定的性能优势。统计三种算法在自主试验数据与CBPEDS数据上的错检波形数目,可得到对应算法的整体检测准确率,如\autoref{tab:scd}所示,其中需要注意的信息
已用粉红色底色、黑色字体加粗突出显示。从\autoref{tab:scd}可以发现,三种算法中,SCD算法的错检波形数目最少,识别的准确率最高。

\begin{center}
    \zihao{-5}
    \begin{longtable}{m{2cm}<{\centering}m{1.5cm}<{\centering}m{1.5cm}<{\centering}m{1cm}<{\centering}m{1.5cm}<{\centering}m{1cm}<{\centering}m{1.5cm}<{\centering}m{1cm}<{\centering}}
		\caption{三种PPG检波算法性能对比统计明细}\\
		\label{tab:scd}\\
		\topline
        \colorhead      &  & \multicolumn{2}{c}{\textbf{SCD算法}} & \multicolumn{2}{c}{\textbf{DMTW算法}} & \multicolumn{2}{c}{\textbf{HeartPy算法}}  \\
        \colorhead \multirow{-2}{*}{\textbf{数据源}}      & \multirow{-2}{*}{\textbf{波形总数/个}} & \textbf{错检数/个}      & \textbf{准确率} & \textbf{错检数/个}      & \textbf{准确率} & \textbf{错检数/个}      & \textbf{准确率} \\
        \endfirsthead
        \caption[]{(续)}\\
        \topline
        \colorhead\textbf{复核标准}      & \textbf{个体全部波形参与均值计算百分比区间} & \textbf{判为有效波形的均值相对区间} & \textbf{备注} \\
        \midline
        \endhead 
        \hline
        \endfoot
        \bottomline
        \endlastfoot
        \colorrowa 自主试验数据 & 7864 & \colorcellb \textbf{26} &  \colorcellb \textbf{99.6\%} & 112 & 98.6\% & 97 & 98.7\% \\
        \colorrowc CBPEDS & 17562 &  \colorcellb \textbf{50} &  \colorcellb \textbf{99.7\%} & 182 & 99.0\% & 168 & 99.0\% \\
    \end{longtable}
\end{center}
\vspace{-1cm}

此外,由于CBPEDS本身是重症监护患者的临床医学数据记录,部分患者的PPG数据波形畸变严重,显著异于常人,如\autoref{fig:ucibp_abnormal}所示。
SCD算法可以准确地识别这些异常波形,不进行波形标注。由于如\autoref{fig:ucibp_abnormal}所示的异常数据不在本研究选作算法性能评估的50条数据内,SCD算法的检测结果未能在\autoref{tab:scd}所示的统计值中得以体现。

\begin{figure}[h]
    \centering
    \subfigure[\label{fig:c_0179}编号0179原始数据波形(局部)]{
        \includegraphics[width=7.5cm]{pulse_preprocess/check/0179}
    }
    \quad
    \subfigure[\label{fig:c_0533}编号0533原始数据波形(局部)]{
        \includegraphics[width=7.5cm]{pulse_preprocess/check/0533}
    }
    \quad
    \caption{\label{fig:ucibp_abnormal}UCI-BP数据库中畸形信号示意}
\end{figure}

五、讨论

本研究对PPG波形的检测过程进行了模式设计,提出了一种新型SCD算法,算法由初筛、复核与决策等三部分组成。
在初筛时,通过搜索窗的方法确定波形;在复核时,通过功率、标准差、波峰相对位置与基线漂移程度等PPG形态学特征,区分有效波形与异常干扰
;在决策时,通过加权投票的策略确定PPG波形是否为有效波形。SCD算法可根据使用场景进行二次开发,支持更改初筛算法、增删复核标准与切换决策策略等设置。
经验证,该算法各模块设计合理,与其他算法对比结果表明,对PPG波形的识别准确率高、抗干扰能力强。SCD算法为多平台、多研究应用下的PPG分析检测的不同场景提供了一定的参考。

\subsection{重搏波与切迹检测}
重博波是PPG最具标志性的波形特征之一,其波峰与切迹的检测是一个完整的PPG检测算法必不可少的定位项之一\cite{Wang2012}。但在实际应用中,并不是每个人的PPG信号都有着明显的重搏波,特别是当被试出现外周阻力增加、血管壁弹性下降的情况后\cite{mmt}。
此时,随着PPG波形向外周传播,尖锐的重搏波切迹(incisura)常会变形甚至丢失,退变成一简单拐点,甚至重搏波波峰都会在下降支中不显著,导致最后无法从采集得到的信号中对两者直接进行精准定位,如\autoref{fig:incisura}所示。

\begin{figure}[htbp]
    \centering
    \subfigure[重博波明显的PPG波形]{
    \includegraphics[width=5.5cm]{pulse_preprocess/ri1}
    }
    \quad
    \subfigure[重博波不明显的PPG波形]{
    \includegraphics[width=5.5cm]{pulse_preprocess/ri2}
    }
    \caption[重搏波及切迹定位原理]{\label{fig:incisura}重搏波及切迹定位原理\cite{Wang2012,Su2014}}
\end{figure}

本研究借鉴了王选等\cite{Wang2012}于2012年提出的一种基于曲率$K$(curvature)的定位算法对切迹进行检测,该算法的核心思想是PPG信号在切迹退化成的变形点的邻域内具有最大的曲率值,如\autoref{alg:incisuras_detect}所示。
数学中,将一般曲线函数在点$(x,y)$处的曲率$K$定义为
\begin{equation}
    \label{equ:curvature}
    K=\frac{|y^{''}|}{{(1+{y^{'}}^2)}^{3/2}}
\end{equation}
其中,$y^{'}$与$y^{''}$分别是曲线函数在该点的一阶导数与二阶导数。

\begin{breakablealgorithm}
    \caption{PPG波形切迹定位检测算法}
    \label{alg:incisuras_detect}
    \begin{algorithmic}[1] %每行显示行号
        \Require 原始数据数组$Points$,检测完毕的波形$pulse$,原始数据的一阶差分数组$D$
        \Ensure 当前波形的切迹位置
        \Function {DetectIncisura}{$Points, pulse, D$}
            \State $s \gets pulse.peak.x + \frac{3}{10}(pulse.trough.x-pulse.peak.x)$
            \State $e \gets pulse.peak.x + \frac{9}{10}(pulse.trough.x-pulse.peak.x)$
            \State \Comment 切迹可能出现的位置为波形的下降支中后部分。
            \State $M  \gets \Call{Max}{$D[s],D[e]$}$
            \State \Comment 寻找到上述区间内一阶导数最大值点记为$M$。
            \If{$D[M] >0$}
                \State \Comment 下降支中存在明显的重搏波。
                \State $Incisura \gets \Call{Min}{Points[s],Points[M]}$
                \State \Comment 重搏波波峰即为点M至脉搏波终点之间的极大值点,切迹为脉搏波波峰至点M之间的最小值点。
            \Else
                \State \Comment 下降支一直单调下降,不存在明显的重搏波。
                \State $Incisura \gets \textproc{Min}({\Call{Curvature}{Points[M],Points[e]}})$
                \State \Comment 重搏波波峰定义为点M至脉搏波终点之间的曲率最大处,切迹定义为脉搏波波峰至点M之间的曲率最小处。
            \EndIf
            \State \Return{$Incisuras$}
        \EndFunction
    \end{algorithmic}
\end{breakablealgorithm}

\subsection{去除基线漂移}
由于呼吸干扰等原因,实际得到的PPG波形的起始点幅值(即连续两个波形的波谷幅值)很难保持一致。这种差异最终会导致PPG信号的基线出现波动漂移,如\autoref{fig:drift}所示。为满足后续特定PPG形态特征计算需求,
需要消除基线漂移的影响。
\begin{figure}[htbp]
    \centering
    \includegraphics[width=.6\linewidth]{pulse_preprocess/baselineadjust}
    \caption{\label{fig:drift}基线漂移去除原理示意及处理前后对比}
\end{figure}

本研究借助一种基于线性变换的思路完成了基线漂移处理。将一个包含$n$个采样点的PPG波形的幅值表示为序列$x_i$,其中$0 \le i \le n-1$。
由于始末位置脉搏波幅值不等,则明显两者之间存在一条斜率$k \ne 0$的直线,其中
\begin{equation}
    \label{equ:linek}
    k=\frac{x_0-x_{n-1}}{n-1}
\end{equation}
则该直线上任意点即代表了在该时刻PPG波形与水平基线的偏移量,即
\begin{equation}
    \label{equ:liney}
    \Delta_i=ki+x_0
\end{equation}
此时,去除基线漂移后的PPG信号可标示为
\begin{equation}
    \label{equ:adjusta}
    y_i=x_i-\Delta_i
\end{equation}

\subsection{信号的重采样}
通常而言,在一项具体的应用研究中,PPG数据的采样率维持不变的。但在某些情况下,原始数据采样率不能满足实际需求,需要进行采样率调整才能满足特定的数值计算需求。
其中,减少抽样率的过程称为信号的抽取,也称抽样率压缩;增加抽样率的过程则称之为信号的插值,也即抽样率扩张\cite{Cheng2008}。信号的抽取与插值都会导致原始信号出现频谱迁移,即改变信号的频率成份\cite{Cheng2008}。

相对而言,信号的抽取过程更容易理解。若需要用整数$D$对$x(n)$进行抽取,以使抽样率降低到原始值的$1/D$,可按照每连贯的$D$个抽样中取出一个信号值。这样的处理称为整数$D$抽取\cite{Cheng2008}。
插值是抽取的逆过程,通过某些已知的数据点去推断一个(系列)特定的函数,使得所有已知数据点均在该函数图像上,从而去推断更多未知数据点,这一过程如\autoref{fig:spline}所示。若不考虑可能出现的信号失真,理论上
可以通过调整抽取与插值的数值对原始信号的采样率进行任意的调整。
\begin{figure}[htbp]
    \centering
    \subfigure[经过点(1,2),(2,1),(4,4)和(5,3)的线性样条]{
    \includegraphics[width=5.5cm]{pulse_preprocess/spline1}
    }
    \quad
    \subfigure[经过相同点的一种可能的三次样条插值]{
    \includegraphics[width=5.5cm]{pulse_preprocess/spline2}
    }
    \caption{\label{fig:spline}经过4点的线性样条插值与三次样条插值对比}
\end{figure}

多项值插值与样条插值是插值最常用的两种算法\cite{Timothy2018,Carl2008}。多项式插值给出满足所有原始数据点的单一公式,由于此过程中仅涉及浮点数加法与乘法,可以很方便的在PC及嵌入式设备上实现,因而得到广泛应用。
而样条插值则使用多个公式来通过所有数据点,其中每个公式均为低阶多项式。三次样条插值是样条插值在工业领域应用最广泛的算法之一,可以得到光滑的拟合插值曲线。
对原始数据点$(x_1,y_1),(x_1,y_1),\cdots,(x_n,y_n)$,三次样条的插值曲线$S(x)$在每两个数据分段区间$[x_i,x_{i+1}]$内均使用三阶多项式
\begin{equation}
    \label{equ:spline}
    S_{i}=y_{i}+b_{i}(x-x_{i})+c_{i}{(x-x_{i})}^2+d_{i}{(x-x_{i})}^3
\end{equation}
并保证每两个多项式在端点(即原始数据点)处不仅数值相等,相应的斜率与曲率均相等,即
\begin{equation}
    \label{equ:cubiccha}
    \left \{
    \begin{aligned}
        S_{i}(x_{i})&=y_i,&\text{i=1,$\cdots$,n-1}\\
        S_{i}(x_{i+1})&=y_{i+1},&\text{i=1,$\cdots$,n-1}\\
        S_{i}^{'}(x_{i-1})&=S_{i}^{'}(x_{i}),&\text{i=2,$\cdots$,n-1} \\
        S_{i}^{''}(x_{i-1})&=S_{i}^{''}(x_{i}),&\text{i=2,$\cdots$,n-1}
    \end{aligned}
    \right.
\end{equation}
将\autoref{equ:spline}代入\autoref{equ:cubiccha}化简整理后,可以得到包含$3n-5$个独立方程的方程组。由于每个局部$S_i$中有三个未知参数,一共有$3n-3$个待解参数。
由线性代数相关知识可知,该方程组有无穷多组解,此时可以构造处无穷多条通过所有数据点$(x_i,y_i)$的样条曲线。
因此,需要添加额外的方程对样条曲线进行约束,一般的约束条件都是对样条左右端点处进行限定,常见的附加边界条件如\autoref{tab:splinekind}所示\cite{Timothy2018}。
在补充\autoref{tab:splinekind}中的两个端点条件公式后,即可完成上述方程组求解,从而确定唯一的三次样条插值曲线。

\begin{center}
    \zihao{-5}
    \begin{longtable}{m{5cm}<{\centering}m{8cm}<{\centering}}
		\caption{几种常见的三次样条端点条件}\\
		\label{tab:splinekind}\\
		\topline
        \colorhead \textbf{样条种类}&\textbf{端点条件}\\
        \midline
        \endfirsthead
        \caption[]{(续)}\\
        \topline
        \colorhead \textbf{样条种类}&\textbf{端点条件}\\
        \midline
        \endhead 
        \hline
        \endfoot
        \bottomline
        \endlastfoot
        \colorrowa 自然三次样条&
        $
            S_{1}^{''}(x_{1})=0,
            S_{n-1}^{''}(x_{n})=0
        $
        \\
        \colorrowc 曲率调整三次样条&
        $\left \{
        \begin{aligned}
            &S_{1}^{''}(x_{1})=v_1,&v_{1}\neq0\\
            &S_{n-1}^{''}(x_{n})=v_n,&v_{n}\neq0
        \end{aligned}
        \right.
        $
        \\
        \colorrowa 钳制三次样条&
        $\left \{
        \begin{aligned}
            &S_{1}^{'}(x_{1})=v_1,&v_{1}\neq0\\
            &S_{n-1}^{'}(x_{n})=v_n,&v_{n}\neq0
        \end{aligned}
        \right.
        $
        \\
        \colorrowc 抛物线端点三次样条&
        $
            d_1=0,d_{n-1}=0
        $
        \\
        \colorrowa 非纽结三次样条&
        $
            d_1=d_2, d_{n-2}=d_{n-1}
        $
        \\
    \end{longtable}
\end{center}
\vspace{-0.8cm}

由于3.1节中数据试验所采用的设备GE B450设备的采样率仅为100$Hz$,为计算本研究后续章节提出的多种PPG时域特征,本研究使用了基于自然边界条件的三次样条算法对PPG信号进行均匀插值\cite{ttk2021},
重采样处理后的PPG数据采样率被提高至2000$Hz$。
\subsection{数据标准化}
在上述各项预处理过程完成后,PPG波形在幅值上仍然有很大的个体差异,即使对同一被试对象,其波形在幅值上也会有一定的波动。为消除个体差异对特定波形特征计算的影响,需要对PPG波形进行归一化处理。

时间标准化与幅值标准化是PPG标准化两类常见的方式\cite{mmt}。时间标准化的基本原理是将PPG信号进行分组并计算每组信号的平均心动周期,随后对每组内的PPG信号进行时间尺度上的缩放,使最终的总平均心动周期保持一致。
由于不同个体的时间尺度上的缩放比例不一致,很容易导致PPG信号波形发生畸变失真,严重影响后续特征计算。
第二章已经介绍过,PPG信号的波形幅值对不同个体而言并无实际的生理意义,故本研究采取了幅值标准化的处理方式对PPG信号进行处理。

与去除基线漂移过程类似,为使幅值标准化,只需对PPG波形内所有采样点进行一次线性变换。
为避免处理大量的0至1之间的小数,本研究没有采用最常见的归一化处理,而是将单个波形内所有数据点按同一尺度缩放映射到[0,1000]区间内。
所有波形的标准化系数即缩放比例也同时被保存记录下来。

\section{脉搏波的时域特征}
绪论中已经介绍过,PPG信号本身蕴含着丰富的血液动力学信息,其形态、强度、速率、节律等特征可以反映心脏的功能与状态,也可以反映出各级动脉及分支中血管壁弹性、血管阻力、血液黏度等信息,是评价人体心血管系统生理病理状态的重要依据\cite{PPGYY}。
这也就是说,PPG信号携带了能够表征人体心血管系统生理病理状态的全部信息,这些信息蕴藏在了PPG波形的变化之中。因此,用何种方式描述、获取、挖掘这些信息,是基于脉搏波的分析工作能够开展的前置条件。

通常而言,对PPG信号进行分析可以采取时域频域联合分析方法,分别从时域及频域提取相关特征进行参数描述及后续分析工作。
但本研究的PPG数据信号已经在硬件采集设备上进行过频段选择,其频域带宽窄,频率成分构成相对简单,故不适用频域特征分析方法。
另外,本研究的PPG数据信号时长有限,单条数据长度约1至3分钟,因此,另一类常见的非线性参数特征参数的分析也不适用。
故本研究仅从\textbf{时域角度}对PPG进行特征提取与描述工作。

在PPG波形的诸多时域特征参数中,时间类参数、幅度类参数、面积类参数与斜率类参数往往定义清晰明了、意义明确易于理解,易与脉搏波产生原理过程联系起来,可解释性强。
因此,这些参数在基于PPG的各类研究中得到了最广泛的应用\cite{cwl,mmt}。

本节对此类时域特征进行汇总整理,特别地,本节也对已在PE研究中得到应用的PPG进行说明。

\subsection{时间类特征参数}
时间类特征参数描述了PPG波形中某些特殊点之间的时长值或多个这样时长之间的比例关系。脉搏传导时间(Pulse Transit Time,PTT)则是其中最有代表性的时间类参数之一。
如\autoref{fig:ri}所示,PTT是对脉搏波主波峰与重博波波峰或反射波导致的拐点之间的时间间隔的量化描述\cite{Brumfield2005,Su2014}。PTT在基于PPG的各类研究中得到了广泛应用。
\begin{figure}[htbp]
    \centering
    \subfigure[\label{fig:ri1}重博波明显的PPG波形]{
    \includegraphics[width=5.5cm]{pulse_preprocess/ri1}
    }
    \quad
    \subfigure[\label{fig:ri2}重博波不明显的PPG波形]{
    \includegraphics[width=5.5cm]{pulse_preprocess/ri2}
    }
    \caption[PTT及RI计算原理示意]{\label{fig:ri}PTT及RI计算原理示意\cite{Su2014}}
\end{figure}

除PTT外,时间类特征参数还包含脉搏波脉动周期、上升支间期、下降支间期等。2016年,王梦婷提出了一种描述动脉内高压力水平维持的时间参数,并在此基础上归一化处理后得到了多项衍生参数,
包含动脉高压力持续时间、血管硬度指数、心肌收缩系数及心博出系数等\cite{mmt}。
2018年,陈婉琳等人基于斜率极值提出了上升支最大斜率间期、下降支最小斜率间期等参数\cite{cwl}。以上参数的定义参见\autoref{fig:timefeature}及\autoref{tab:timefeature}所示。
\begin{figure}[htbp]
    \centering
    \includegraphics[width=.6\linewidth]{pulse_preprocess/timefeature}
    \caption[PPG时间及幅值类参数示意]{\label{fig:timefeature}PPG时间及幅值类参数示意。其中,AA'的幅值为主波峰值的2/3,D点为降中峡。}
\end{figure}

\begin{center}
    \zihao{5}
    \begin{longtable}{m{4cm}<{\centering}m{9cm}<{\centering}m{2cm}<{\centering}}
		\caption{常见PPG时间类参数定义}\\
		\label{tab:timefeature}\\
		\topline
        \colorhead \textbf{参数} & \textbf{物理意义} & \textbf{表达式} \\
        \midline
        \endfirsthead
        \caption[]{(续)}\\
        \topline
        \colorhead \textbf{参数} & \textbf{物理意义} & \textbf{表达式} \\
        \midline
        \endhead 
        \hline
        \endfoot
        \bottomline
        \endlastfoot
        \colorrowa 脉动周期      &  相邻两个波谷之间的时间间隔         &  $SE$\\
        \colorrowc 上升支间期      &  从波形起点(波谷)至主波峰之间的时间间隔         &  $SO$\\
        \colorrowa 下降支间期      &  从主波峰至波形终点(波谷)之间的时间间隔        &  $OE$\\
        \colorrowc 动脉高压力持续时间    &  主波维持在峰值幅值2/3高度的时间间隔         &    $A'B'$   \\
        \colorrowa 血管硬度指数    &  动脉高压力持续时间与脉动周期的比值         &   $\frac{A'B'}{SE}$    \\
        \colorrowc 心肌收缩系数    &  上升支间期与脉动周期的比值         &  $\frac{SO}{SE}$    \\
        \colorrowa 心搏出系数      &   从波峰至降中峡的时间间隔与脉动周期的比值       &   $\frac{OB'}{SE}$\\
        \colorrowc 上升支最大斜率间期      &   上升支波谷与上升支斜率最大点之间的时间间隔      &   /    \\
        \colorrowa 下降支最大斜率间期      &   下降支斜率最小点与下降支波谷之间的时间间隔      &    /  \\
    \end{longtable}
\end{center}

\subsection{幅值类特征参数}

幅值类特征参数描述了PPG波形中某个时刻的幅值高度或多个时刻幅值高度之间的比例关系。在各项研究中得到普遍应用的脉搏波增强指数(Augmentation Index,AIX)、反射指数(Reflection Index,RI)及K值等
均属于幅值类参数。

AIX指心脏收缩期早期与晚期拐点之间的差值与峰值之比,计算的关键需要先找到表征脉搏波反射波上升冲程(up-stroke)的拐点$P1$,如\autoref{fig:aix}所示\cite{Su2014}。
\begin{equation}
    \label{equ:aix}
    AIX = \pm \frac{\Delta P}{PP}
\end{equation}
特别地,\autoref{equ:character}中$\pm$的取值视拐点$P1$与峰值点的相对位置而定。当拐点在脉搏波波峰点左侧时,AIX符号取正,表示主波峰是经过叠加得到的,如图\autoref{fig:aix1}所示;反之,
当拐点在脉搏波波峰点右侧时,AIX符号取负,表明主波峰未得到“增强”,如图\autoref{fig:aix2}所示。
\begin{figure}[htbp]
    \centering
    \subfigure[\label{fig:aix1}拐点在波峰左侧]{
    \includegraphics[width=5.5cm]{pulse_preprocess/aix1}
    }
    \quad
    \subfigure[\label{fig:aix2}拐点在波峰右侧]{
    \includegraphics[width=5.5cm]{pulse_preprocess/aix2}
    }
    \caption[AIX计算原理示意]{\label{fig:aix}AIX计算原理示意\cite{Su2014}}
\end{figure}

RI是由脉搏波舒张峰值或拐点处幅值与收缩谷值之比计算得到的,如\autoref{fig:ri}所示\cite{Su2014,Elgendi2012}
\begin{equation}
    \label{equ:ri}
    RI = \frac{b}{a} \cdot 100\%
\end{equation}

K值是另一种在临床得到广泛应用的基于脉搏波面积变化的形态学特征无量纲参数,最早由罗志昌等人提出\cite{Luo1988,PPGYY}
\begin{equation}
    \label{equ:ppgk}
    K=\frac{P_m-P_d}{P_s-P_d}
\end{equation}
其中,$P_m=\frac{1}{T}\int_{0}^{T}P(t)dt$为一个心动周期内脉搏压力$P(t)$的平均值,$P_s$与$P_d$分别为收缩压与舒张压。罗志昌等人的研究表明,K值依赖于脉搏波的波形动态变化,
同时能够在很大程度上反应血管外周阻力与血管壁硬化程度等生理因素的变化。
\begin{figure}[htbp]
    \centering
    \includegraphics[width=.55\linewidth]{pulse_preprocess/k}
    \caption{\label{fig:k}K值的计算原理示意}
\end{figure}

除上述参数外,幅值类特征参数还包括脉搏波波峰幅值、波谷幅值差值、降中峡幅值及外周阻力系数等\cite{cwl,mmt}。这些参数的具体定义参见\autoref{fig:timefeature}及\autoref{tab:heightfeature}所示。
\begin{center}
    \zihao{-5}
    \begin{longtable}{m{4cm}<{\centering}m{9cm}<{\centering}m{2cm}<{\centering}}
		\caption{常见PPG幅值类参数定义}\\
		\label{tab:heightfeature}\\
		\topline
        \colorhead \textbf{参数} & \textbf{物理意义} & \textbf{表达式} \\
        \midline
        \endfirsthead
        \caption[]{(续)}\\
        \topline
        \colorhead \textbf{参数} & \textbf{物理意义} & \textbf{表达式} \\
        \midline
        \endhead 
        \hline
        \endfoot
        \bottomline
        \endlastfoot
        \colorrowa 波峰幅值      &  主波峰幅值         &  $OP$\\
        \colorrowc 波谷幅值差值      &  相邻波谷之间的幅值之差         &  /\\
        \colorrowa 降中峡幅值      &  降中峡至基线幅值         &  $DD'$\\
        \colorrowc 外周阻力系数      &  降中峡幅值与主波峰幅值之比         &  $\frac{DD'}{OP}$\\
    \end{longtable}
\end{center}

\subsection{面积类特征参数}

面积类参数计算依赖与脉搏波波形与基线所形成的面积,主要通过按时间方向对PPG波形曲线积分完成计算过程。
面积类参数可以看成是血管在一段时间内的血液容积的映射。
王梦婷在其研究中指出\cite{mmt},PPG上升支主要受心脏射血量与血压脉动影响,下降支主要受外周阻力影响。其中,上升支随心脏射血量增加愈益陡峭,射血时间越短,
其形态图像越向左上方凸出;而当外周阻力增加时,会导致动脉血管壁弹性下降,下降支波形向上方凸出。

常见的面积类特征参数包括上升支面积、下降支面积、
全周期面积、上升支面积比、下降支面积比、全周期面积比及(下降支)面积差值比(Area Difference Ratio,ADR)\cite{Feng2018}等。
以上参数的具体定义参见\autoref{tab:areafeature}及\autoref{fig:areafeature}。
\begin{center}
    \zihao{-5}
	\begin{longtable}{m{3.5cm}<{\centering}m{6.5cm}<{\centering}m{4.5cm}<{\centering}}
		\caption{PPG面积类参数示意}\\
		\label{tab:areafeature}\\
        \topline
        \colorhead \textbf{参数} & \textbf{物理意义} & \textbf{表达式} \\
        \midline
        \endfirsthead
        \caption[]{(续)}\\
        \topline
        \colorhead \textbf{参数} & \textbf{物理意义} & \textbf{表达式} \\
        \midline
        \endhead 
        \midline
        \endfoot
        \bottomline
        \endlastfoot
        \colorrowa 上升支面积      &  从波形起点至峰值点对波形曲线积分所得面积值         &  $S_r=\int_{T_S}^{T_P}P(t)dt$\\
        \colorrowc 下降支面积      &  从波形峰值点至终点对波形曲线积分所得面积值         &  $S_f=\int_{T_P}^{T_E}P(t)dt$\\
        \colorrowa 全周期面积      &  脉搏波波形在一个完整周期内的积分所得面积值         &  $S_t=\int_{T_S}^{T_E}P(t)dt$\\
        \colorrowc 上升支面积比    &  上升支面积与$\triangle OPS$面积之比         &   $R_r=\frac{S_r}{S_{\triangle OPS}}$    \\
        \colorrowa 下降支面积比    &  下降支面积与$\triangle OPE$面积之比        &   $R_f=\frac{S_f}{S_{\triangle OPE}}$    \\
        \colorrowc 全周期面积比    &  全周期面积与$\triangle SPE$面积之比         &   $R_t=\frac{S_t}{S_{\triangle SPE}}$    \\
        \colorrowa 下降支面积差值比&  下降支面积和$\triangle OPE$面积之差与$\triangle OPE$面积之比        &    $\Delta R_f=\frac{S_{\triangle OPE}-S_f}{S_{\triangle OPE}}=1-R_f$\\
	\end{longtable}
\end{center}

% \clearpage
\begin{figure}[htbp]
    \centering
    \includegraphics[width=.6\linewidth]{pulse_preprocess/areafeature}
    \caption{\label{fig:areafeature}PPG面积及斜率类参数示意}
\end{figure}

值得一提的是,Feng等人已经将ADR应用于子痫前期的研究中,其研究结果显示患有PE的孕妇与正常孕妇在PPG信号的ADR参数上具有统计意义上的显著差异(0.752 VS 0.723,P<0.01)\cite{Feng2018}。

另外需要注意的是,从\autoref{tab:areafeature}中的参数定义不难发现,ADR与下降支面积比完全线形相关,两者之间满足
\begin{equation}
    \label{equ:adr}
    ADR = 1-R_f
\end{equation}
此外,王选在其研究中也证明了全周期面积比与幅值类参数K值完全线性相关\cite{Wang2012}
\begin{equation}
    \label{equ:kandart}
    R_t=\frac{S_t}{S_{\triangle SPE}}=\frac{\int_{0}^{T}P(t)dt}{\frac{1}{2}P_sT}=\frac{2\cdot P_mT}{P_sT}=2\cdot K
\end{equation}
因此,若需要使用这两组参数对PPG进行描述时,只需视情况分别从每组中选取一个即可。

\subsection{斜率类特征参数}

斜率类参数是对PPG波形在一段时间内幅值的上升/下降速率的快慢的量化描述,反应了血液容积在这段时间内的平均流速。

光电容积斜率指数(photoplethysmography slope index,PSI)是斜率类参数中比较有代表性的一种\cite{Chen2019}。其计算的基本思想是将PPG波形分段后再描述各段之间的幅值变化,
如\autoref{fig:psi}所示。值得一提的是,陈婉琳等人的研究发现PSI可以在一定程度上对PE患者进行识别\cite{Chen2019}。
\begin{figure}[htbp]
    \centering
    \includegraphics[width=.6\linewidth]{pulse_preprocess/psi}
    \caption[PSI计算原理示意]{\label{fig:psi}PSI计算原理示意\cite{Chen2019}}
\end{figure}

此外,常见的斜率参数还包括上升支平均斜率、上升支最大斜率、下降支平均斜率、下降支最大斜率等。这些参数的具体定义参见\autoref{fig:areafeature}及\autoref{tab:slopefeature}所示。
\begin{center}
    \zihao{-5}
    \begin{longtable}{m{4cm}<{\centering}m{9cm}<{\centering}m{2cm}<{\centering}}
		\caption{常见PPG斜率类参数定义}\\
		\label{tab:slopefeature}\\
		\topline
        \colorhead \textbf{参数} & \textbf{物理意义} & \textbf{表达式} \\
        \midline
        \endfirsthead
        \caption[]{(续)}\\
        \topline
        \colorhead \textbf{参数} & \textbf{物理意义} & \textbf{表达式} \\
        \midline
        \endhead 
        \hline
        \endfoot
        \bottomline
        \endlastfoot
        \colorrowa 上升支平均斜率      &  主波峰值与上升支时间之比         &  $\frac{OP}{SO}$\\
        \colorrowc 下降支平均斜率      &  主波峰值与下降支时间之比         &  $-\frac{OP}{OE}$\\
        \colorrowa 上升支最大斜率      &           &  /\\
        \colorrowc 下降支最小斜率      &           &   /    \\
    \end{longtable}
\end{center}

\subsection{其他特征}

除上述特征参数外,在PPG的其他研究领域还有一些较为知名的特征参数,大动脉僵硬指数(stiffness index,SI)与脉搏波波形速度(Pulse Wave Velocity,PWV)就是其中最为典型的代表。

SI被定义为是被试身高$h$与脉搏波的主波峰值降中峡的时间间隔$\Delta T$的比值\cite{Elgendi2012,Millasseau2002,Brumfield2005},即
\begin{equation}
    \label{equ:si}
    SI = \frac{h}{\Delta T}
\end{equation}
Millasseau等人的研究表明随着大动脉硬度增加、主动脉和大动脉中压力波的脉搏波速度增加,
收缩和舒张峰值之间的时间延迟也会随着年龄的增长而减少,导致SI也会随着年龄增长而增长,如\autoref{fig:si}所示\cite{Elgendi2012,Millasseau2002,Brumfield2005}。
\begin{figure}[htbp]
    \centering
    \includegraphics[width=.4\linewidth]{pulse_preprocess/si}
    \caption[SI计算原理示意]{\label{fig:si}SI计算原理示意\cite{Elgendi2012,Millasseau2002,Brumfield2005}}
\end{figure}

脉搏波波形速度(Pulse Wave Velocity,PWV)是指心脏每次搏动射血产生的沿大动脉壁传播的压力波传导速度\cite{Van2012}。在计算PWV时,需要在人体两部位分别测量脉搏波信号。
若记两测量部位的直线距离的80\%为$d$,两部位之间的脉搏波传导时间差为$t$,则有
\begin{equation}
    \label{equ:pwv}
    PWV = \frac{d}{t}
\end{equation}
PWV已被证实与动脉扩张性、僵硬度、管壁厚度和血液黏稠度密切相关。
特别地,PWV已经在子痫前期的诸多研究中得到应用\cite{Tomsin2012,Katsipi2014,VivianaIvan2018,Ira2014}。
需要注意的是,由于PWV的测量依赖于在人体不同部位测量脉搏波(且测量多利用压力传感器完成),故\textbf{不适用}单点测量的光电容积脉搏波研究。

\section{小结}
本章主要对脉搏波信号的预处理过程与参数描述进行了详细的说明。针对脉搏波预处理,本章对信号滤波、波型检测、重博波检测、基线漂移去除、插值及标准化过程进行了介绍,并重点解释
了本研究提出的一种新型模块化脉搏波波形检测算法,即初筛-复核-投票算法。该算法具有检测准确度高、抗干扰能力强、可拓展性强等优点。
针对脉搏波的参数描述,本章首先从方法论的角度概括了参数描述的意义与本质,随后从常见的脉搏波时域描述参数及本研究原创提出的描述参数对研究内容进行了介绍。
其中,原创提出的脉搏波描述参数包含描述脉搏波波形及波形之间差异性等两大类。
本章的相关研究内容为后续章节所进行数据分析工作打下了基础。