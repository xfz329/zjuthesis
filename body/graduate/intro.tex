\chapter{绪论}

\section{引言}
子痫前期(preeclampsia, PE)又作先兆子痫,是孕妇妊娠期特有的一种多系统进展性疾病, 与妊娠期高血压(gestational hypertension)、子痫(eclampsia)、
慢性高血压并发子痫前期(chronic hypertension with superimposed preeclampsia)以及妊娠合并慢性高血压(chronic hypertension)统称妊娠期
高血压疾病(hypertension disorders of pregnancy, HDP)\cite{OAG9,HDASOM,2000s1}。
原发性高血压与蛋白尿是临床PE患者最有代表性的并发症之一。
随着近年对PE相关研究的深入,世界卫生组织对PE的涵盖范围进行了进一步的拓展,将其定义为:在妊娠20周后出现新发(原发)高血压,在两次间隔$4h$或$4h$以上的血压(blood pressure, BP)测定中,收缩压(systolic blood pressure, SBP) ≥ $140mmHg$和(或)
舒张压(diastolic blood pressure, DBP) ≥ $90mmHg$,且伴有下列任一项或多项\cite{OAG9,FIGO}:

1.孕妇出现蛋白尿症状,其24小时尿蛋白总量 ≥ $300mg$,或尿蛋白/肌酸酐比值 ≥ $30mg/mol$,或随机尿蛋白 ≥ (+);

2.孕妇无尿蛋白相关症状但伴有以下任一器官或系统功能紊乱、受累受损:心、肺(肺水肿)、肝(血清转氨酶水平为正常值2倍以上)、肾(血肌酐 ≥ $1.1mg/dl$
或为正常值2倍以上)等重要器官功能紊乱,或血液系统(血小板 < $100 \times 10^{9}/L$等)、消化系统、神经系统的异常改变等;

3.胎盘-胎儿受到累及,出现生长受限、脐动脉多普勒分析检测异常、甚至死胎等症状。

妊娠期高血压疾病可引起严重的母胎并发症,是孕产妇和围产儿病死率升高的主要原因\cite{OAG9}。
据世界卫生组织统计,PE在孕妇中发病率高达5\%-10\%,是体内大出血以外致使孕妇死亡的第二大危险因素,每年可导致全球范围内约76 000名孕妇死亡,并进一步导致约500 000
名胎儿/婴儿的死亡\cite{DAM2015,LCT2006}。为推广普及人们对危及母婴生命安全的PE的认知,同时教育女性了解她们当前及长期的健康风险,
全球孕妇保健组织自2017年起将每年的5月22日确定为世界子痫日(world preeclampsia day)。

就现阶段我国国情而言,排名世界第一的人口基数导致了我国庞大的妊娠孕妇数及新生儿数\cite{nbs2022}。
最新地区人口普查信息显示,妇女峰值生育率微幅下降、高峰年龄区间后移并缩短;妇女晚育现象比较普遍;自全面开放二孩政策后,二孩、多孩生育率持续回升\cite{zjtjj2022}。而临床研究已经证实,
高龄与多次妊娠均属于可能导致PE的风险因子(risk factors),会增加孕妇的PE患病可能\cite{Duckitt2005,FIGO,Yogev2010,Poon2010,Lee2000,Coonrod1995,Robillard1993}。
因此,实现PE的预测与快速有效的医学诊断乃至精确治疗与干预,可以有效保障孕妇及围产儿的生命健康安全,具有重大的临床应用价值。
\section{子痫前期概述}
截止目前,医学界对PE的病因与发病机制尚未明确,相关研究还在继续进行之中。但得到公认的一点是,PE病发具有异质性,多种因素、机制与通路均对PE的病发有所影响,不能仅以“一元论”的观点对待。
目前临床对PE的病因和发病机制的研究可以概括如下:

一、主流学说

目前临床最为普遍接受的主流学说认为,PE是由子宫螺旋小动脉重铸不足导致的,PE的发病与妊娠早期胎盘功能紊乱密切相关\cite{OAG9,Duvekot2010,2009ix}。其具体作用机制可以概括为两个阶段,如\autoref{fig:ppp}所示。
在第一阶段,孕妇子宫螺旋动脉重构受损、出现重铸障碍,绒毛外滋养细胞浸润能力受损,导致胎盘缺血、缺氧,释放多种胎盘因子,该阶段无明显临床现象。在第二阶段,各种胎盘因子进入母体血液循环,血管阻力增大,胎盘灌注减少,
促进系统性炎症反应的激活及血管内皮损伤引起PE多样化的临床表现。
\begin{figure}[htbp]
    \centering
    \includegraphics[width=\linewidth]{intro/ppp}
    \caption[PE发病机制的子宫螺旋小动脉重铸不足学说示意]{\label{fig:ppp}PE发病机制的子宫螺旋小动脉重铸不足学说示意\cite{Duvekot2010,2009ix}。}
\end{figure}

二、其他学说

临床的专家学者们还提出了多种可能的PE病发机制,包括炎症免疫过度激活学说、血管内皮细胞损伤及前列腺素合成失调学说、遗传学说、胎盘因子学说及营养不足学说等。
炎症免疫过度激活学说认为PE是母系-父系免疫适应不良导致的,即胎儿胎盘具有的半抗原性移植体特性导致同种异体移植排斥,最终引起的母系同种免疫反应从而诱发PE\cite{Sibai2005,OAG9,Shi2006,Moffett2002}。
血管内皮细胞损伤及前列腺素合成失调观点认为,血管内皮细胞出现损伤后导致前列腺素合成失调,并进一步使扩张血管的物质如一氧化氮、前列环素等合成减少,同时使收缩血管的物质如内皮素、
血栓素等合成增加,从而促发血管痉挛导致PE病发\cite{OAG9,Sibai2005}。遗传学说则是以PE具有家族倾向性的临床现象为基础\cite{OAG9,Sibai2005,Ge2013}。
而近年来,学者们相继提出的胎盘因子学说\cite{Shi2006}与营养不足学说\cite{OAG9}等假设仍需进一步临床研究与理论研究来证实。
\raggedbottom

\section{子痫前期识别诊断的研究现状}
现代医学对PE的认知经历了漫长的探索\cite{BJOG2016}。早在古希腊Hippocrates时代(约公元前460-370年),医学界就对子痫的症状有一定的认识。但直至18世纪,医学界才将子痫与癫痫加以区分对待。
19世纪中叶,医学界开始认识到子痫之前存在一个前驱状态。法国医生Pierre Rayer首次描述了子痫孕妇的蛋白尿症状,英国医生John Lever更进一步证实了蛋白尿与子痫发病的特异性关系。
20世纪初,尿液分析与血压测量开始用于子痫的诊断,而进一步细化的PE的相关概念开始出现。20世纪中后叶,由于医学诊断技术的发展,多种新的检测指标与技术不断被引入子痫及PE的识别诊断中。
进入21世纪后,由于智能化技术的发展与电子设备的普及,基于医学大数据的人工智能相关研究又成为了新的热点。

本小节将从现阶段临床对PE病发评估的常用检测参数指标、医用检测设备及新兴人工智能检测技术等三方面对PE的识别诊断现状进行介绍。

\subsection{检测参数}
现阶段临床使用的PE的检测参数指标可概括为风险因子与生物标志物(biomarkers)等两大类。

一、风险因子筛查

已有研究表明,孕妇自身的一些基础信息与与PE的病发密切相关,这些信息被统称为PE风险因子\cite{Magee2008,FIGO,Lowe2015,Heazell2010}。临床医生往往会以量表问卷的形式向孕妇采集风险因子信息,
对孕妇PE的病发可能进行初筛\cite{risks},如\autoref{fig:risk}所示。常见的PE风险因子及其可能的影响如\autoref{tab:riskfactors}所示。
\begin{figure}[htbp]
    \centering
    \includegraphics[width=.5\linewidth]{intro/risk}
    \caption[英国胎儿医学基金会使用的PE风险因子评估量表]{\label{fig:risk}英国胎儿医学基金会使用的PE风险因子评估量表\cite{risks}}
\end{figure}

\begin{center}
    \zihao{5}
	\begin{longtable}{m{2.5cm}<{\centering}m{13cm}<{\centering}}
		\caption{\label{tab:riskfactors}常见的PE风险因子}\\
        \topline
        \colorhead\textbf{风险因子} & \textbf{可能的影响}\\
        \midline
        \endfirsthead
        \caption[]{(续)}\\
        \midline
        \colorhead\textbf{风险因子} & \textbf{可能的影响}\\
        \midline
        \endhead 
        \midline
        \endfoot
        \bottomline
        \endlastfoot
        \colorrowa 妊娠年龄    &  PE的病发可能随妊娠年龄升高而增加\cite{Duckitt2005,FIGO,Yogev2010,Poon2010}。    \\
        \colorrowc 胎产次/既往史&    首次分娩孕妇的PE病发可能比无PE既往史的二胎孕妇大\cite{Lee2000,Duckitt2005,Coonrod1995,Robillard1993,Sonia2009}。   \\
        \colorrowa 妊娠间隔 & 多次妊娠的间隔时间过长或过短都会在一定程度上增加PE发生的风险\cite{Rousso2002,Duckitt2005,Conde2007,Mignini2016,Rolv2002}。\\
        \colorrowc 辅助生殖 & 通过辅助生殖技术受孕的孕妇的PE病发可能增加\cite{Jackson2004,Trogstad2009,Martin2016}。\\
        \colorrowa 家族史 & PE在一定程度上显现出家族性易感性\cite{ARNGRIMSSON1990,OAG9,Williams2011,Cincotta1998,FIGO}。\\
        \colorrowc 肥胖 & 若孕妇出现肥胖症状($BMI > 30 kg/m^2$),其PE病发风险会增加\cite{Duckitt2005,Williams2011,FIGO,Zintzaras2006,Sebire2001}。\\
        \colorrowa 种族&不同民族、区域、肤色的孕妇的PE病发可能存在一定的差异\cite{Ghosh2014,Khalil2013}。\\
        \colorrowc 并发症 & 当孕妇已患有某些疾病或感染某些症状时,其PE病发可能也会增加\cite{FIGO,Ray2016,OAG9,Lee2000,Garner1990,Martinell1990,Stamilio2000,Dreyfus2001,Marchetti2016}。\\
	\end{longtable}
\end{center}
\vspace{-1cm} 
二、生物标志物筛查

除上述量表外,另一种筛查PE的方法将孕妇的PE风险因子、特定病史的先验风险及其多项生物物理、生物化学检测结果相结合,基于贝叶斯定理估计该孕妇的PE病发风险\cite{FIGO}。
这一过程应用了参数估计中的极大后验概率估计原理(Maximum a Posteriori,MAP)\cite{Qiu2012},通过观测值及观测值的所属类别的后验概率,判断当前值最可能属于哪一类别并进行分类决策。
如基于MAP的二分类过程可以表示为
\begin{equation}
    \label{equ:maxap}
    H_{x}=
    \left \{
    \begin{aligned}
        &H_{0}, \text P(H_{0}|x)&≥P(H_{1}|x), \\
        &H_{1}, \text P(H_{0}|x)&<P(H_{1}|x),
    \end{aligned}
    \right.  
\end{equation}
其中,$P(H|x)$为观测值$x$属于$H$的后验概率。

上述筛选过程中使用到的孕妇生理、生化检测结果统称为生物标志物。其中,生理检测参数主要包括BP、子宫动脉搏动指数(Uterine artery pulsatility index, UTPI)等;生化检测参数则种类繁多,且新参数层出不穷\cite{Rene2008,Zhong2015,Zeisler2016,Rana2012}。
这里选取妊娠相关血清蛋白-A(Pregnancy-associated plasma protein A,PAPP-A)与胎盘生长因子(Placental growth factor,PLGF)为代表进行介绍。

1. BP

由于PE是妊娠期高血压疾病的一种,BP一直是临床用于PE监测、诊断的重要指标之一\cite{OAG9,HDASOM,2000s1}。在进行BP测量时,应至少对孕妇的同一手臂测量两次。若两次的测量结果的SBP超过$90mmHg$和(或)DBP超过$90mmHg$,则将该孕妇诊断为高血压;
对首次发现血压升高的孕妇,应至少间隔4小时后再次测量确认\cite{OAG9}。此外,国际妇产科联盟也推荐使用平均动脉压(mean arterial pressure,MAP)作为实际标定诊断中的筛查指标\cite{FIGO},其计算方法如\autoref{equ:map}所示
\begin{equation}
    \label{equ:map}
    MAP=DBP+(SBP-DBP)/3
\end{equation}

Leona C.Y. Poon等人对5590名单胎孕妇的一项研究表明,单独使用MAP检测PE的检出率为38\%;而将孕妇病史等风险因子与MAP结合使用时,PE的检出率可达63\%,其假阳性率为10\%\cite{Poon2008}。
Stamilio等人的另一项研究也发现,孕妇在初次产前检查发现MAP大于$90mmHg$与其PE病发的相关性显著\cite{Stamilio2000}。

2. UTPI

UTPI是国际妇产科联盟与妇产科超声学会推荐对PE进行筛查的参数之一\cite{FIGO,Sotiriadis2019}。
UTPI本质上也是一种血液动力学参数,其基本定义与MAP类似\cite{Cnossen2008}
\begin{equation}
    \label{equ:utpi}
    UTPI=\frac{A-B}{M}
\end{equation}
其中,$A$为子宫动脉收缩期血液峰值流量,$B$为子宫动脉舒张末期血液流量,$M$为该期间内的血液平均流量。

常与UTPI一起检测使用的还包括子宫阻力指数(resistance index,RI)、子宫动脉收缩压/舒张压比(systolic to diastolic ratio,SDR)及切迹相关参数等\cite{Cnossen2008}。
\autoref{fig:utpi}展示了一例孕早期经腹多普勒超声检查子宫动脉的结果\cite{Sotiriadis2019}。
\begin{figure}[htbp]
    \centering
    \includegraphics[width=.6\linewidth]{intro/utpi}
    \caption{\label{fig:utpi}孕早期经腹多普勒超声检查子宫动脉}
\end{figure}

Jeltsje S. Cnossen等人\cite{Cnossen2008}的一项回顾研究表明,在孕妇妊娠$11^{+0}-13^{+6}$周时,若其UTPI数值上升或出现子宫动脉舒张早期切迹等现象,该孕妇PE病发的可能性将增加\cite{OAG9,Plasencia2008}。

3.PAPP-A

PAPP-A是由细胞滋养层分泌的一种金属蛋白胰岛素生长因子结合蛋白,在胎盘的生长发育中起着重要的作用。PE已被证明与低水平的PAPP-A循环有关\cite{FIGO}。但相关临床研究表明,仅使用PAPP-A单指标无法准确预测PE\cite{Smith2002}。
因此,PAPP-A常与其他检测参数一起配合使用\cite{Poon2009,Tan2018,Ray2018}。 

4. PLGF

PLGF是由绒毛状细胞滋养层膜细胞滋养层合成的一种糖基化二聚糖蛋白,具有血管生成合血管修复的功能。临床研究显示,PLGF的血管生成功能在妊娠过程中发挥很大作用,PLGF水平或其抑制受体水平的变化可能与PE的病发有关\cite{Levine2004,Ahmad2004}。
同时,临床证据表明,在孕早期患有PE的孕妇其PLGF的浓度较正常妊娠孕妇更低\cite{Chau2017}。

2015年,Zhong等人\cite{Zhong2015}对多项血清生化指标在PE预测方面的性能进行了比较,结果显示,PLGF的综合性能优于PAPP-A等其他血清生化指标。
2019年,Duhig等人\cite{Duhig2019}对多名疑似PE的孕妇进行了追踪实验,在整个实验期间一直持续性地检测孕妇的PLGF水平(总实验人数$N=1019$,实验组人数$N_{PE}=573$,对照组人数$N_{Control}=446$)。
结果显示,实验组的PE诊断时间的中位数为4.1天,而对照组仅为1.9天,这证明PLGF可以有效的缩短PE确诊的时间。
鉴于此,\textbf{FIGO组织特别将胎盘生长因子推荐为PE检测首选生化指标}\cite{FIGO}。

\subsection{检测设备}
本小节从现阶段在临床使用的标准医疗设备与新兴微型化智能设备等两个方面对PE检测设备进行了介绍。

一、标准医疗设备

目前临床所使用的PE筛查检测设备主要以检测生化标志物为主,国内外医疗器械公司均研发推出了一系列的软硬件综合分析系统。但整体而言,国内医疗器械设备公司研发起步晚,较国外同类型设备而言,可检测指标数目也更少。

1. 德国Thermo Fisher Scientific公司的BRAHMS系列产前检测设备

德国Thermo Fisher Scientific公司一直致力于对各种生物标志物检测的研究中,其公司BRAHMS系列的KRYPTOR GOLD与KRYPTOR compact PLUS\cite{B·R·A·H·M·S2021}等型号设备可对多种生物标志物的进行检测。
除上面介绍过的PAPP-A与PlGF外,这些设备还可对游离雌三醇(Unconjugated estriol, uE3)、甲胎蛋白(alpha fetoprotein,AFP)、人绒毛膜促性腺激素(Human chorionic gonadotropin, HCG)及
可溶性fms样酪氨酸激酶-1(soluble fms like tyrosine kinase 1, sFlt-1)等多种PE生物标志物的进行检测,
满足PE的早期筛查及诊断等多种应用场景,如\autoref{fig:B·R·A·H·M·S}所示。
同时该公司还提供了与硬件配套的Fast Screen pre I plus$^\text{TM}$综合软件分析系统,可结合检测结果对孕妇PE病发可能进行风险评估。

\begin{figure}[htbp]
    \centering
    \subfigure[KRYPTOR GOLD]{
    \includegraphics[width=5.5cm]{intro/brahms-kryptor-gold-686}
    }
    \quad
    \subfigure[KRYPTOR compact PLUS]{
    \includegraphics[width=5.5cm]{intro/brahms-kryptor-compact-plus-686}
    }
    \caption{\label{fig:B·R·A·H·M·S}Thermo Fisher Scientific公司的KRYPTOR检测设备}
\end{figure}
2.美国PerkinElmer公司的产前检测设备

为推进早期临床检测在医疗领域的应用,美国珀金埃尔默公司(PerkinElmer)也提供了较为全面的筛查和诊断解决方案组合。该公司的AutoDELFIA系列、
VICTOR2系列及DELFIA Xpress系列的多款免疫荧光分析仪平台,可应用于多种孕期综合并发症的临床检测之中,如\autoref{fig:PerkinElmer}所示。这些设备可对包括AFP、hCG、PAPP-A、PlGF、
sFlt-1、uE3等多种PE生物标志物进行检测标定。此外,该公司也配套研发了LifeCycle软件分析系统,可实现从样品接收、检验检测、风险评估及产出综合报告的全自动分析工作流程。
\begin{figure}[h]
    \centering
    \subfigure[VICTOR2 D荧光检测仪]{
    \includegraphics[width=5.5cm]{intro/Victor-2D-72ppi}
    }
    \quad
    \subfigure[DELFIA® Xpress免疫分析仪平台]{
    \includegraphics[width=5.5cm]{intro/dx}
    }
    \caption{\label{fig:PerkinElmer}珀金埃尔默公司的两款检测设备}
\end{figure}

3. 中国宁波奥丞生物科技有限公司的荧光检测设备

作为中国新兴的医疗器械设备公司,奥丞致力于提供专业的妇幼健康临床诊断解决方案,力求通过提供可靠、快速与便捷的体外诊断产品为诊疗提供精准的检测结果。
目前,奥丞公司提供了微小型床边诊断设备与大型实验室标定两种类型的设备,多款化学荧光免疫平台产品均支持对PE生物标记物PLGF与sFlt-1等两种指标的检测\cite{aucheer2021}
,如\autoref{fig:aucheer}所示。
\begin{figure}[h]
    \centering
    \subfigure[微流控荧光免疫定量检测系统 iSort300]{
    \includegraphics[width=6cm]{intro/isort300}
    }
    \quad
    \subfigure[全自动化学发光免疫检测系统 Shine i1910]{
    \includegraphics[width=6cm]{intro/shinei1910}
    }
    \caption{\label{fig:aucheer}奥丞生物科技公司的荧光免疫平台}
\end{figure}


二、微型智能设备

近年来随着移动智能医疗和可穿戴式设备的发展,通过微型智能设备实现对PE相关指标的实时、动态、多场景检测也逐渐成为了新的研究热点。
但整体而言,基于智能设备的PE检测设备仍处于研发阶段,能真正实现多场景检测的成熟的PE分析系统尚未出现。

\begin{figure}[htbp]
    \centering
    \includegraphics[width=.6\linewidth]{intro/mobile}
    \caption[基于智能穿戴设备的血压检测系统框架图]{\label{fig:mobile}基于智能穿戴设备的血压检测系统框架图\cite{Marin2019,Marin2020}}
\end{figure}

2019年,Iuliana Marin等人\cite{Marin2019,Marin2020}通过一款智能腕部血压检测穿戴设备对孕妇的血压数据进行了监测。他们在综合考虑孕妇的年龄、体重等风险因素的基础上,通过维特比动态规划算法(Viterbi algorithm),分析决策孕妇的PE
病发可能,该分析系统原理框架图如\autoref{fig:mobile}所示。Iuliana Marin团队利用该设备
对多名孕妇的进行了测试($N=105$)。结果显示,最终的生成模型可以达到总体准确率80\%,其中敏感性为92.5\%,特异性为72\%\cite{Marin2019}。

\subsection{分析技术}
由于现代化医学检测设备的普及及医学信息化技术的发展,医学诊断过程中也开始涌现出海量检测数据。这些激增的数据也促进了计算机领域机器学习(Machine learning, ML)技术在医学相关问题上的应用与创新发展。
近年来,专家学者们也逐渐开始将ML中的相关技术与算法应用于PE的识别、判断及预测等问题的研究中。这些研究可按应用研究方向大体分为数据挖掘与分类聚类分析等大两类\cite{Mehta2016}。

一、数据挖掘

数据挖掘,又称为关联规则,力求发现多项数据之间的隐藏的逻辑与关系\cite{Han2006}。在PE的相关研究中,利用数据挖掘技术从血浆代谢物、蛋白质及RNA等生化指标中筛选出与PE病发相关性较强的成份一直是此类研究的热点。
这些研究实现了数据降维,同时也揭示了数据之间隐藏的联系。

2005年,Louise C. Kenny等人\cite{Kenny2005}从利用基因遗传算法(genetic programming, GP),通过血浆中特定代谢物成份来识别PE病发的孕妇($N_{PE}=87$,$N_{Control}=87$)。在对受试孕妇的血浆成份进行了分析后
,他们通过GP算法最后训练得到了仅使用三种代谢物峰值变量的PE预测模型。经验证,该模型可以较好的区分PE病发孕妇与正常孕妇,其灵敏度高达100\%,特异性高达98\%。

2018年,Liron Yoffe等人\cite{Yoffe2018}为寻找能有效区分识别PE的转录RNA,从孕妇妊娠前三个月的血浆中,对非编码循环RNA的丰度进行了分析($N_{PE}=75$,$N_{Control}=75$)。在对非编码RNA测序后,他们确定了实验组与对照组
之间差异表达的25个RNA,并基于该结果训练生成了一个逻辑回归预测模型。经验证,该模型预测PE的曲线下面积(area under curve,AUC)数值可达0.86。

2018年,Muhlis Tahir等人\cite{Tahir2018,Tahir2018-2}对比了神经网络(neural networks, NN)和深度学习(deep learning, DL)等两类算法在预测妊娠期孕妇PE病发的风险水平的结果($N=1077$)。
他们使用粒子群优化(particle swarm optimization, PSO)算法进行特征选择,将原始数据集的17个参数缩减至9个。
通过留一法(leave one out,LOO)验证表明,由DL训练所得的模型具有95.12\%的识别准确率,使用缩减后的数据集还可使准确性进一步提升至95.68\%。而使用NN算法时,模型识别的准确性亦可高达96.66\%。

2020年,Jose F Carre˜no等人\cite{Carreno2020}基于蛋白质组学的数据,比较了降维(dimension reduction,DR)与时间序列总结(time-series summary,TSS)等两类算法对PE的预测的效果($N=202$)。
DR算法包括帝国竞争与基因集簇等两种算法,而TSS算法包括全局平均与三点平均(对应孕前中晚期数据)等两种。
两类算法在两个独立数据集上达到了约90\%的准确性。同时他们发现,较孕晚期的蛋白质组学数据而言,使用孕早期与孕中期的数据进行PE预测的结果会更准确。

二、 分类与聚类分析

分类分析(classification)是按照某种标准给对象一定的数据标签,再根据数据标签来区分归类的过程,也即通过现有数据建立模型预测新数据所属类别的过程。
而聚类分析(clustering)是在没有数据标签的情况下,找出对象之间存在着的聚集性原因的过程,也即按照一定规则将所有数据分组,
使每组内的数据尽可能相似并有别于其他组数据的过程\cite{Han2006}。
分类分析与聚类分析是PE在ML领域最活跃的方向,而前文介绍过的PE风险因子则是此类分析下最为广泛使用的研究数据。

2017年,Pia M. Villa等人\cite{Villa2017}通过贝叶斯聚类算法,通过PE风险因子的数据对受试孕妇进行了聚类分析($N=903$)。他们还计算了聚类分析得到的各簇的PE病发可能。
结果表明,PE的患发可能随孕妇具有的PE风险因子数量增加而呈指数增长。同时,PE病发的程度也往往随孕妇具有的风险因子的种类差异而呈现不同。

2020年,Ivana Mari{\'{c}}\cite{Maric2020}等人利用弹性网(elastic net,EN)算法与梯度提升(gradient boosting,GB)算法从67项PE的风险因子等中训练了PE预测模型与早发PE预测模型($N=16 370$)。经验证,PE预测模型的AUC数值可达0.79,识别假阳性率为8.1\%;早发PE预测模型的
AUC高达0.89,真阳性率为72.3\%,假阳性率为8.8\%。

2020年,Herdiantri Sufriyana等人\cite{Sufriyana2020-1}基于多项PE风险因子与包括sFlt-1、UPTI与PlGF的多项生化指标对孕妇的PE病发可能进行了研究($N_{PE}=66$,$N_{Control}=29$)。
他们发现利用决策树(decision tree,DT)算法得到的PE识别模型具有最佳分类效果,可实现100\%的识别精确度与95\%的敏感度。在最佳识别模型下,贡献度较大的的特征包括体重、BMI、UPTI、sFlt-1、PlGF及sFlt-1与PlGF比值等。
同年,Sufriyana团队也就上述风险因子与生化指标与PE相关性进行了研究\cite{Sufriyana2020}。他们在来自印度尼西亚孕妇的包

\begin{landscape}
    \zihao{-5}
	\begin{longtable}{m{1cm}<{\centering}m{4cm}<{\centering}m{3cm}<{\centering}m{5.5cm}<{\centering}m{5.5cm}<{\centering}m{2cm}<{\centering}}
		\caption[使用ML技术进行PE相关问题的多项研究小结]{使用ML技术进行PE相关问题的多项研究小结}\\
		\label{tab:AIinPE}\\
		\topline
        \colorhead\textbf{时间}&\textbf{研究者}&\textbf{主要任务}&\textbf{涉及的机器学习方法}&\textbf{涉及参数}&\textbf{数据规模}\\
        \midline
        \endfirsthead
        \caption[]{(续)}\\
        \midline
        \colorhead\textbf{时间}&\textbf{研究者}&\textbf{主要任务}&\textbf{涉及的机器学习方法}&\textbf{涉及参数}&\textbf{数据规模}\\
        \midline
        \endhead 
        \midline
        \endfoot
        \bottomline
        \endlastfoot
        \colorrowa 2005&Louise C. Kenny\cite{Kenny2005}&降维、分类&基因遗传算法&多种血浆代谢物&87+87\\
        \colorrowc 2009&Costas K. Neocleous\cite{Neocleous2009}&降维、分类&神经网络&PE风险因子及生化指标&6838\\
        \colorrowc 2016&Mário W. L. Moreira\cite{Moreira2016}&降维、分类&贝叶斯网络&PE风险因子及多项生理症状&164\\
        \colorrowa 2017&Pia M. Villa\cite{Villa2017}&聚类分析&贝叶斯聚类算法&PE风险因子&903\\
        \colorrowc 2018&Muhlis Tahir\cite{Tahir2018,Tahir2018-2}&降维、分类&粒子群优化算法&PE风险因子&1077\\
        \colorrowa 2018&Liron Yoffe\cite{Yoffe2018}&降维、分类&逻辑回归&非编码循环RNA&75+75\\
        \colorrowc 2018&Antonieta Martínez-Velasco\cite{Martinez2018}&分类&多种ML算法&PE风险因子及多项其他指标&269+1365\\
        \colorrowa 2019&Jong Hyun Jhee\cite{Jhee2019}&模式识别、聚类分析&\tabincell{c}{逻辑回归、决策树、\\朴素贝叶斯、随机森林等}&PE风险因子及多项其他指标&11006\\
        \colorrowc 2020&Jose F Carre˜no\cite{Carreno2020}&降维、分类&\tabincell{c}{帝国竞争算法、基因集簇算法、\\时间序列总结算法}&多种蛋白质组学物质&202\\
        \colorrowa 2020&Ivana Mari{\'{c}}\cite{Maric2020}&分类&弹性网算法&PE风险因子&16370\\
        \colorrowc 2020&Oknalita Simbolon\cite{Simbolon2020}&分类&投票&PE风险因子&402\\
        \colorrowa 2020&Herdiantri Sufriyana\cite{Sufriyana2020-1}&分类&决策树&PE风险因子及生化指标&66+29\\
        \colorrowc 2020&Herdiantri Sufriyana\cite{Sufriyana2020}&降维、分类&随机森林算法&PE风险因子及多项其他指标&3318+19883\\
        \colorrowa 2021&Rong Guo\cite{Guo2021}&降维、分类&\tabincell{c}{集成学习、 决策树、\\自适应增强、多层感知机}&胎盘mRNA&157+173\\
	\end{longtable}
\end{landscape}

\noindent
含95个特征的数据集上建立了PE识别模型($N_{PE}=3318$,$N_{Control}=19883$)。结果表明,
由17项特征生成的随机森林模型对PE具有最好的预测分析效果。

\autoref{tab:AIinPE}总结了近年来使用ML技术对PE相关问题的多项研究。从\autoref{tab:AIinPE}可以发现,\textbf{
利用ML技术分析前小节提到的PE各项风险因子、生物标志物等参数是目前炽手可热的多学科交叉研究点,但利用人体电生理参数进行分析处理的研究目前还较为稀少}(已在\autoref{tab:AIinPE}进行了突出显示)。

\subsection{存在的不足与分析}
综合上述分析,目前已有很多基于PE识别与诊断的相关研究。随着对PE的了解不断深入,临床对PE已经具有较为成熟的识别技术与检测手段。借助相关专业检测仪器设备,通过风险因子与生物标记物筛查等方法,
现阶段已经可以对PE进行较高准确率的识别与诊断。但上述检测方法仍然存在着一定的缺陷与不足。

综合上述分析可知,随着人们对PE的了解不断深入,现阶段已经具有较为成熟的PE识别技术与检测手段。目前已有很多基于PE识别与诊断的相关研究。

一、风险因子筛查

风险因子可以对孕妇罹患PE的可能性进行一定程度上的初筛,但整体而言,仅通过风险因子进行筛选,识别与诊断的准确性有限。此外,风险因子往往是对孕妇某些特性的静态描述,无法体现
整个孕期内的任何动态变化,具有一定的滞后性。

二、生物标志物筛查

相较而言,生物标志物筛查对PE的识别与诊断有着更高的准确性与可靠性。但这些生物标志物(如血清妊娠相关蛋白A、胎盘生长因子等)必须借助专业设备在医院进行有创采样检测,
对检测场所、操作流程、检测成本等多方面均有着较高要求。

三、机器学习分析技术

如\autoref{tab:AIinPE}所示,此前诸多学者使用机器学习方法进行分析时,使用的数据往往是获取有着较大的难度与较高的专业壁垒的蛋白质、mRNA等生物标志物,很少使用心电、脉搏波等可方便获取的人体电生理信号等指标。

鉴于现有技术具有上述不足,实现对PE的便捷、快速、无创、准确的临床诊断乃至全孕期内的动态监测无疑有着广阔的研究空间与应用前景。
针对PE的识别与检测,可按以下思路进行探索性研究:

\textbf{一、寻求更好的能够表征PE的参数指标}

根据现有的PE的发病机制的相关理论,PE会导致孕妇多器官、多系统的病生理变化。存在着从这些受影响的器官、系统中挖掘出新的PE表征参数的理论可能性与实践可行性。

\textbf{二、利用机器学习领域的相关算法建立更好的分析模型}

将机器学习分析技术与PE的识别分析结合起来,综合使用多种机器模型、算法乃对上述指标数据进行分析,最终分析模型并进行对比及优化。

寻求更好的能够表征PE的参数指标无疑是重中之重,是后续分析研究得以开展的基础。由于PE会引起全身小动脉痉挛,必然会引起血液动力学上的变化。
而光电容积脉搏波(Photoplethysmography, PPG)是心脏周期性搏动的体现,包含了有关人体血液微循环方面的更富细节的生理信息\cite{PPGYY}。
由于血管直径富于变化,脉搏波到达之处血液流动速率也将发生一定的脉动变化,因此PE导致的血液动力学变化理论上可以从容积脉搏波上得到体现。
此外,光电容积脉搏波的采集可便捷、快速、无创进行,具有采集过程定位简单、易于操作,采集得到的信号质量高、稳定性强等优点。
\textbf{鉴于此,本研究使用PPG信号对PE的识别进行相关研究。}

\subsection{脉搏波在PE领域研究现状}
近年来,基于光电容积脉搏波的PE的研究也有了进一步的发展。下面将按传统指标与新型指标两大类对其介绍,其中,传统指标是指已经在诸多基于PPG的研究中得到应用的特征参数;而新型指标指代
最近几年在PE研究领域被相关学者新提出的PPG特征参数。而参数的具体定义参见本研究第三章。

一、传统指标

2008年,KARLIJN VOLLEBREGT等人\cite{KARLIJN2008}研究了使用脉搏波数据进行PE初筛的可能。通过对健康孕妇($N=223$)的跟踪研究表明舒张压、心输出量与脉搏波增强指数均对PE预测模型有明显影响。结合PE风险因子,最终得到的模型
ROC数值可达0.95,其中敏感性90\%、特异性86\%。

2012年,Kathleen Tomsin等人\cite{Tomsin2012}对静脉脉搏波传播速度(Pulse Wave Velocity,PWV)进行了研究。研究结果显示在正常妊娠期孕妇所有脏器的PWV均呈现逐渐升高的趋势。
但PE的早期(n=12)及晚期患者(n=14)的PWV数值较正常孕妇(n=16)数值更低。Ira Bernstein等人\cite{Ira2014}、MIHAELA VIVIANA IVAN等人\cite{VivianaIvan2018}及Irene Katsipi等人\cite{Katsipi2014}
的多项研究也进一步验证PWV在PE检测识别中的作用。

2014年,Su Fangming等人\cite{Su2014}的研究发现,可以通过光电容积脉搏波无创评估整个妊娠过程中心血管活动情况。他们提取分析了脉搏波增强指数(AIX)、反射指数(RI)、传导时间(PTT)等,
发现这些参数随着妊娠时间的增长在孕期内均有统计意义上的明显改变。

2014年,Nan Han等人\cite{Han2014}研究孕妇心血管系统与子宫动脉血流系统的一致性也得到了与Kathleen Tomsin等人\cite{Tomsin2012}类似的结论。他们以三个月为单位对子宫动脉多普勒超声与指端光电容积脉搏波的检测结
果进行了对比分析($N_{PE}=10$,$N_{Normal}=80$)。结果显示,正常孕妇的子宫动脉阻力指数(UtA RI)与脉搏波反射指数(RI)均随妊娠时间的延长而显著降低,且趋势基本一致。而PE患者的两项参数的数值均明显高于正常孕妇。

2018年,Tammy Y. Euliano等人\cite{Euliano2018}通过无创监测心电图与光电脉搏图对识别PE患者($N_{PE}=25$,$N_{Normal}=31$)进行了研究。他们发现PE患者的脉搏波波峰时长较正常孕妇长、弹性系数较正常孕妇小。
将这些脉搏波特征与其他心电特征训练得到的分类器的ROC数值高达0.907,敏感性78.2\%,特异性89.9\%。

二、新型指标

2018年,Ying Feng等人\cite{Feng2018}基于光电容积脉搏波提出了差异面积比参数(ADR),发现PE会导致孕妇ADR较正常妊娠孕妇低。

2019年,陈婉琳等人\cite{Chen2019}基于光电容积脉搏波提出了光电容积斜率指数(photoplethysmography slope index,PSI),并通过对50例孕妇(子痫患者23例,正常孕妇27例)的脉搏波数据对PSI进行了检验,
结果表明其灵敏性和特异性分别达到 87.0\%和 96.3\%,准确性达到 92.0\%。

上述研究证明,\textbf{PE患者的血流动力学变化与光电容积脉搏波之间确实存在着一定的联系},但更进一步的研究亟待开展
容积脉搏波蕴含着丰富的关于人体血液微循环的信息,除动脉血流量、脉搏波速度指数等参数之外,\textbf{更多特征参数有待挖掘}。
综上,基于机器学习的PE研究可总结为\autoref{tab:PPGinPE}所示。

\begin{center}
	\begin{longtable}{m{1cm}<{\centering}m{3cm}<{\centering}m{3.5cm}<{\centering}m{6.5cm}<{\centering}}
		\caption{基于脉搏波的PE研究小结}\\
		\label{tab:PPGinPE}\\
        \topline
        \colorhead\textbf{时间}&\textbf{研究者}&\textbf{涉及的脉搏波参数}&\textbf{主要研究结论}\\
        \midline
        \endfirsthead
        \caption[]{(续)}\\
        \midline
        \colorhead\textbf{时间}&\textbf{研究者}&\textbf{涉及的脉搏波参数}&\textbf{主要研究结论}\\
        \midline
        \endhead 
        \midline
        \endfoot
        \bottomline
        \endlastfoot
        \colorrowa 2008    &   KARLIJN VOLLEBREGT\cite{KARLIJN2008}    &   AIX     &   AIX对PE预测模型有明显影响。\\
        \colorrowc 2012    &   Kathleen Tomsin\cite{Tomsin2012}    &   PWV     &   PE患者的PWV数值较正常孕妇数值更低。 \\
        \colorrowa 2014    &   Ira Bernstein\cite{Ira2014}     &   PWV &   \\
        \colorrowc 2014    &   Irene Katsipi\cite{Katsipi2014}     &   PWV &   \\
        \colorrowa 2014    &   Nan Han\cite{Han2014}     &   RI &  PE患者的RI数值明显高于正常孕妇。 \\
        \colorrowc 2014    &   Su Fangming\cite{Su2014}    &   AIX、RI、PTT    &   这些参数在孕期内有统计意义上的明显改变。\\
        \colorrowa 2018    &   MIHAELA VIVIANA IVAN\cite{VivianaIvan2018}     &   PWV &   \\
        \colorrowc 2018    &   Tammy Y. Euliano\cite{Euliano2018}     &   波峰时长、弹性系数 &   PE患者的脉搏波波峰时长较正常孕妇长、弹性系数较正常孕妇小\\
        \colorrowa 2018    &   Ying Feng\cite{Feng2018}    &   ADR &  PE患者的ADR数值较正常孕妇数值更低。 \\
        \colorrowc 2019    &   陈婉琳\cite{Chen2019}     &   PSI &   \\
	\end{longtable}
\end{center}

\section{研究目标与研究内容}

\subsection{研究目标}
设计研发新型光电容积脉搏波形态学特征参数,构建通用的光电容积脉搏波的描述特征集合。在此基础上,通过特征筛选、压缩等算法提取出对PE有一定甄别能力的特征子集,
使用机器学习算法构建出PE筛选识别模型,并对最终模型的性能进行验证评估。
\subsection{研究内容}
全文研究内容可分为数据获取、信号预处理与特征参数提取及PE甄别模型训练构建与评估等具体三个部分,每部分具体研究内容包括:
% 如\autoref{fig:study_details}所示。
% \begin{figure}[htbp]
%     \centering
%     \includegraphics[width=.8\linewidth]{intro/study_details}
%     \caption{\label{fig:study_details}研究内容的总体结构框架}
% \end{figure}

一、数据获取

介绍本研究采用的数据来源(自行设计实验方案获得),数据格式等基本信息。

二、信号预处理与特征参数提取

介绍光电容积脉搏波的基本处理流程及涉及算法,介绍通用的脉搏波描述特征参数,介绍原创提出的脉搏波特征参数。

三、PE甄别模型训练构建与评估

综合运用机器学习算法,对上述脉搏波描述参数进行分析,建立PE的甄别模型,评估涉及的脉搏波特征参数及模型性能。

\textbf{各章节的具体内容安排如下:}

第一章是绪论。介绍PE的研究背景,PE的导致的症状及危害,梳理了目前临床已应用的检测方法及指标,分析各项方法及指标的缺陷与不足,并引入使用光电容积脉搏波甄别判断PE的可能性。
随后也补充介绍了一些光电容积脉搏波在PE领域的应用及研究。最后确定提出了本文的研究目标与内容。

第二章是PE及光电容积脉搏波的生理学基础。分别从PE对孕妇各器官的影响及危害,及脉搏波的产生原理、采集原理、特征特性、生理及非生理模型及描述本质等方面对光电容积脉搏波的研究现状也进行了介绍。

第三章是光电容积脉搏波的预处理。介绍了本研究所使用的数据来源,包括实验设备、数据采集规范、完整实验流程、数据导出及复核等。对被试孕妇的人口信息学特征
进行了基本分析。对脉搏波的一般预处理流程进行了介绍,包括滤波处理、波形检测、去除基线漂移、样条插值及标准化等步骤。特别地,详细介绍了本研究提出的一种基于策略与机制
分离的初筛-复核-投票的脉搏波波形检测算法。此外,本章也对脉搏波的特征描述进行了介绍,列举了常见了脉搏波时域特征,随后介绍了本研究进行的新型描述特征的研发工作,并提出了几种
描述脉搏波间的差异的特征参数。

第四章是机器学习概述与数据特征工程。本章介绍了机器学习的基本方法与基本流程,并着重对其中的数据特征工程进行了说明。在第三章初步提出的多项脉搏波描述特征的工作基础上,运用数据特征工程的一般处理方法
以使其满足相关机器学习算法的要求。

第五章是基于光电容积脉搏波的PE识别模型原理。介绍了本研究为通过脉搏波相关特征识别PE而使用的机器学习模型原理。包括机器学习的决策树算法、K均值算法、集成学习的随机森林算法等。

第六章是基于光电容积脉搏波的PE识别模型研究结果与性能分析。介绍了第五章中介绍的各机器学习模型的识别PE的效果与性能,结合数据及实验背景,对其中的一部分现象也给出了分析。

第七章是总结与展望。对本论文的全部研究工作进行系统性总结,阐述本论文的创新工作点,并对下一阶段的研究工作内容进行了规划与展望。