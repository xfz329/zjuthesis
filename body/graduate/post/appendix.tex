\cleardoublepage
{
    \chapternonum{附录}

    \appendixsecmajornumbering

    \section{一个附录}
    开发软件及工具

    \begin{figure}[htb]
        \centering
        \includegraphics{example-image}
        \caption{附录中的图片}
        \label{fig:test-appendix}
    \end{figure}

    \section{另一个附录}
    资源文件
    批处理操作及参数设置说明
2022.05.06
1. 目的
	保存当前对PPG读取分析的相关操作,供下次调用快速使用。典型的应用场景为,调整了脉搏波的输出特征后,需要对所有数据进行新一轮的分析,此时脉搏波的定位结果可以完全沿用上一轮分析结果。若上一轮分析结果已被保存,则当前新一轮分析即可批量处理完成。
2. 操作
	a. 设置需要批处理的目录
		批处理是以目录为单位进行。通过批处理->生成模板,选择目录后,即可在该目录下生成空白的配置文件。
	b. 手动调整配置文件的相关参数
		默认的配置文件具有的功能有限。需要根据具体数据灵活调整,由于一般数据文件在整个分析周期内是固定不变的。因此,配置文件设置完成后(特别是对脉搏波波形的取舍设置),可以在整个分析周期内通用。
	c. 选择设置好的配置文件完成批处理
		通过软件选择批处理->开始批处理,选择相应的配置文件,按配置完成批处理过程。
3. 配置参数说明
	 a. 语法
		配置文件必须是合法的Json文件。
	b. 字段说明
		i. path 
			当前需要批处理的目录。无需手动调整。
		ii. PEstate
			当前目录下所有数据的PE状态,值域[0,1]。需要手动设置。
		iii. mode
			当前批处理所需要进行的批处理操作,可自定义设置。目前支持打开文件、算法初步确认、人工手动调整检测结果、导出检测结果、开始特征分析、导出分析特征及分析结果上传等7种操作。其中,后6种文件操作在批处理过程中统一配置。分别通过mode的1位来完成控制。算法初步确认对应mode的最低位。mode的默认值为31,即0b011111,即对当前目录下所有数据文件进行出数据特征上传外的所有操作。
			需要注意的是,可配置的6个操作有着逻辑上的依赖关系,某些配置可能会导致程序无法运行。如出现此种情况,请联系作者(江锋 silencejiang@zju.edu.cn)。
		iv. files
			当前批处理操作所需处理的所有数据文件项。批处理会依次对file中所有数据项进行处理。需要对其中的一定数据项进行自定义配置。
		v. path in files
			需要处理的数据文件名。默认生成,无需修改。
		vi. skipped in files
			是否跳过处理该数据文件,可自定义配置。默认为false,即不跳过。
		vii. pulses in files
			对该数据波形自动分析结果得到的脉搏波波形的删除或新增操作,可自定义配置。默认为空,即接受算法全部处理结果。若需要配置,需要新增一个包含type与points两个字段的Json数据项至pulses。可根据需要配置多个这样的数据项。
		viii. type in pulses in files
			指定对脉搏波波形操作类型,必须配置。可选字段,A与R。前者对应新增(add),后者对应删除(remove)。目前不接受其他字段。
		ix. Points in pulses in files
			定义需要新增或移除的脉搏波的坐标位置,必须是三元数组(对应起点、峰值点及终点),必须配置。其中,峰值点数值决定了当前波形的新增或移除是否成功。新增波形时,三者的数值准确性要求较低,可允10以内的误差。移除时,必须保证坐标位置的准确性。
4. 配置范例及说明
	a. demo	
	b. 说明

}