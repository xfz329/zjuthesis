\chapter{绪论}

\section{引言}

\section{子痫监测的研究现状}

\section{脉搏波的研究现状}

\section{脉搏波时域特征研究现状}
\section{用脉搏波检测子痫的优势}
\section{研究目标与内容}
\subsection{研究目标}
\subsection{研究内容}


\chapter{生理学基础}
\section{引言}
\section{子痫前期的生理学基础}
\section{光电容积脉搏波的生理学基础}
\section{小结}


\chapter{新型脉搏波特征参数及特征集的构建}
\section{引言}
\section{信号预处理}
\section{脉搏波波形检测及纠错}
\section{时域特征参数设计与特征集构建}
\subsection{角度、幅值、长度等}
\section{小结}

\chapter{基于数据特征集的模型构建}
\section{引言}

\section{数据来源}
\section{构建方法分析}
\section{模型构建}
\section{小结}

\chapter{模型评估}
\section{引言}
\section{评估方式与标准}
\section{具体模型表现对比}
\section{小结}

\chapter{低耦合、高拓展的软件综合分析系统的设计与实现}
\section{背景分析}
* 子痫前期检测的特定场景:孕妇、高龄、周期长、时间

* 实验室已有前期硬件研究基础,家俊研究基础

\section{需求分析:模块化、低耦合、高拓展}
* 多场景

* 特征计算拓展

* 数据管理

* 模型识别判断

\section{整体设计框图展示}

\section{各模块具体设计}
* 硬件采集支持

* 对数据源的拓展——多种数据格式均可兼容

* 对数据预处理的拓展——波形定位算法可拓展、纠错算法可拓展

* 对数据类型的拓展——目前只涉及脉搏波,但可方便拓展ECG等信号

* 对特征集构建拓展——方便随时研发新的特征参数进入特征集

\section{实现展示与测试}
界面
输入输出
\section{小结}

\chapter{总结与展望}
\section{研究工作总结}
\section{主要创新点}
* 提出了多种新型脉搏波特征参数,从时域形态特征方面构建了一般通用的脉搏波特征集

* 从特征集中筛选出对子痫前期具有识别能力的有效特征集合,并基于有效特征集合通过机器学习方法构建了子痫前期综合预测模型

* 设计并实现了一种低耦合、高拓展的方便类似研究工作开展的软件综合分析系统

\section{下一阶段工作展望}


测试

本模板根据浙江大学研究生院编写的《浙江大学研究生学位论文编写规则》~\cite{zjugradthesisrules,zjuthesis,tikz},
在原有的 zjuthesis 模板~\cite{zjuthesis}基础上开发而来。

本模板的本科生版本\cite{zjuthesisrules}得到了浙江大学本科生院老师的支持与审核,

\begin{equation}
    \sqrt { \frac { 1 + \sqrt { 5 } } { 2 } + 2 } - \frac { 1 + \sqrt { 5 } } { 2 } = \frac { e ^ { - \frac { 2 \pi } { 5 } } } { 1 + \frac { e ^ { - 2 \pi } } { 1 + \frac { e ^ { - 4 \pi } } { 1 + \frac { e ^ { - 6 \pi } } { 1 + \cdots } } } }
\end{equation}

\begin{equation}FSS=\left[k_{1}, k_{2}, \ldots \ldots, k_{n-1}\right]\end{equation}



% \begin{itemize}
%     \item 删除根目录的 ``.latexmkrc'' 文件,否则编译失败且不报任何错误
%     \item 字体有版权所以本模板不能附带字体,请务必手动上传字体文件,并在各个专业模板下手动指定字体。
%         具体方法参照 GitHub 主页的说明。
%     \item 当前的Overleaf默认使用TexLive 2017进行编译,但一些伪粗体复制乱码的问题需要TexLive 2019版本来解决。
%         所以各位同学可以在Overleaf上编写论文时务必切换到TexLive 2019或更新版本来编译,以免产生查重相关问题。
%         具体说明参照 GitHub 主页。
% \end{itemize}


\section{节标题}

我们可以用includegraphics来插入现有的jpg等格式的图片,
如\autoref{fig:zju-logo}所示。

\begin{figure}[htbp]
    \centering
    \includegraphics[width=.3\linewidth]{logo/zju}
    \caption{\label{fig:zju-logo}浙江大学LOGO}
\end{figure}


\subsection{小节标题}


\par 如\autoref{tab:sample}所示,这是一张自动调节列宽的表格。

\begin{table}[htbp]
    \caption{\label{tab:sample}自动调节列宽的表格}
    \begin{tabularx}{\linewidth}{c|X<{\centering}}
        \hline
        第一列 & 第二列 \\ \hline
        xxx & xxx \\ \hline
        xxx & xxx \\ \hline
        xxx & xxx \\ \hline
    \end{tabularx}
\end{table}


\par 如\autoref{equ:sample},这是一个公式

\begin{equation}
    \label{equ:sample}
    A=\overbrace{(a+b+c)+\underbrace{i(d+e+f)}_{\text{虚数}}}^{\text{复数}}
\end{equation}

% \chapter{另一章}


\begin{figure}[htbp]
    \centering
    \includegraphics[width=.3\linewidth]{example-image-a}
    \caption{\label{fig:fig-placeholder}图片占位符}
\end{figure}
