\chapter{总结与展望}
\section{研究工作总结}
本论文研究了通过光电容积脉搏波(PPG)识别判断妊娠孕妇PE病发的可能性。设计了临床数据采集实验,通过标准医疗级设备,采集得到了80名孕妇的有效PPG信号数据,其中,
实验组包含35例PE病发孕妇,对照组包含44例正常孕妇,1例数据因采集时长过短、信号质量太差等原因被剔除。使用U检验方法分析了全部被试的人口统计学特征。
对PPG信号完成了包括滤波处理、波形检测、去除基线漂移、信号重采样与数据标准化等在内的预处理工作。
结合PPG形态特征,设计了描述PPG的时间、幅值、角度、弧度、面积、斜率等多维度新型时域特征参数构建了PPG多维度时域特征集合;设计了描述PPG波形间的差异的时域特征参数
构建了PPG采样序列时域特征集合。在上述两个时域特征集上,
将被试PPG数据的单个波形与全波波形分别作为最小分析单位,利用以随机森林、决策树、K近邻为代表的多种机器学习算法,分别进行了PE识别模型的研究。
通过各模型在测试集上的泛化效果评估其整体性能,评估了PPG时域特征集中的各特征与PE的相关性,分析了两个时域特征集中各特征的重要程度,分析了时域特征与PPG波形的对应关系,
对比分析了两个时域集在PE识别问题上的体现出的描述能力的差异。

总结论文的研究工作,主要包括以下五个方面:

一、采集PE病发孕妇的PPG数据

基于PPG数据对PE病发可能的同类研究在国内外均较为稀少,业内无任何可公开使用的数据库或数据源。
本论文自主设计了临床实验方案,通过标准医疗级设备,在浙江大学医学院附属妇产科医院完成了79例有效的孕妇PPG信号数据的采集工作,其中,
实验组包含35例PE病发孕妇,对照组包含44例正常孕妇;PPG数据采样率为100 Hz,每例数据的采集时长不少于1min。这批实验数据填补了国内空白,也是本研究后续研究的基础。

二、对PPG信号进行预处理

本研究对PPG数据预处理工作包括滤波处理、波形检测、重搏波与切迹检测、去除基线漂移、信号重采样与数据标准化等。
选用了5阶平均滑动滤波器对PPG进行了滤波处理;
通过本研究提出的初筛—复核—决策的新型算法完成波形检测;
使用了一种基于曲率的定位算法对重搏波不明显及切迹发生退化的PPG波形进行定位检测;
使用了线性变换的思想去除了PPG信号中的基线漂移;
通过插值与抽取的方法,调整了原始PPG数据的采样率;
按幅值标准化的方式,对PPG波形进行了标准化处理。

三、设计了PPG新型时域特征参数,构建了PPG时域特征描述集合

对PPG时域特征的设计过程进行了方法学的归纳,提出了PPG特征描述向量的概念,提出了多种基于时间、幅值、角度、弧度、面积、斜率等多维度的新型PPG时域特征;
提出了描述PPG波形间差异的新型时域特征。
在上述多维度特征的基础上,通过左视策略、中视策略与分层策略,构建了PPG多维度时域特征集。
基于PPG原始采样序列的幅值,构建了PPG采样序列时域特征集,并通过重采样策略、补齐策略与截取策略解决了数据维度不匹配的问题。
对两个PPG时域特征集进行了数据集划分、数据缩放及数据降维等机器学习的数据集预处理工作。

四、PE的识别模型的构建与评估

经算法初筛,本论文将决策树、K近邻与随机森林三种算法作为PE识别模型的主要研究方法。
按照只依据单个PPG波形与结合被试所有PPG波形进行PE识别分析的两个角度,在上述两个PPG时域特征集,使用上述三种算法,进行了PE识别模型的研究。
结果表明,按被波形进行PE识别模型研究时,在PPG多维度时域特征集上,由随机森林算法可得到最佳模型(AUC为0.99,训练集准确率95.2\%,测试集准确率97.0\%);
在PPG采样序列时域特征集上,由随机森林算法可得到最佳模型(AUC为0.967,训练集准确率90.0\%,测试集准确率92.5\%)。
在波形进行PE识别模型研究时,在PPG多维度时域特征集上,由随机森林算法可得到最佳模型(AUC为0.952,测试集准确率93.8\%);
在PPG采样序列时域特征集上,由随机森林算法可得到最佳模型(AUC为0.929,测试集准确率87.5\%)。

五、基于PPG的PE识别分析软件系统的设计与实现

设计并实现了一款基于PPG信号数据的PE识别分析软件系统,系统由数据预处理、跨平台客户端、PE识别模型训练生成与云服务器程序等四个功能模块组成,
具有从数据预处理到PE病发状态的识别预测的完整分析功能。
同时,软件系统也进行了多项兼容性设计,支持对核心处理算法、识别模型等进行更新迭代,进一步保证了软件系统的兼容性与拓展性。
经测试验证,软件系统的各项功能均能按照设计预期正常工作。

以上结果表明,本论文构建的PPG多维度时域特征集与PPG采样序列时域特征集可作为PE识别的数据基础,其中,各ML算法模型在PPG多维度时域特征集效果更好。
对比各ML算法模型在测试集上的准确率表明,随机森林算法性能优于决策树与K近邻算法。对比两个特征集中有效特征贡献度分析表明,
两类时域特征集中对随机森林模型贡献度高的特征所对应的PPG形态位置高度相似,且集中出现在PPG主波峰后与下降支末端附近,PPG波形在这些位置的具体形态可能是识别PE的关键。

另一方面,作为以上研究工作的软件载体,本论文为PE的及早诊断与多使用场景而开发的一款模块化的基于PPG的PE识别分析软件系统。
经测试验证,该软件系统具备较好的兼容性与拓展性,具有一定的临床实际意义与应用前景。
\section{主要创新点}
本论文的创新点主要包括以下三点:

一、基于策略与机制分离思想,提出了PPG波形检测的初筛—复核—投票算法

本研究对 PPG 波形的检测过程进行了模式设计,提出了一种新型 SCD 算法,算
法由初筛、复核与决策等三部分组成。在初筛时,通过搜索窗的方法确定波形;在复核时,
通过功率、标准差、波峰相对位置与基线漂移程度等 PPG 形态学特征,区分有效波形与异
常干扰;在决策时,通过加权投票的策略确定 PPG 波形是否为有效波形。SCD 算法可根据
使用场景进行二次开发,支持更改初筛算法、增删复核标准与切换决策策略等设置。经验证,
该算法各模块设计合理,与其他算法对比结果表明,对 PPG 波形的识别准确率高、抗干扰
能力强。SCD 算法为多平台、多研究应用下的 PPG 分析检测的不同场景提供了一定的参考。

二、设计了多种新型PPG时域形态特征,构建了完整的PPG信号时域描述特征集合

本论文对PPG时域特征的设计过程进行了方法学的归纳,提出了PPG特征描述向量的概念,提出了多种基于时间、幅值、角度、弧度、面积、斜率等多维度的新型PPG时域特征;
提出了描述PPG波形间差异的新型时域特征。
在上述多维度特征的基础上,通过左视策略、中视策略与分层策略,构建了PPG多维度时域特征集。
基于PPG原始采样序列的幅值,构建了PPG采样序列时域特征集,并通过重采样策略、补齐策略与截取策略解决了数据维度不匹配的问题。
对两个PPG时域特征集进行了数据集划分、数据缩放及数据降维等机器学习的数据集预处理工作。


三、使用多种机器学习算法构建了PE的识别模型

经过对9种监督学习算法的筛选,本论文将决策树、K近邻与随机森林三种算法作为PE识别模型的主要研究方法。
按照将被试PPG数据的单个波形与全波波形分别作为最小分析单位的两个研究角度,基于上述两个PPG时域特征集,使用决策树、K近邻与随机森林三种算法,进行了PE识别模型的研究。
在将被试PPG数据的单个波形作为最小分析单位进行研究时,将原始数据按波形划分训练模型,判断该波形是否来自PE病发的孕妇;
在将被试PPG数据的全部波形作为最小分析单位进行研究时,将原始数据按被试划分,先训练波形的识别模型,再进行群体决策,判断孕妇是否PE病发。

\section{展望}

本论文研究了通过无创PPG信号利用机器学习算法构建PE识别模型的可行性。虽然取得了一定的成果,但整个研究过程中的一些环节仍存在一些不完善之处,
值得开展进一步地深入和拓展:

一、增加实验数据

更多的实验数据能够训练更好的模型,避免由数据量不足导致的过拟合问题。

二、进一步完善SCD算法

可从初筛时的波形检测算法、复核时使用的标准、决策时使用的策略机制等方面,进一步完善SCD算法。可进一步拓展SCD算法使其也能满足其他人体电生理信号的检测要求。

三、构建更好的PPG特征,特别是与PE相关的特征

PPG的数据特征是模型分析的基础,更好的PPG特征能保证更好的模型性能。此外,也可考虑将PPG的频域特征与非线性特征引入进PE识别模型的构建中。

四、优化模型算法

本研究目前主要研究了监督学习下的决策树、K近邻与随机森林三种算法,对其他算法的挖掘程度不够。其次,也可以考虑将机器学习中的非监督学习算法,深度学习中的时间序列等算法引入PE识别模型的构建中。

五、丰富数据信号种类

探究更多种类的生理信号在PE识别上的应用,通过多信号、多维度、多参数模型的构建提高准确性和稳定性。

六、进一步完善软件系统

增强软件系统的其他功能,实现软件系统真正的云部署。
