\chapter{总结与展望}
\section{研究工作总结}
本论文深入研究了通过光电容积脉搏波(PPG)识别判断妊娠孕妇PE病发的可能性。设计了临床数据采集实验,通过标准医疗级设备,采集得到了80名孕妇的有效PPG信号数据,其中,
实验组包含35例PE病发孕妇,对照组包含44例正常孕妇,1例数据因采集时长过短、信号质量太差等原因被剔除。使用U检验方法分析了全部被试的人口统计学特征。
根据所得PPG信号质量,同时结合后续分析需求,对PPG信号完成了包括滤波处理、波形检测、去除基线漂移、信号重采样与数据标准化等在内的预处理工作,
并为PPG的波形检测提出了一种可在多平台、多研究背景下应用的基于初筛—复核—决策的新型波形检测算法。
结合PPG形态特征,设计了描述PPG的时间、幅值、角度、弧度、面积、斜率等多维度新型时域特征参数;设计了描述PPG波形间的差异的时域特征参数。
在上述时域特征基础上,构建了PPG多维度时域特征集合;将PPG信号的原始采样序列的幅值作为特殊的时域描述特征,构建了PPG采样序列时域特征集合。
评估了PPG多维度时域特征集中的各特征与PE的相关性,初步进行了特征降维处理。提出了按照从单个PPG波形与被试所有PPG波形进行PE识别分析的两个角度。
利用以随机森林、决策树、K近邻为代表的多种机器学习算法,在上述两个时域特征集上,按两个分析角度,分别进行了PE识别模型的研究。
通过各模型在测试集上的泛化效果评估其整体性能。进一步通过随机森林算法进行了特征降维处理,分析了两个时域特征集中各特征的重要程度,分析了较为重要的时域特征与PPG波形的对应关系。
通过上述机器学习模型的性能对比分析了两个时域集在PE识别问题上的体现出的描述能力的差异。

总结论文的研究工作,主要包括以下四个方面:

一、收集PE病发孕妇的PPG数据

基于PPG数据对PE病发可能的同类研究在国内外均较为稀少,业内无任何可公开使用的数据库或数据源。
本论文自主设计了临床实验方案,通过标准医疗级设备,在浙江大学医学院附属妇产科医院完成了80例孕妇的PPG信号数据的采集工作,其中,
实验组包含35例PE病发孕妇,对照组包含44例正常孕妇,1例数据因采集时长过短、信号质量太差等原因被剔除。
其中,PPG数据采样率为100 $Hz$,每例数据的采集时长不少于1 $min$。这批实验数据填补了国内空白,也是本研究后续研究的基础。

二、对PPG信号进行了全面的预处理

本研究对PPG数据预处理工作包括滤波处理、波形检测、重博波与切迹检测、去除基线漂移、信号重采样与数据标准化等。
选用了5阶平均滑动滤波器对PPG进行了滤波处理;
通过本研究提出的初筛—复核—决策的新型算法完成波形检测;
使用了一种基于曲率的定位算法对重博波不明显及切迹发生退化的PPG波形进行定位检测;
使用了线性变换的思想去除了PPG信号中的基线漂移;
通过插值与抽取的方法,调整了原始PPG数据的采样率;
按幅值标准化的方式,对PPG波形进行了标准化处理。

三、设计了脉搏波新型时域特征参数,构建了脉搏波时域特征描述集合

对PPG时域特征的设计过程进行了方法学的归纳,提出了脉搏波特征描述向量的概念,提出了多种基于时间、幅值、角度、弧度、面积、斜率等多维度的新型PPG时域特征;
提出了描述PPG波形间差异的新型时域特征。
在上述多维度特征的基础上,通过左视策略、中视策略与分层策略,构建了PPG多维度时域特征集。
另将基于PPG原始采样序列的幅值,构建了PPG采样序列时域特征集,并通过重采样策略、补齐策略与截取策略解决了数据维度不匹配的问题。
对两个PPG时域特征集进行了数据集划分、数据缩放及数据降维等机器学习的数据集预处理工作。

四、子痫前期的识别模型的构建与评估

经算法初筛,本论文将决策树、K近邻与随机森林三种算法作为PE识别模型的主要研究方法。
按照只依据单个PPG波形与结合被试所有PPG波形进行PE识别分析的两个角度,在上述两个PPG时域特征集,使用上述三种方法,进行了PE识别模型的研究。
结果表明,按被波形进行PE识别模型研究时,在PPG多维度时域特征集上,由随机森林算法可得到最佳模型(AUC为0.99,训练集准确率95.2\%,测试集准确率97.0\%);
在PPG采样序列时域特征集上,由随机森林算法可得到最佳模型(AUC为0.967,训练集准确率90.0\%,测试集准确率92.5\%)。
在波形进行PE识别模型研究时,在PPG多维度时域特征集上,由随机森林算法可得到最佳模型(AUC为0.952,测试集准确率93.8\%);
在PPG采样序列时域特征集上,由随机森林算法可得到最佳模型(AUC为0.929,测试集准确率87.5\%)。
以上结果表明,本文构建PPG多维度时域特征集与PPG采样序列时域特征集可作为PE识别的数据依据,其中,包含多种新型特征的PPG多维度时域特征集效果更好。
更进一步对算法模型中的使用的特征的贡献度分析发现,两类时域特征集中具有PE表征能力的特征出现的位置高度相似,集中出现在在下降支的前端与尾端,这也
表明这些位置的 PPG 形态可能是识别 PE 的关键。


\section{主要创新点}
本论文的创新点主要包括以下三点:

一、基于策略与机制分离思想,提出了脉搏波波形检测的初筛-复核-投票算法

本研究对 PPG 波形的检测过程进行了模式设计,提出了一种新型 SCD 算法,算
法由初筛、复核与决策等三部分组成。在初筛时,通过搜索窗的方法确定波形;在复核时,
通过功率、标准差、波峰相对位置与基线漂移程度等 PPG 形态学特征,区分有效波形与异
常干扰;在决策时,通过加权投票的策略确定 PPG 波形是否为有效波形。SCD 算法可根据
使用场景进行二次开发,支持更改初筛算法、增删复核标准与切换决策策略等设置。经验证,
该算法各模块设计合理,与其他算法对比结果表明,对 PPG 波形的识别准确率高、抗干扰
能力强。SCD 算法为多平台、多研究应用下的 PPG 分析检测的不同场景提供了一定的参考。

二、设计了多种新型脉搏波时域形态特征,构建了完整的脉搏波信号时域描述特征集合

本论文对PPG时域特征的设计过程进行了方法学的归纳,提出了脉搏波特征描述向量的概念,提出了多种基于时间、幅值、角度、弧度、面积、斜率等多维度的新型PPG时域特征;
提出了描述PPG波形间差异的新型时域特征。
在上述多维度特征的基础上,通过左视策略、中视策略与分层策略,构建了PPG多维度时域特征集。


三、使用多种机器学习算法构建了子痫前期的识别模型

经算法初筛,本论文将决策树、K近邻与随机森林三种算法作为PE识别模型的主要研究方法。
按照只依据单个PPG波形与结合被试所有PPG波形进行PE识别分析的两个角度,在上述两个PPG时域特征集,使用上述三种方法,进行了PE识别模型的研究。
结果表明,按被波形进行PE识别模型研究时,在PPG多维度时域特征集上,由随机森林算法可得到最佳模型(AUC为0.99,训练集准确率95.2\%,测试集准确率97.0\%);
在PPG采样序列时域特征集上,由随机森林算法可得到最佳模型(AUC为0.967,训练集准确率90.0\%,测试集准确率92.5\%)。
在波形进行PE识别模型研究时,在PPG多维度时域特征集上,由随机森林算法可得到最佳模型(AUC为0.952,测试集准确率93.8\%);
在PPG采样序列时域特征集上,由随机森林算法可得到最佳模型(AUC为0.929,测试集准确率87.5\%)。
以上结果表明,本文构建PPG多维度时域特征集与PPG采样序列时域特征集可作为PE识别的数据依据,其中,包含多种新型特征的PPG多维度时域特征集效果更好。
更进一步对算法模型中的使用的特征的贡献度分析发现,两类时域特征集中具有PE表征能力的特征出现的位置高度相似,集中出现在在下降支的前端与尾端,这也
表明这些位置的 PPG 形态可能是识别 PE 的关键。


\section{展望}

