\chapter{总结与展望}
\section{研究工作总结}
本论文研究了通过光电容积脉搏波(PPG)识别判断妊娠孕妇PE病发的可能性。首先,设计了临床数据采集实验,通过标准医疗级设备,遵循医学伦理,按照操作规范,
采集得到了80名孕妇的有效PPG信号数据。
其中,实验组包含35例PE病发孕妇,对照组包含44例正常孕妇,1例数据因采集时长过短、信号质量太差等原因被剔除。
其次,使用U检验方法分析了实验组与对照组的人口统计学特征差异。%结果仅有收缩压与舒张压存在统计意义上的显著差别(\textit{p}<0.001)。
随后,对采集得到的PPG信号完成了包括滤波处理、波形检测、去除基线漂移、样条插值与数据标准化等在内的预处理工作。
基于信号质量,确定了从时域开始PPG相关特征的构建工作;在此前的研究的基础上,进一步扩展了描述PPG的时间、幅值、角度、弧度、面积、斜率等多维度新型时域特征参数,
构建了PPG多维度时域特征集合;基于原始采样值序列构建了PPG采样序列时域特征集合。同时也设计提出了描述PPG波形间的差异的时域特征参数。
为人工扩充数据集规模,在上述两个时域特征集上,分别将被试PPG数据的单个波形与全波波形分别作为最小分析单位,
利用以随机森林、决策树、K近邻为代表的多种机器学习算法,分别进行了PE识别模型的研究,通过各模型在测试集上的泛化效果评估其整体性能。
利用随机森林算法,评估了PPG时域特征集中的各特征与PE的相关性,分析了两个时域特征集中各特征的重要程度。
最后,对比分析了两个时域集在PE识别问题上的体现出的描述能力的差异,分析了总结时域特征与PPG波形的对应关系。


总结论文的研究工作,主要包括以下五个方面:

一、采集PE病发孕妇的PPG数据

基于PPG数据对PE病发可能的同类研究在国内外均较为稀少,业内无任何可公开使用的数据库或数据源。
本论文自主设计了临床实验方案,通过标准医疗级设备,在浙江大学医学院附属妇产科医院完成了79例有效的孕妇PPG信号数据的采集工作,其中,
实验组包含35例PE病发孕妇,对照组包含44例正常孕妇;PPG数据采样率为100 Hz,每例数据的采集时长不少于1min。这批实验数据填补了国内空白,也是本研究后续研究的基础。

二、对PPG信号进行预处理

本研究对PPG数据预处理工作包括滤波处理、波形检测、重搏波与切迹检测、去除基线漂移、样条插值与数据标准化等。
选用了5阶平均滑动滤波器对PPG进行了滤波处理;
通过本研究提出的初筛—复核—决策的新型算法完成波形检测;
使用了一种基于曲率的切迹检测定位算法;
使用了线性变换的思想去除了PPG信号中的基线漂移;
通过样条插值算法,提升了原始PPG数据的采样率;
按幅值标准化的方式,对PPG波形进行了标准化处理。

三、设计了PPG新型时域特征参数,构建了PPG时域特征描述集合

基于信号质量,确定了从时域开始PPG相关特征的构建工作;
对PPG时域特征的设计过程进行了方法学的归纳,提出了PPG特征描述向量的概念,提出了多种基于时间、幅值、角度、弧度、面积、斜率等多维度的新型PPG时域特征;
提出了描述PPG波形间差异的新型时域特征。
在上述多维度特征的基础上,通过左视策略、中视策略与分层策略,构建了PPG多维度时域特征集。
基于PPG原始采样序列的幅值,构建了PPG采样序列时域特征集,并通过重采样策略、补齐策略与截取策略解决了数据维度不匹配的问题。
为扩充数据集规模,确定了按被试的单个PPG波形与全部PPG波形进行PE识别模型的研究角度。
对两个PPG时域特征集进行了数据集划分、数据缩放及数据降维等机器学习的数据集预处理工作。

四、PE的识别模型的构建与评估

经算法初筛,本论文将决策树、K近邻与随机森林三种算法作为PE识别模型的主要研究方法。
按照只依据单个PPG波形与结合被试所有PPG波形进行PE识别分析的两个角度,在上述两个PPG时域特征集,使用上述三种算法,进行了PE识别模型的研究。
结果表明,按被波形进行PE识别模型研究时,在PPG多维度时域特征集上,由随机森林算法可得到最佳模型(AUC为0.99,训练集准确率95.2\%,测试集准确率97.0\%);
在PPG采样序列时域特征集上,由随机森林算法可得到最佳模型(AUC为0.967,训练集准确率90.0\%,测试集准确率92.5\%)。
在波形进行PE识别模型研究时,在PPG多维度时域特征集上,由随机森林算法可得到最佳模型(AUC为0.952,测试集准确率93.8\%);
在PPG采样序列时域特征集上,由随机森林算法可得到最佳模型(AUC为0.929,测试集准确率87.5\%)。

五、基于PPG的PE识别分析软件系统的设计与实现

设计并实现了一款基于PPG信号数据的PE识别分析软件系统,系统由数据预处理、跨平台客户端、PE识别模型训练生成与云服务器程序等四个功能模块组成,
具有从数据预处理到PE病发状态的识别预测的完整分析功能。
同时,软件系统也进行了多项兼容性设计,支持对核心处理算法、识别模型等进行更新迭代,进一步保证了软件系统的兼容性与拓展性。
经测试验证,软件系统的各项功能均能按照设计预期正常工作。

以上结果表明,通过PPG信号进行PE的识别分析具有一定的可行性,在本论文构建的PPG多维度时域特征集与PPG采样序列时域特征集数据基础上,
能够通过决策树、K近邻与随机森林算法能够构建出较好的PE识别模型。
对比各ML算法模型在测试集上的准确率表明,随机森林算法构建的识别模型性能优于决策树与K近邻算法。
对比各ML算法模型在两个数据集上的性能表明,PPG多维度时域特征集较PPG采样序列时域特征集更有优势。
对比两个特征集中有效特征贡献度分析表明,两类时域特征集中对随机森林模型贡献度高的特征所对应的PPG形态位置高度相似,
且集中出现在PPG主波峰后与下降支末端附近,PPG波形在这些位置的具体形态可能是识别PE的关键。

另一方面,作为以上研究工作的软件载体,本论文为PE的及早诊断与多使用场景而开发的一款模块化的基于PPG的PE识别分析软件系统。
经测试验证,该软件系统具备较好的兼容性与拓展性,具有一定的临床实际意义与应用前景。
\section{主要创新点}
本论文的创新点主要包括以下三点:

一、提出了一种基于初筛-复核-投票的PPG波形检测算法

本研究对 PPG 波形的检测过程进行了模式设计,提出了一种新型 SCD 算法,算
法由初筛、复核与决策等三部分组成。在初筛时,通过搜索窗的方法确定波形;在复核时,
通过功率、标准差、波峰相对位置与基线漂移程度等 PPG 形态学特征,区分有效波形与异
常干扰;在决策时,通过加权投票的策略确定 PPG 波形是否为有效波形。SCD 算法可根据
使用场景进行二次开发,支持更改初筛算法、增删复核标准与切换决策策略等设置。
经临床及公开数据库验证,该算法对PPG波形的检测准确率高、抗干扰能力强,
为多平台、多研究应用下的PPG分析检测的不同场景提供了一定的参考。

二、构建了PPG信号和采样序列的多维度时域特征集,提取描述PPG波形间差异的时域特征,
利用决策树、K 近邻与随机森林三种算法进行子痫前期识别模型的研究

本论文对PPG时域特征的设计过程进行了方法学的归纳,提出了PPG特征描述向量的概念,
提出了多种基于时间、幅值、角度、弧度、面积、斜率等多维度的新型PPG时域特征;
提出了描述PPG波形间差异的新型时域特征。
在上述多维度特征的基础上,通过左视策略、中视策略与分层策略,构建了PPG多维度时域特征集。
经过算法筛选,利用决策树、K 近邻与随机森林三种算法进行子痫前期识别模型的研究。
结果表明,基于PPG信号多维时域特征的随机森林算法的准确率最高。

三、设计并实现了一款可以满足多场景的PE识别分析软件系统,具有从预处理到PE状态识别的完整分析功能

本论文设计并实现了一款可以满足实验室研究、医院监护、社区检查、居家监护等多应用场景的PE识别分析软件系统,
系统由数据预处理、跨平台客户端、PE识别模型训练生成与云服务器程序等四个功能模块组成,具有从数据预处理到PE病发状态的识别预测的完整分析功能。
同时,软件系统也进行了多项兼容性设计,支持对核心处理算法、识别模型等进行更新迭代。
经测试验证,该PE识别分析软件系统的各项功能均能按照设计预期正常工作,能满足多场景下的不同使用需求。
同时,该系统具有良好的拓展性,能根据实际需求,拓展至其他基于PPG的应用中,为类似研究工作的开展提供便利。

\section{展望}

本论文研究了通过无创PPG信号利用机器学习算法构建PE识别模型的可行性。虽然取得了一定的成果,但整个研究过程中的一些环节仍存在一些不完善之处,
值得开展进一步地深入和拓展:

一、增加多中心验证

增加多医疗中心的数据验证实验环节能够进一步检验本论文得到PE识别模型的各项结论的可靠性与适用性,
也能使本论文的研究工作能在临床PE实际检测中发挥作用。

二、增加实验数据

本论文的所有数据来源为79例孕妇数据,样本量偏小。更多的实验数据能够促使PE识别模型更好地收敛,
有助于训练出具有更高准确性的、更贴切实际应用的PE识别模型。

三、优化模型构建算法

本研究目前主要研究了监督学习下的决策树、K近邻与随机森林三种算法,而其他算法的适用性有待进一步进入挖掘。
其次,也可以考虑将机器学习中的非监督学习算法,深度学习中的时间序列等算法引入PE识别模型的构建中,
探索这些算法在PE识别领域的应用可能。

四、构建更好的PPG特征,特别是与PE相关的特征

在监督学习中,特征数据是构建模型的基础。因此,更好的、能够表征PE病发的PPG特征
有助于训练出性能更优的PE识别模型。在数据条件允许的条件下,也可以将本研究尚未使用的
频域特征与非线性特征引入到PE识别模型的构建中。

五、丰富数据信号种类

PE的病发会引起孕妇机体多系统器官的一系列病生理变化,理论上可以通过多种生理信号得到反应。
目前,本研究仅使用了其中的PPG信号。
探究更多种类的生理信号在PE识别领域的应用,有助于进一步提高PE识别模型的准确性与可靠性。

六、进一步完善SCD算法

本论文提出的SCD算法目前仅针对PPG波形的检测,可进一步开展研究,使其也能满足其他人体电生理信号的检测要求。
此外,也可从从初筛时的波形检测算法、复核时使用的标准、决策时使用的策略机制等方面,进一步完善SCD算法,
并简化SCD算法的使用流程,使算法的配置与拓展更为便捷。

七、进一步完善PE识别分析软件系统

严格意义上来说,目前的PE识别分析软件系统仅为一个雏形,需要进一步的研究工作才能使其真正部署至云端。
此外,软件功能有待进一步增强与完善。
为方便应用于其他基于PPG的研究中,软件系统需要进一步提高兼容性与可配置性。
