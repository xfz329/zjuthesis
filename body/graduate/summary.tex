\chapter{总结与展望}
\section{研究工作总结}
子痫前期是一种严重危害妊娠孕妇健康的多系统进展性疾病,本论文深入研究了在无创的前提下使用光电容积脉搏波信号识别孕妇是否患有子痫前期的可行性。本文全部研究内容及
工作可总结如下:

一、完成了患有子痫前期孕妇与正常妊娠孕妇光电容积脉搏波数据采集工作

本研究于2017年6月至2019年4月期间于浙江大学附属妇产科医院进行了数据采集实验,完成了子痫前期与正常妊娠孕妇光电容积脉搏波原始数据集合的构建。
在规范实验流程的基础上,通过临床标准医疗设备,总计完成了80例孕妇的数据采集(其中,1例因数据质量被剔除、患有子痫前期的孕妇数据44例、正常妊娠孕妇数据35例)。
其中,实验方案经过相关伦理委员会审批批准后进行,被试孕妇也全部知情并授权同意参与采集实验。

二、完成了光电容积脉搏波信号的预处理与特征点检测工作

本研究将所有脉搏波原始进行了统一流水处理,完成包括信号滤波、波形检测、基线漂移去除、样条插值及数据标准化等在内的预处理工作。特别地,为适配可能出现的各种存在复杂干扰的脉搏波原始信号、
提高脉搏波波形检测的抗干扰能力与检测准确率,本研究基于策略与机制分离思想,提出了一种新型脉搏波波形检测算法,即初筛-复核-投票算法。该算法有效识别脉搏波原始数据中的干扰噪声与畸变信号。
经公开脉搏波数据库及本次采集实验得到的数据验证,初筛-复核-投票算法可对存在特定干扰信号的脉搏波数据实现极高的检测准确率。

三、汇总了已有的脉搏波时域特征,设计了多种新型脉搏波时域形态特征,构建了完整的脉搏波信号时域描述特征集合

本研究对此前诸多学者对脉搏波时域形态学特征进行了系统性的总结。在此基础上,概述了基于波形的脉搏波时域描述特征的设计基础,提出了脉搏波波形特征描述向量的概念,并优化了脉搏波的面积类形态学特征的描述方法。
此外,受心电分析中心率图的启发,本研究也提出了通过脉搏波波形间的分布差异来进行描述的概念方法,使用了包括相关系数、弗朗明歇距离及包络面积差等三种特征。

同时,为探索使用光电容积脉搏波信号识别孕妇是否患有子痫前期的可行性,本研究构建了脉搏波形态学的左视类指标、中视类指标与分层类指标,从线段长度、曲线长度、斜率、弧度及面积等角度设计了
286个新型时域形态描述特征。此外,本研究也探索了不进行额外的参数设计、直接使用脉搏波原始采样值描述脉搏波形态并进行后续分析的可能。考虑到脉搏波波形时长的差异会导致不同波形的采样点不一致,
本研究使用了补零处理与重采样等两种策略来进行采样点数的调整对齐。

四、基于新型时域波形描述特征集与脉搏波原始采样点,使用多种机器学习算法构建了子痫前期的识别模型

由于本研究采集得到的脉搏波样例数目远小于通常的机器学习数据规模,因此,本研究在进行子痫前期识别模型的构建时,将数据集按照全部波形与被试分别分层抽样得到最终的训练集与测试集。
在将新型时域波形描述特征集与脉搏波原始采样点进行数据集划分、特征缩放、特征降维等处理后,在使用多种算法进行识别模型的初步构建与筛选之后,最终选取了
决策树算法、K近邻算法与随机森林算法进行后续分析。其中,随机森林是集成学习算法的一种,也可应用在原始特征集的降维处理上。

以各算法训练得到的模型在测试集上的准确率为评价模型识别能力的标准,综合来看使用随机森林构建所得的模型性能最好,K近邻模型略好于决策树模型。
此外,本研究提出的新型时域波形描述特征集及脉搏波原始采样点在子痫前期的识别能力上具有一致性。在相同的研究场景中,使用相同算法在两类时域特征集构建的模型
在测试集上的准确率数值极为接近,且均是在新型时域波形描述特征集上的构建模型性能优于在脉搏波原始采样点上构建的模型。

多种分析场景下具体的值如何比较列举。


最后,分析在新型时域波形描述特征集与脉搏波原始采样点构建的随机森林模型,可以得到所有特征/采样点对随机森林模型的贡献度。对特征贡献度分析后
可以发现两类时域特征集中具有子痫前期表征能力的特征出现的位置高度相似,且贡献度高的特征多出现在下降支中,且集中在下降支的前端与尾端。这说明上述位置的脉搏波形态可能是识别子痫前期的关键。


\section{主要创新点}
本论文的创新点主要包括以下三点:

一、基于策略与机制分离思想,提出了脉搏波波形检测的初筛-复核-投票算法

初筛-复核-投票算法改进优化了一般的波形检测流程,可有效增强对复杂信号的处理能力、提高算法的检测准确度,同时极大地保证了算法的适配性与调整的灵活性。
初筛-复核-投票算法主要依据脉搏波波峰的局部最大值原理进行波形初筛,引入了针对脉搏波波形的多标准二次复核与最终决策确认机制以减少错检发生的概率。所有
初筛得到的“可疑波形”经特定标准复核后均会得到基于该标准的一个输出判定结果。由检测算法的决策模块负责对多个标准的复核结果进行裁决判定,将“可疑波形”判断
识别为正确波形或错检干扰段。

二、设计了多种新型脉搏波时域形态特征,构建了完整的脉搏波信号时域描述特征集合

本论文研究了基于波形的脉搏波时域描述特征的设计基础,提出了脉搏波波形特征描述向量的概念。使用脉搏波波形特征描述向量构建了脉搏波左视类指标、中视类指标及分层类指标等
三大类共计286个特征,从线段长度、曲线长度、斜率、弧度及面积等方面对脉搏波时域形态进行描述。本研究也探索了不进行额外的参数设计、直接使用脉搏波原始采样值描述脉搏波形态并进行后续分析的可能。
此外,本研究也提出了通过脉搏波波形间的分布差异来进行描述的概念方法,
提出了包括相关系数、弗朗明歇距离及包络面积差等三种特征。

三、基于新型时域波形描述特征集与脉搏波原始采样点,使用多种机器学习算法构建了子痫前期的识别模型


\section{展望}
本论文研究了在无创的前提下通过从光电容积脉搏波信号构建描述特征,同时使用机器学习算法构建子痫前期识别模型的可行性。虽然取得了一定的成果,但整个研究过程中的一些环节仍存在一些不完善之处,
值得开展进一步地深入和拓展:

一、对脉搏波波形检测的初筛-复核-投票算法的改进

1. 检测算法的改进

初筛阶段现阶段使用的算法可能对某些特殊的信号检测性能欠佳,此时可视具体情景进行调整改进。

2. 复核指标的拓展

SCV算法在进行设计时无法穷极模拟各种类型的干扰信号,现阶段的复核指标无法做到对所有干扰的识别,因此,可在必要时重新设计出更能概括异常干扰与正常波形之间差异的新的复核标准。

3. 决策方法的设计

从机器学习的角度来看,脉搏波波形检测SCV算法本质上也是一个二分类的分类器,各个复核标准是用来区分异常与有效波形的输入“特征”,各复核标准的输出即为一个简单二分类器的输出,而最终的
决策输出即为综合多个二分类学习器得到最终输出。因此,在原则上可以借鉴集成学习中的相关算法及设计思想对现阶段的使用的加权投票决策方法进行改进升级。经过数据训练得到的决策模型的参数理论上会比现阶段
直接设置参数数值来决策对实际数据有更好的分类效果。

4. 异常信号的保留取舍

现阶段的SCV算法对于检测得到的异常信号是全部舍弃的,但在某些情景下,可能异常信号可能蕴含着更多信息,是相应研究的重点。因此,SCV算法可拓展成具有可配置选项,自由选择
输出特定类型的波形或更具体的满足特定复核标准的输出的波形。

5.用于其他生理信号的检测

SCV算法是针对脉搏波的检测而设计,但其检测思路也可作为其他人体电生理信号检测算法设计时的参考,可在进行其他信号类似的检测任务时进行一定的移植与改造。
