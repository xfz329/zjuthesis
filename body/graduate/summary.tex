\chapter{总结与展望}
\section{研究工作总结}
本论文深入研究了通过光电容积脉搏波(PPG)识别判断妊娠孕妇是否PE病发的可能性。
首先,设计了临床数据采集实验,通过临床标准医疗级设备,采集得到了81名孕妇的PPG信号数据,其中,1例数据因采集时长过短、信号质量太差等原因被剔除,
35例PE病发孕妇作为实验组,44例正常孕妇作为对照组。
根据所得PPG信号质量,同时结合后续分析具体需求,对PPG信号也进行了包括滤波处理、波形检测、
去除基线漂移、样条插值与数据标准化等在内的预处理工作,并为PPG的波形检测提出了一种可在多平台、多研究背
景下应用的一种基于初筛——复核——决策的新型PPG波形检测算法。
结合PPG形态特征,从时间、幅值、角度、弧度、面积、斜率等维度设计了多种PPG新型时域特征,同时,也从
描述多个PPG波形间的差异的角度出发,设计了三种描述PPG波形间差异时域特征参数。
在PPG新型时域特征的基础上,构建了PPG多维度时域特征集合;同时,将PPG信号
的原始采样序列的幅值作为特殊的描述特征,构建了PPG采样序列时域特征集合。
分析了PPG多维度时域特征集中的特征与PE的相关性,进行了降维处理。
最后,利用多种机器学习算法,在PPG多维度时域特征集与PPG采样序列时域特征集的基础上,分别构建了多种PE的识别模型,通过各模型在测试集上的泛化效果评估了各算法的性能,
也对两个PPG特征集本身表征PE的能力进行了评估。

总结论文的研究工作,主要包括以下五个方面:

一、获取构建了PE病发孕妇的PPG数据

由于直接利用PPG数据对孕妇的PE病发可能的同类研究在国内外均极为稀少,本论文自行完成了80例孕妇的PPG信号数据的采集工作,其中,35例PE病发孕妇作为实验组,44例正常孕妇作为对照组。
数据采样率为100 $Hz$,每例数据的采集时长不少于1 $min$。在数据全部获取后,使用U检验方法分析了参与临床实验的全部被试的人口统计学特征。

二、研究了PPG信号的预处理

根据PPG信号特征,选取了5阶平均滑动滤波器对PPG原始信号进行了滤波处理,通过信噪比、均方误差等指标评价了滤波效果。
借鉴计算机科学领域的策略与机制分离思想,遵循软件设计的相关准则,对PPG波形的检测过程进行了模式设计,提出一种
基于初筛—复核—决策(screening-checking-deciding,SCD)的新型算法,SCD 算法以“宽进严出”为原则,通过
初筛、复核及决策等机制检测有效PPG波形,支持直接更改初筛算法,增删复核标准,切换决策策略
等设置,保证了 SCD 算法的适用性与调整的灵活性,使高效准确地检测受到干扰或存在畸
变的PPG数据成为可能。
依据线性变换的思想,去除了PPG信号中由于呼吸等原因引入的基线漂移干扰。
依据信号的采样与重采样定理,通过插值与抽取的方法,调整原始PPG数据的采样率,保证了多种时域特征的计算工作得以进行。
对PPG波形进行了标准化处理以使其更符合后续特征计算工作的开展。

三、设计了脉搏波新型时域特征参数

对PPG时域特征的设计过程进行了方法学的归纳,提出PPG波形上的每一点均存在着与之对应的时间、幅值、角度、弧度、面积、斜率等维度参数,反之,在得到这些多维度参数的数值后,也能在PPG
波形上确定一具体的点。提出了脉搏波特征描述向量的相关概念,在PPG波形上确定多个点作为基准参照,针对某一具体指标,即可得到一组有联系的特征描述具体数值。
提出了描述PPG两个波形间差异的相关性系数、弗朗明歇距离与包络面积差等三种时域参数,并将这三种参数从对两个波形的差异性的描述推广至多个波形间的差异性描述。

四、脉搏波时域特征描述集合构建

通过左视策略、中视策略与分层策略确定了上述新型PPG时域特征的计算的基准参考点,同时,以在保留基本信息描述能力的前提下尽可能减少参考点的个数为原则,确定了三种策略对应的基准点个数,
构建得到了基于时间、幅值、角度、弧度、面积、斜率等多维度特征集合。此外,将PPG原始采样序列的幅值作为特殊的量化描述指标,构建了PPG采样序列时域特征集。提出了重采样策略、补齐策略
与截取策略等三种方法解决了PPG采样序列时域特征集中数据维度不同的问题。

五、子痫前期的识别模型的构建与评估

由于PPG是一种平稳随机信号,短时间内采样得到的数据波形理应具有较高的相似度。据此提出了通过单个PPG波形与被试所有波形进行PE识别分析的两个研究角度。在此基础上,按照机器学习
的处理流程对上述两个PPG时域特征集进行了数据集划分、数据缩放及数据降维等处理。在此基础上,综合使用多种机器学习算法,训练构建子痫前期的识别模型,同时在预留的测试集上评估了各算法模型的
性能,同时也对两个时域特征集中与PE识别相关的特征进行了分析,并对两个时域特征集本身的性能进行了对比。

\section{主要创新点}
本论文的创新点主要包括以下三点:

一、基于策略与机制分离思想,提出了脉搏波波形检测的初筛-复核-投票算法

本研究对 PPG 波形的检测过程进行了模式设计,提出了一种新型 SCD 算法,算
法由初筛、复核与决策等三部分组成。在初筛时,通过搜索窗的方法确定波形;在复核时,
通过功率、标准差、波峰相对位置与基线漂移程度等 PPG 形态学特征,区分有效波形与异
常干扰;在决策时,通过加权投票的策略确定 PPG 波形是否为有效波形。SCD 算法可根据
使用场景进行二次开发,支持更改初筛算法、增删复核标准与切换决策策略等设置。经验证,
该算法各模块设计合理,与其他算法对比结果表明,对 PPG 波形的识别准确率高、抗干扰
能力强。SCD 算法为多平台、多研究应用下的 PPG 分析检测的不同场景提供了一定的参考。

二、设计了多种新型脉搏波时域形态特征,构建了完整的脉搏波信号时域描述特征集合



三、基于新型时域波形描述特征集与脉搏波原始采样点,使用多种机器学习算法构建了子痫前期的识别模型


\section{展望}

