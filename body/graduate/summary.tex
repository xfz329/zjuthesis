\chapter{总结与展望}
\section{研究工作总结}
本论文深入研究了借助光电容积脉搏波信号识别孕妇是否子痫前期的可行性。首先,完成了患有子痫前期与正常妊娠孕妇光电容积脉搏波原始数据集合的构建,总计完成了80例孕妇的数据采集(1例
因数据质量被剔除、患有子痫前期的孕妇数据44例、正常妊娠孕妇数据35例)。

\section{主要创新点}
1. 基于策略与机制分离思想,提出了一种新型脉搏波波形检测算法,即初筛-复核-投票(Screen-Check-Vote)算法,可对较为复杂的原始脉搏波波形进行检测,有效提高波形检测算法的抗干扰能力。

2. 提出了多种新型脉搏波时域描述特征,构建了三类脉搏波时域描述特征(集合),可作为通用脉搏波描述集合从时域角度对脉搏波波形进行定量描述。

3. 基于上述脉搏波时域描述特征集合,综合使用机器学习的多种算法模型,从中筛选出于能够有效表征子痫前期的时域特征子集。在此基础上建立了多种子痫前期的识别模型,其中,基于随机森林的
算法模型的综合识别性能最为优秀。

\section{展望}

针对算法的改进方向

针对具体环境下的PPG信号检测,SCV还可以从以下方面做出改进与拓展:

1. 检测算法的改进

初筛阶段现阶段使用的算法可能对某些特殊的信号检测性能欠佳(如\autoref{fig:ucibp_abnormal}),此时可视具体情景进行调整改进。

2. 复核指标的拓展

SCV算法在进行设计时无法穷极模拟各种类型的干扰信号,现阶段的复核指标无法做到对所有干扰的识别,因此,可在必要时重新设计出更能概括异常干扰与正常波形之间差异的新的复核标准。

3. 决策方法的设计

从机器学习的角度来看,脉搏波波形检测SCV算法本质上也是一个二分类的分类器,各个复核标准是用来区分异常与有效波形的输入“特征”,各复核标准的输出即为一个简单二分类器的输出,而最终的
决策输出即为综合多个二分类学习器得到最终输出。因此,在原则上可以借鉴集成学习中的相关算法及设计思想对现阶段的使用的加权投票决策方法进行改进升级。经过数据训练得到的决策模型的参数理论上会比现阶段
直接设置参数数值来决策对实际数据有更好的分类效果。

4. 异常信号的保留取舍

现阶段的SCV算法对于检测得到的异常信号是全部舍弃的,但在某些情景下,可能异常信号可能蕴含着更多信息,是相应研究的重点。因此,SCV算法可拓展成具有可配置选项,自由选择
输出特定类型的波形或更具体的满足特定复核标准的输出的波形。

5.用于其他生理信号的检测

SCV算法是针对脉搏波的检测而设计,但其检测思路也可作为其他人体电生理信号检测算法设计时的参考,可在进行其他信号类似的检测任务时进行一定的移植与改造。