\chapter{基于多维度特征的子痫前期识别模型的构建与分析}
\section{引言}
本章在第四章构建的PPG多维度时域特征集与PPG采样值时域特征集的基础上利用机器学习的多种算法完成PE识别模型的构建与分析工作。
本章也介绍了在上述过程中涉及的典型机器学习算法的原理与评估模型优劣的性能指标。针对在第四章提出的探寻被试单个PPG波形、所有PPG波形与PE之间关系的两个具体研究目标,本章也基于上述两个PPG时域特征集
分别完成了机器学习模型的筛选、模型超参数优化、特征贡献度分析与降维等研究工作。
最后,本章也对比了两个研究目标下基于两个PPG时域特征集得到机器学习的结果的异同点,并分析了可能的原因。

\section{机器学习算法原理与模型评价标准}
一般认为,机器学习是一门致力于研究通过计算的手段、利用已有的经验来改善系统自身性能的学科和艺术\cite{Zhou2016,Aurélien2018}。其中,Tom Mitchell对机器学习给出了一种最为经典的形式化定义:
计算机程序利用经验$E$学习任务$T$,其性能是$P$,如果针对任务$T$的性能$P$随着经验$E$不断增长,那么我们就说关于$T$与$P$,该程序对$E$进行了学习\cite{mitchell1997,Zhou2016}。
对计算机程序而言,$E$通常以数据的形式存在,因此机器学习也可以看成从相关数据中产生模型的算法过程,不显式编程是机器学习最典型的特征。
本文末附录D给出了部分机器学习领域常见的术语及其解释。

尽管机器学习的相关概念早在上世纪五十年代就已经被提出,但直到进入新世纪后,机器学习才真正迎来井喷式发展的黄金期。在过去二十年中,由于半导体电子计算机行业的充分发展,人类收集、传输、处理数据的能力取得了长足的进步,
人类各种社会活动中出现的海量数据具备了能够被挖掘、分析的硬件基础与需要被分析并加以利用的客观需求。在此背景下,机器学习受到了学者们的广泛关注并进入了蓬勃发展阶段,不论是理论基础方面亦或是应用研究方面都
得到了巨大的发展,取得了重大突破。目前,机器学习技术已经被成功应用在模式识别、数据挖掘、自然语言处理、语言识别、图像识别、芯片设计、信息检索及生物信息学等学科领域,
尤其是为交叉学科的发展研究提供了新的技术支撑与突破点\cite{Zhou2016,Aurélien2018,Li2017}。   

本小节将从本研究所使用到的机器学习的算法原理及评价使用机器算法训练得到的模型性能时的评价标准进行介绍。
\subsection{决策树与随机森林的算法原理}
监督学习是机器学习的重要研究方向,主要用于推断观察数据(也称为输入数据)与目标变量(因变量或标签)之间的潜在关系。
基于监督学习算法的经过良好训练的函数模型可以准确预测隐藏在不熟悉或未观察到的数据实例中的隐藏现象的类标签。
本研究的所使用的机器学习算法均属于此范畴。下面以决策树算法与随机森林算法为代表进行算法原理介绍。

一、决策树

决策树(Decision Tree,DT)是数据挖掘的经典算法之一,是一种类似流程图的树结构,可以用于连续数值型变量的回归预测及离散型数值型变量的分类问题\cite{Li2017,Liu2018}。
决策树算法的最显著优点是简单直观,易于可视化、可读性强。

1、决策树的结构与原理

分类决策树模型是对实例进行分类的树形结构的描述,如\autoref{fig:dt}所示。一般而言,决策树由结点和有向边组成,而结点又可分为内部结点与叶节点。其中,前者表示一个特征或属性,后者对应决策结果,一般是一个类\cite{Li2017,Zhou2016}。
决策树可以看成一个if-else规则的集合,决策树的根节点到叶节点的每条路径分别对应着一条规则:路径上的内部结点集合构成了规则的判断条件,而叶节点所属的具体类则对应着该规则的结论。
决策树的学习目的就是从训练数据集中集中归纳出一组分类规则,产生一颗与训练数据的矛盾小的、泛化能力强的逻辑判断树。
\begin{figure}[htbp]
      \centering
      \includegraphics[width=.6\linewidth]{models/dt.png}
      \caption{\label{fig:dt}决策树模型示意}
\end{figure}

2、决策树的特征选择

为生成一颗分类能力强的决策树,一种可行的策略是只选取对分类效果有提升的特征参与构建。基尼指数(Gini index)与信息增益(information gain)是两种最常用的用于筛选最佳特征的指标。

基尼指数,也称为基尼不纯度,其定义为
\begin{equation}
      \label{equ:gini}
      G_i = 1 - \sum_{k=1}^n{p_{i,k}}^2
\end{equation}
其中,$p_{i,k}$是第$i$个节点上,类别为$k$的训练实例占比。基尼指数数值越大,样本集合的不确定性也越大。特别地,若样本集合$D$根据特征$A$是否取某一可能值$a$而被分割成$D_1$和$D_2$两部分,即
\begin{equation}
      \label{equ:daset}
      \left \{
      \begin{aligned}
            D_1 &= \{ (x,y) \in D \mid A(x) = a\} \\
            D_2 &= D - D_1
      \end{aligned}
      \right.
      \end{equation}
那么,在特征$A$的条件下,样本集合的基尼指数可以表示为
\begin{equation}
      \label{equ:ginia}
      G(D,A) = \frac{|D_1|}{|D|}G(D_1) + \frac{|D_2|}{|D|}G(D_2)
\end{equation}
其中,$|D|$表示样本集合$D$的样本数量。\autoref{equ:ginia}描述了$D$经特征$A=a$分割后的不确定性,此时,筛选最佳特征的过程可以转换为寻找使\autoref{equ:ginia}取值最小的特征$A$的具体数值。

另一方面,信息增益是在引入了信息论中的信息熵(information entropy)概念进行定义的并计算使用的,其作用与具体使用方法与基尼指数类似,这里不再进行赘述\cite{Zhou2016,Li2017}。

3、决策树的生成

决策树的生成构建过程就是递归地选择最优特征,并根据最优特征对训练数据进行分割,使该分割对各个新的子数据集有最优分类效果的过程。其中,最经典生成算法包括ID3(Iterative Dichotomiser 3,第三代迭代二分器)决策树学习算法、
C4.5(Classifier 4.5,第4.5代分类器)决策树学习算法及CART(classification and regression tree,分类与回归树)决策树学习算法等三种\cite{quinlan1986,quinlan1993,breiman1984}。
在这三种算法中,只有CART算法生成的决策树既可以执行分类任务也可以执行回归任务,故CART算法的应用也最为广泛。

CART算法采用最小基尼指数来选择特征,生成的决策树为二叉树。其工作的基本原理如\autoref{alg:cart}所示。
\begin{breakablealgorithm}
      \caption[CART生成算法]{CART递归生成算法\cite{Li2017}}
      \label{alg:cart}
      \begin{algorithmic}[1] %每行显示行号
            \Require 训练数据集$D$。
            \Ensure CART决策树。
            \State 建立一颗空树$CART$,设该树的根结点为$root$。
            \Function{GenerateCart}{$CART,D_c,root$}
                  \State $D_c$为当前结点的训练数据集,计算现有特征对该数据集的基尼指数。对每一个特征$A$,对其可能取的每个值$a$,根据样本点对$A=a$的测试为“是”或“否”将$D_c$分割成$D_l$与$D_r$两部分,使用\autoref{equ:ginia}计算$A=a$时的基尼指数。
                  \State 在所有可能的特征$A$以及它们所有可能的切分点$a$中,选择基尼指数最小的特征及其对应的切分点作为最有特征与最优切分点。
                  \State 依最优特征与最优切分点,从现结点生成两个子节点$left$与$right$,将训练数据集依特征分配到两个子节点中去,即$D_l$与$D_r$。更新当前$CART$。
                  \If {结点中的样本个数小于预定阈值 \textbf{or} 样本集的基尼指数小于预定阈值 \textbf{or} 没有更多特征}
                  \State \Return{$CART$}
                  \Else    
                  \State \Call{GenerateCart}{$CART,D_l,left$}
                  \State \Call{GenerateCart}{$CART,D_r,right$}
                  \EndIf
            \EndFunction
      \end{algorithmic}
\end{breakablealgorithm}

4、决策树的剪枝

为防止决策树出现过拟合的情况,通常都会对过于“茂密”的树进行剪枝(pruning)处理。决策树的剪枝方法可以分为预剪枝与后剪枝两大类。
前者的工作原理是在决策树的生长阶段就对其进行一定的限制,包括限制树最大生长深度、限制决策树生成的最多叶节点数量等。
后剪枝则是在决策树得到完全生长之后进行,其处理算法也更复杂、训练时间等开销也更大\cite{Zhou2016,Liu2018}。

二、随机森林

随机森林是一种在Bagging算法基础上发展起来的集成学习算法,最早由Leo Breiman于2001年提出\cite{breiman2001}。

1、Bagging

为获得泛化性能强的不同的基学习器,一种可行的策略是对所有学习器使用同一种训练算法,但是在训练集的不同随机子集上进行训练,使这些基学习器的能够有一定的差异。
\autoref{fig:bp}所示,在上述过程的随机子集的建立过程中,若对原始数据样本的采样后放回,这种方法即为bagging;与之对应的,采样后不放回的方法称为pasting\cite{Aurélien2018,Zhou2016}。
由于bagging方法可以获得的随机子集数量要远远高于pasting方法,bagging方法应用得也更加广泛。
\begin{figure}[htbp]
      \centering
      \includegraphics[width=.6\linewidth]{models/bp}
      \caption[Bagging与Pasting示意]{\label{fig:bp}Bagging与Pasting示意\cite{Aurélien2018}}
\end{figure}

当使用有采样后即放回的方法从包含$m$个原始样本的训练集$D$抽取出$m$个新的样本构成此次训练数据集$D_{bs}$时,显然,$D$有部分数据样本在$D_{bs}$重复出现,部分数据则从未被抽样过。
易知样本在$m$次均未被抽中的概率为$(1-\frac{1}{m})^m$,当$m \to \infty$时,
\begin{equation}
      \label{equ:me}
      \lim_{m \to \infty}{(1-\frac{1}{m})}^m = \frac{1}{e} \approx 0.368
\end{equation}
\autoref{equ:me}说明约有36.8\%的原始样本未出现在采样集$D_{bs}$中,这部分数据可以作为当前训练算法的测试集。

按上述思想,可从原始样本训练集$D$采样得到$T$个包含$m$个训练样本的采样集,基于这些采样集,使用特定的机器学习算法可以训练得到$T$个基学习器,结合这些基学习器的输出即为Bagging算法
的基本流程,如\autoref{alg:bagging}所示。投票法和平均法是Bagging在进行基学习器的输出时常采用的策略。
\begin{breakablealgorithm}
      \caption[Bagging算法]{Bagging算法\cite{Zhou2016}}
      \label{alg:bagging}
      \begin{algorithmic}[1] %每行显示行号
            \Require 训练集$D=\{(x_1,y_1),(x_2,y_2),\dots,(x_m,y_m)\}$;基学习算法$\xi$;训练轮数$T$。
            \Ensure $H(x)=\arg \max \limits_{y \in Y} \sum_{t=1}^T \mathbb{I}(h_t(x)=y)$,其中$\mathbb{I}(.)$为指示函数,在$.$为真或假时函数值分别为1或0。
            \For {$t=1,2,\dots,T$}
                  \State 从原始训练集$D$自助采样得到此次的样本分布$D_{bs}$
                  \State $h_t=\xi (D,D_{bs})$
            \EndFor
      \end{algorithmic}
\end{breakablealgorithm}

2、随机森林

如\autoref{fig:rf}所示,随机森林算法是由多棵\autoref{fig:dt}所示的决策树构成,这些决策树一般都是经过充分生长的、未经剪枝处理的CART决策树。
而“随机”一词有两重含义,首先是同Bagging算法一样,每棵决策树在训练时使用的训练样本是随机抽取的;其次,与\autoref{alg:cart}所示的一般CART决策树生成算法不同,随机森林中的CART
决策树在生长时并不是在当前结点的$d$个属性集合$A$中选取最优特征及其最优切分点,而是先从$A$中随机生成一个包含$k$个属性子集的$A_{bs}$,随后再从$A_{bs}$中选择最优属性进行划分\cite{Zhou2016,Liu2018,breiman2001}。其中,$k$的推荐取值为
$\lfloor \log_2m + 1 \rfloor$\cite{breiman2001}。

\begin{figure}[htbp]
      \centering
      \includegraphics[width=.6\linewidth]{models/rf}
      \caption{\label{fig:rf}随机森林示意}
\end{figure}

对回归问题而言,随机森林算法的输出是所有决策树的输出的均值;而对分类任务而言,算法输出是所有决策树输出投票的结果。随机森林在树的生长过程中的两次随机使决策树具有更大的多样性,相当于用更高的偏差换取更低的方差。因此,
最终的集成结果有着出色的泛化性能,有效避免了单决策树可能导致的过拟合问题。随机森林算法运行速度快、准确率高且泛化性能优秀,被誉为“代表集成学习技术水平的方法”\cite{Zhou2016,Liu2018}。

此外,随机森林算法往往也会在特征选择的过程中得到应用\cite{Aurélien2018}。重新考察\autoref{fig:dt}中的决策树可以发现,越靠近根结点位置的特征对决策过程的重要程度也越高,而不重要的特征多出现在靠近结点的位置、甚至不出现在决策树中。
因此,特征的重要性(或贡献度)可以通过计算其在随机森林众多决策树的平均深度来进行量化衡量。

\subsection{机器学习模型的评价标准}
为量化描述机器学习算法训练得到的模型的性能表现,学者们提出了多种衡量指标。

一、混淆矩阵及其衍生指标

混淆矩阵(confusion matrix)是评估分类器分类效果优劣的常用工具\cite{Zhou2016,Aurélien2018}。其总体思路就是分别统计A类别实例被划分成B类别实例的数目。理论上混淆矩阵的行列没有上限,而在实际应用中,二分类任务的混淆矩阵是最常见的。
此时,将样例依据其真实所属类别与分类器预测类别进行组合可得到四种结果:真阳性(true positive,TP)、假阳性(false positive,FP)、真阴性(true negative,TN)及假阴性(false negative,TN),如\autoref{tab:cm}所示。此时显然有
$TP+FP+TN+FN=\text{样例总数}$。
\begin{center}
      \zihao{-5}
      \begin{longtable}{m{3cm}<{\centering}m{3cm}<{\centering}m{3cm}<{\centering}}
      \caption{二分类任务的混淆矩阵}\\
      \label{tab:cm}\\
      \topline
      \colorhead  & \multicolumn{2}{c}{\textbf{预测结果}} \\
      \colorhead \multirow{-2}{*}{\textbf{真实情况}}  & 阳性(1) & 阴性(0)\\
      \midline
      \endfirsthead
      \caption[]{(续)}\\
      \topline
      \colorhead  & \multicolumn{2}{c}{\textbf{预测结果}} \\
      \colorhead \multirow{-2}{*}{\textbf{真实情况}}  & 阳性(1) & 阴性(0)\\
      \midline
      \endhead 
      \hline
      \endfoot
      \bottomline
      \endlastfoot
      \colorrowa 阳性(1) & 真阳性(TP) & 假阴性(FN) \\
      \colorrowc 阴性(0) & 假阳性(FP) & 真阴性(TN) \\
\end{longtable}
\end{center}

为量化分类器的具体性能,人们在混淆矩阵的基础上衍生定义了一系列数字指标,包括查全率(recall)、查准率(precison)、准确率(accuracy)及特异性(specificity)等,如\autoref{equ:measures}所示。
\begin{equation}
      \label{equ:measures}
      \left \{
      \begin{aligned}
            Recall      &=\frac{TP}{TP+FN}         \\
            Precison    &=\frac{TP}{TP+FP}          \\
            Accuracy    &=\frac{TP+TN}{TP+FP+TN+FN} \\
            Specificity &=\frac{TN}{TN+FP}       \\
      \end{aligned}
      \right.
\end{equation}
其中,查全率亦称召回率、灵敏性(sensitivity)或真阳性率(true positive rate,TPR),查准率亦称精准率,特异性亦称真阴性率。查全率与查准率是应用的最广泛的两个指标\cite{Zhou2016,Aurélien2018}。
一般而言,查全率与查准率是对相互矛盾的度量指标,一个指标性能的提高意味着另一个指标性能的下降。通常只有在简单分类任务中,
才能同时获得较高的查准率与查全率。这称为精度-召回率权衡。为评估查全率与查准率均不相等的分类器性能,人们进一步定义了$F_1\text{分数}$,如\autoref{equ:f1}所示。
\begin{equation}
      \label{equ:f1}
      F_1=\frac{2}{\frac{1}{Precison}+\frac{1}{Recall}}=\frac{2\cdot Precison\cdot Recall}{Precison+Recall}=\frac{TP}{TP+\frac{FN+FP}{2}}
\end{equation}
$F_1\text{分数}$是召回率与精准率的谐波均值。召回率与精准率相近的分类器易获得更高的$F_1\text{分数}$。

在评估分类器性能时需要根据场景,从\autoref{equ:measures}与\autoref{equ:f1}中灵活选取恰当的评价指标。

二、ROC曲线、AUC与约登指数

受试者工作特征(Receiver Operating Characteristic,ROC)曲线是另一种常用于二分类问题的分析工具。ROC绘制的是真阳性率和假阳性率(false positive rate,FPR)之间的变化关系,其中
\begin{equation}
      \label{equ:fpr}
      FPR=\frac{TN}{TN+FP}=1-Specificity
\end{equation}
因此,ROC曲线也被称为灵敏度与1-特异性曲线。绘制曲线时,以分类器的预测结果对样例进行升序排列,依次将样本作为阳性进行预测,计算对应的TPR与FPR后,可得一坐标点$({FPR}_i,{TPR}_i)$,最后将所有坐标点连线即可,如\autoref{fig:roc}所示。
其中,虚线表示纯随机分类器的ROC曲线,理想性能的分类器应无限逼近左上角,即坐标点$(0,1)$。
\begin{figure}[htbp]
      \centering
      \includegraphics[width=.6\linewidth]{data_plan/roc}
      \caption[ROC曲线与AUC数值]{\label{fig:roc}ROC曲线与AUC数值。各分类器的AUC具体数值参见图例。}
\end{figure}

在衡量多个分类器性能优劣时,常将分类器对应的ROC曲线下面积作为判据,即为AUC(Area Under Curve)。纯随机分类器ROC的AUC数值为0.5,而理想分类器ROC的AUC数值为1,如\autoref{fig:roc}所示。

此外,约登指数(Youden Index)也是用来评价分类器效果的一个指标。若在评估分类器性能时,给予将分类器假阴性和假阳性以相同权重,即可应用约登指数
\begin{equation}
      \label{equ:yi}
      \begin{aligned}
            YI&=Sensitivity-(1-Specificity)\\
            &=Sensitivity+Specificity-1
      \end{aligned}
\end{equation}
一般认为,当YI取值最大时,此时对应的分类阈值为最佳阈值\cite{cwl}。

\section{基于PPG多维度时域特征集的机器学习分析}
本小节基于PPG多维度时域特征集,分别研究了被试单个PPG波形、所有PPG波形与PE之间关系。而这两类研究本质上都是二分类任务。
\subsection{单个PPG波形与PE之间关系}
在被试的单个PPG波形具备了能够表征被试PE患病状态的假设前提下,本部分进行了机器学习算法初筛、算法超参数优化及特征降维等研究工作。

一、算法初筛

在按PPG波形划分数据集的基础上,本研究使用了多种监督学习的算法进行了PE识别模型的前期研究,具体包括随机梯度下降、决策树、K近邻、高斯朴素贝叶斯、逻辑回归、线性支持向量机、核支持向量机、C-支持向量机及多层感知机等九种分类算法。
前期研究未对上述算法的超参数进行调整,全部使用默认数值或推荐数值\cite{scikit-learn}。
通过这些算法训练得到的机器学习模型(以下简称为模型)在训练集与测试集上的结果如\autoref{tab:model_screen}所示,其中训练集的统计结果是经5层交叉验证处理后得到的。
\begin{center}
      \zihao{-5}
      \begin{longtable}{m{1.5cm}<{\centering}m{1cm}<{\centering}m{1cm}<{\centering}m{1cm}<{\centering}m{1cm}<{\centering}m{1cm}<{\centering}m{1cm}<{\centering}m{1cm}<{\centering}m{1cm}<{\centering}m{1cm}<{\centering}m{1cm}<{\centering}}
            \caption{初筛结果}\\
            \label{tab:model_screen}\\
            \topline
            \colorhead &  & \multicolumn{5}{c}{\textbf{训练集(5层交叉验证)}} & \multicolumn{4}{c}{\textbf{验证集}}                                                                                                                                                                                                      \\
            \colorhead \multirow{-2}{*}{\textbf{模型类型}} & \multirow{-2}{*}{\textbf{训练时间}} &  \textbf{精确率} &  \textbf{召回率} &  \textbf{F1值} &  \textbf{准确率} &  \textbf{AUC}  &  \textbf{精确率} &  \textbf{召回率} &  \textbf{F1值} &  \textbf{准确率}    \\
            \midline
            \endfirsthead
            \caption[]{(续)}\\
            \midline
            \colorhead &  & \multicolumn{5}{c}{\textbf{训练集(5层交叉验证)}} & \multicolumn{4}{c}{\textbf{验证集}}                                                                                                                                                                                                      \\
            \colorhead \multirow{-2}{*}{\textbf{模型类型}} & \multirow{-2}{*}{\textbf{训练时间}} &  \textbf{精确率} &  \textbf{召回率} &  \textbf{F1值} &  \textbf{准确率} &  \textbf{AUC}  &  \textbf{精确率} &  \textbf{召回率} &  \textbf{F1值} &  \textbf{准确率}    \\
            \midline
            \endhead 
            \midline
            \endfoot
            \bottomline
            \endlastfoot
            \colorrowa 随机梯度下降      &   6.17 s   & 87.2\% & 68.9\% & 77.0\% & 75.5\% & 0.876 & 74.8\% & 97.7\% & 84.7\% & 79.0\% \\
            \colorrowc 决策树            &   5.24 s  & 91.1\% & 83.0\% & 86.9\% & 85.1\% & 0.907 & 93.6\% & 81.4\% & 87.1\% & 85.6\% \\
            \colorrowa K近邻算法      &   3.08 s    & 94.7\% & 93.7\% & 94.2\% & 93.1\% & 0.974  & 95.4\% & 93.3\% & 94.3\% & 93.3\% \\
            \colorrowc 高斯朴素贝叶斯算法      &   1.22 s  & 87.9\% & 63.9\% & 74.0\% & 73.2\% & 0.838  & 90.1\% & 65.0\% & 75.5\% & 74.9\% \\
            \colorrowa 逻辑回归算法      &   203.0 s  & 90.2\% & 90.7\% & 90.4\% & 88.6\% & 0.950 & 93.3\% & 93.0\% & 93.1\% & 91.8\% \\
            \colorrowc 线性支持向量机      &   47.22 s  & 79.7\% & 93.2\% & 86.0\% & 81.9\% & 0.917 & 95.9\% & 81.9\% & 88.3\% & 87.1\% \\
            \colorrowa 核支持向量机      &   60.28 s  & 82.5\% & 90.3\% & 86.2\% & 82.8\% & 0.916 & 84.9\% & 91.4\% & 88.0\% & 85.2\% \\
            \colorrowc C-支持向量机      &   42.47 s  & 84.3\% & 90.5\% & 87.3\% & 84.3\% & 0.929 & 87.1\% & 90.8\% & 88.9\% & 86.5\% \\
            \colorrowa 多层感知机      &   26.8 s  & 83.5\% & 75.8\% & 79.5\% & 76.6\% & 0.905 & 89.3\% & 91.1\% & 90.2\% & 88.2\% \\
      \end{longtable}
\end{center}

从\autoref{tab:model_screen}中可以得到以下初步结论:

1、从模型的训练时间方面来看,高斯朴素贝叶斯算法训练所需时间最短,仅需1.22s,而多层感知机、支持向量机模型所需时间较长、逻辑回归算法训练时间最长为203.0s。这些数值也与各算法的
复杂度对应,符合预期。

2、从模型在测试集上AUC方面来看,只有经高斯朴素贝叶斯算法得到的模型AUC数值在0.850以下,而由K近邻算法得到的AUC最高可达为0.974。

3、从各模型在训练集上的混淆矩阵来看,由K近邻算法与逻辑回归算法得到的模型在精度-召回率权衡上表现最好,精确率、召回率及F1数值均在90.0\%以上。由决策树算法与三种支持向量机算法得到的模型在精确率与召回率可以达到90.0\%+80.0\%
(或80.0\%+90.0\%)以上,这些模型的F1值也均在86.0\%以上。而由随机梯度算法、高斯朴素贝叶斯算法与多层感知机算法得到的模型在这些数值上表现较差。

4、从各模型在验证集上的泛化能力来看,由随机梯度算法与高斯朴素贝叶斯算法得到的模型性能最差,出现精确率或召回率数值小于75\%的情况,而由另外七种算法得到的模型泛化能力较强。其中,由决策树算法、逻辑回归算法、三种支持向量机算法及多层感知机算法得到的模型性能接近,精确率与召回率可以达到90.0\%+80.0\%
(或80.0\%+90.0\%),F1值也均在87.0\%以上。而由K近邻算法与逻辑回归算法得到的模型表现最为优秀,精确率、召回率与F1值三者数值更是均在93.0\%以上。

综上,上述数值结果初步说明了本研究提出的PPG多维度时域特征集包含了一定数量的与PE相关性较强的特征参数,在此基础上可以训练得到有一定泛化能力的识别PE的机器学习模型。
此外,对初筛阶段探索使用的九种机器学习算法而言,通过K近邻与逻辑回归算法构建的子痫前期识别模型的性能最为出色,而由随机梯度算法与高斯朴素贝叶斯算法得到的模型性能最差。

二、初筛模型的超参数优化

超参数(Hyperparameter)是指机器学习模型在开始学习过程之前人工设置值的调优参数\cite{scikit-learn,Aurélien2018},而超参数的调整与优化通常会使模型的性能得到一定的提升。
网格搜索是寻找最优超参数最为常用的策略之一,其基本原理为预先设置好该算法的所有待调超参数的可选值集合,通过排列组合的方式得到该算法的多个实例后,
使用这些算法实例分别在训练集上训练模型,在测试集上的性能最优的模型使用的超参数组合即为全局最优超参数\cite{Aurélien2018}。
而判断性能最优可使用多种参数指标,各模型在训练集上的AUC数值是其中最为常见的一种。

由于初筛时\autoref{tab:model_screen}中的数值结果是在各算法使用默认超参数的条件下得到的,因此,本研究也对这些算法的最优超参数进行了研究。
在综合考虑模型的训练时间及初筛性能表现等因素,本研究选取了初筛阶段有代表性的随机梯度下降、高斯朴素贝叶斯、决策树与K近邻等四种算法进行了超参数优化。
其中,由前两类算法训练得到的模型在初筛阶段性能较差,超参数优化是为了探索显著提升其性能的可能;由后两类算法训练得到的模型在初筛阶段表现较好,超参数优化是为了探索其最佳性能。

\begin{center}
      \zihao{-5}
      \begin{longtable}{m{4cm}<{\centering}m{6.5cm}<{\centering}m{4cm}<{\centering}}
            \caption{初筛模型的超参数优化}\\
            \label{tab:super_para}\\
            \topline
            \colorhead \textbf{模型类型} & \textbf{超参数组合值域}     &     \textbf{最优超参数}\\
            \midline
            \endfirsthead
            \caption[]{(续)}\\ 
            \midline
            \colorhead \textbf{模型类型} & \textbf{超参数组合值域}     &     \textbf{最优超参数}\\
            \endhead 
            \midline
            \endfoot
            \bottomline
            \endlastfoot
            \colorrowa 随机梯度下降    & \begin{tabular}[c]{@{}l@{}}loss:{[}\textbf{hinge}, log\_loss,   \\ log, modified\_huber, \\ squared\_hinge, perceptron, \\ squared\_error,  huber,\\  epsilon\_insensitive, \\ squared\_epsilon\_insensitive{]},\\    penalty:{[}\textbf{l2},l1,elasticnet{]},\\   alpha:{[}0.001,\textbf{0.0001},0.00001{]}\end{tabular} &  \begin{tabular}[c]{@{}l@{}}alpha=0.001, \\ loss=squared\_hinge,   \\ penalty=elasticnet\end{tabular}     \\
            \colorrowc 高斯朴素贝叶斯算法   & var\_smoothing:{[}1e-5,1e-7,\textbf{1e-9},1e-11{]}             & var\_smoothing=1e-7                     \\
            \colorrowa 决策树          & \begin{tabular}[c]{@{}l@{}}criterion:{[}\textbf{gini},entropy,log\_loss{]},\\  splitter:{[}\textbf{best},random{]},\\     max\_depth:{[}\textbf{3},4,5{]},\\  max\_features:{[}sqrt,log2,\textbf{None}{]}\end{tabular}        & \begin{tabular}[c]{@{}l@{}}criterion=entropy,\\ splitter=random,\\ max\_depth=5, \\ max\_features=None\end{tabular}         \\
            \colorrowc K近邻算法           & \begin{tabular}[c]{@{}l@{}}n\_neighbors:{[}3,\textbf{5},7,9{]},\\    weights:{[}\textbf{uniform},distance{]}\end{tabular}       & \begin{tabular}[c]{@{}l@{}}n\_neighbors=9,\\  weights=distance\end{tabular}             \\
      \end{longtable}
\end{center}

\begin{figure}[htbp]
      \centering
      \subfigure[\label{fig:cmdl1}各模型在超参数优化前后的性能对比]{
      \includegraphics[width=6.5cm]{results/contrast_model1}
      }
      \quad
      \subfigure[\label{fig:cmdl2}各模型在超参数优化前后的性能对比]{
      \includegraphics[width=6.5cm]{results/contrast_model2}
      }
      \caption{\label{fig:contrast_model}各模型在超参数优化前后的性能对比}
\end{figure}

\autoref{tab:super_para}给出了上述四种算法给出了在进行网格搜索时各超参数的值域组合与最优超参数数值。其中,超参数组合值域一栏加粗显示了各模型的超参数的默认数值,而最优超参数是以各模型在测试集上的AUC取值为衡量标准得到的。
与此同时,\autoref{fig:contrast_model}也对比展示了在超参数优化前后,各模型在训练集上的AUC数值与准确率变化情况。这些统计数值也是在测试集上进行5层交叉验证后求均值得到的。

由于测试集上AUC是作为衡量标准使用,在超参数调优后,\autoref{fig:contrast_model}中的四种模型的AUC数值按预期有一定的数值提升。但从测试集上的准确率来看,只有决策树算法与K近邻算法
延续了上步中较为优秀的性能表现,模型在测试集上的准确率有一定的提升。这也进一步佐证了模型初筛时得到的各项结论。

三、随机森林算法与特征降维

前文中已经介绍过,集成学习通常会较仅使用单一算法的机器学习具有更好的性能表现。本研究以集成学习的常见的随机森林算法为代表在PPG多维度时域特征集
上进行了PE识别模型训练。此外,本研究也使用随机森林算法衡量了PPG多维度时域特征集各属性的贡献度,并在此基础上进行了模型训练后的降维处理。

本研究利用默认超参数在PPG多维度时域特征集上也训练得到了随机森林模型\cite{scikit-learn}。结果表现,由随机森林算法训练得到的模型在训练集与测试集上均有着优秀的表现,
其在训练集上的AUC面积为0.990,在训练集与测试集上的准确率更是分别达到了95.2\%与97.0\%。
这些数值均高于上步中的任一单算法模型的结果,充分体现了集成学习的优势。

\begin{figure}[htbp]
      \centering
      \includegraphics[width=\linewidth]{results/rfdt}
      \caption{\label{fig:dt_clf}由随机森林算法构建得到的决策树示意}
\end{figure}

PPG多维度时域特征集共有286个时域特征,利用随机森林算法构成多棵决策树形成森林之后,可以得到每棵决策树中特征的相对深度,如\autoref{tab:rf_dr_1}所示。
而进一步计算各特征在森林中的平均深度即可得到各特征对最终PE识别模型的贡献度的数值。
\autoref{tab:rf_dr_1}列举了对随机森林累计贡献度达51.9\%的前36个特征。其中,基于脉搏波上升支的特征为红色背景填充,基于脉搏波下降支的特征为蓝色背景填充;特征名后的数字下标代表了
特征所对应的脉搏波位置,越靠近峰值的位置下标值越大。其中,贡献度最高的特征为CVALF\_9,贡献度为4.6\%。
若以特征CVALF\_9为基准,可以得到所有特征对随机森林模型的相对贡献度如\autoref{fig:rf_importance_pulse}所示。

\begin{center}
      \zihao{-5}
      \begin{longtable}{m{1.5cm}<{\centering}m{2cm}<{\centering}m{2cm}<{\centering}m{2cm}<{\centering}m{2cm}<{\centering}m{2cm}<{\centering}}
            \caption[参与构建随机森林的特征贡献度(部分)]{参与构建随机森林的特征贡献度(部分)}\\
            \label{tab:rf_dr_1}\\
            \topline
            \colorhead \textbf{特征名}&\textbf{贡献度}&\textbf{特征名}&\textbf{贡献度}&\textbf{特征名}&\textbf{贡献度}\\
            \midline
            \endfirsthead
            \caption[]{(续)}\\
            \midline
            \colorhead \textbf{特征名}&\textbf{贡献度}&\textbf{特征名}&\textbf{贡献度}&\textbf{特征名}&\textbf{贡献度}\\
            \midline
            \endhead 
            \midline
            \endfoot
            \bottomline
            \endlastfoot
            \cellcolor{cyan}CVALF\_9                         & \cellcolor{cyan}4.6\%                            & \cellcolor{cyan}LVRF\_9                          & \cellcolor{cyan}3.4\%                            & SVD\_10                          & 3.0\%                            \\
            \cellcolor{cyan}SVAF\_10                         & \cellcolor{cyan}2.5\%                            & \cellcolor{cyan}LVALF\_7                         & \cellcolor{cyan}2.4\%                            & SVAT\_10                         & 2.3\%                            \\
            \cellcolor{cyan}CVRF\_11                         & \cellcolor{cyan}2.1\%                            & \cellcolor{pink}SVAR\_10                         & \cellcolor{pink}1.8\%                            & STDZ\_3                          & 1.6\%                            \\
            \cellcolor{cyan}LVRF\_8                          & \cellcolor{cyan}1.6\%                            & \cellcolor{cyan}LVALF\_6                         & \cellcolor{cyan}1.6\%                            & CVD\_11                          & 1.6\%                            \\
            STDZ\_1                                          & 1.5\%                                            & \cellcolor{cyan}CVALF\_8                         & \cellcolor{cyan}1.5\%                            & SVD\_9                           & 1.4\%                            \\
            \cellcolor{pink}SVAAR\_10                        & \cellcolor{pink}1.4\%                            & \cellcolor{cyan}CVALF\_1                         & \cellcolor{cyan}1.4\%                            & \cellcolor{pink}SVAAR\_8                         & \cellcolor{pink}1.3\%                            \\
            SVD\_8                                           & 1.1\%                                            & \cellcolor{pink}SVAR\_9                          & \cellcolor{pink}1.0\%                            & \cellcolor{cyan}SVAF\_9                          & \cellcolor{cyan}1.0\%                            \\
            \cellcolor{cyan}CVAAF\_1                         & \cellcolor{cyan}1.0\%                            & \cellcolor{cyan}LVLF\_8                          & \cellcolor{cyan}0.9\%                            & SVAT\_8                          & 0.9\%                            \\
            \cellcolor{cyan}CVRF\_1                          & \cellcolor{cyan}0.9\%                            & \cellcolor{cyan}CVAAF\_2                         & \cellcolor{cyan}0.9\%                            & \cellcolor{pink}SVAR\_8                          & \cellcolor{pink}0.8\%                            \\
            \cellcolor{cyan}SVAF\_2                          & \cellcolor{cyan}0.8\%                            & \cellcolor{pink}CVRR\_11                         & \cellcolor{pink}0.8\%                            & \cellcolor{cyan}LVRF\_1                          & \cellcolor{cyan}0.7\%                            \\
            \cellcolor{pink}SVSR\_7                          & \cellcolor{pink}0.7\%                            & \cellcolor{cyan}CVRF\_3                          & \cellcolor{cyan}0.7\%                            & \cellcolor{pink}CVALR\_10                        & \cellcolor{pink}0.7\%                            \\
            \cellcolor{cyan}CVALF\_7                         & \cellcolor{cyan}0.7\%                            & \cellcolor{pink}CVAAR\_5                         & \cellcolor{pink}0.7\%                            & \cellcolor{cyan}LVRF\_2                          & \cellcolor{cyan}0.6\%                           
      \end{longtable}
\end{center}

在获取了PPG多维度时域特征集中中所有特征对随机森林的贡献度后,此时即可按照贡献度按从高到低的顺序进行特征筛选。
本研究按10\%为梯度,完成了从贡献度10\%到90\%的特征筛选过程,基于每个梯度下筛选出的特征子集也使用随机森林算法

\begin{center}
      \zihao{-5}
      \begin{longtable}{m{2cm}<{\centering}m{1.3cm}<{\centering}m{1.3cm}<{\centering}m{1.6cm}<{\centering}m{1cm}<{\centering}m{1cm}<{\centering}m{1cm}<{\centering}m{1cm}<{\centering}m{1cm}<{\centering}}
            \caption{随机森林对脉搏波特征降维效果}\\
            \label{tab:rf_dr_2}\\
            \topline
            \colorhead \multicolumn{3}{c}{\textbf{随机森林特征输入}}              &  &  & \multicolumn{4}{c}{\textbf{测试集}}                                          \\
            \colorhead \textbf{贡献度比例} & \textbf{特征数量} & \textbf{数量比例} & \multirow{-2}{*}{\textbf{训练时间}}  & \multirow{-2}{*}{\textbf{AUC}}   & \textbf{精确率} & \textbf{召回率} & \textbf{F1值} & \textbf{准确率} \\
            \midline
            \endfirsthead
            \caption[]{(续)}\\
            \midline
            \colorhead \multicolumn{3}{c}{\textbf{随机森林特征输入}}              &  &  & \multicolumn{4}{c}{\textbf{测试集}}                                          \\
            \colorhead \textbf{贡献度比例} & \textbf{特征数量} & \textbf{数量比例} & \multirow{-2}{*}{\textbf{训练时间}}  & \multirow{-2}{*}{\textbf{AUC}}   & \textbf{精确率} & \textbf{召回率} & \textbf{F1值} & \textbf{准确率} \\
            \midline
            \midline
            \endhead 
            \midline
            \endfoot
            \bottomline
            \endlastfoot
            \colorrowa 10.0\%         & 3             & 1.0\%         & 5.31     & 0.931      & 89.3\%       & 86.8\%       & 88.0\%       & 86.0\%       \\
            \colorrowc 20.0\%         & 7             & 2.4\%         & 7.82     & 0.961      & 94.5\%       & 90.1\%       & 92.3\%       & 91.0\%       \\
            \colorrowa 30.0\%         & 13            & 4.5\%         & 9.79     & 0.987      & 97.4\%       & 96.4\%       & 96.9\%       & 96.3\%       \\
            \colorrowc 40.0\%         & 21            & 7.3\%         & 11.90    & 0.993      & 97.7\%       & 96.8\%       & 97.3\%       & 96.8\%       \\
            \colorrowa 50.0\%         & 34            & 11.9\%        & 15.17    & 0.993      & 97.8\%       & 97.0\%       & 97.4\%       & 97.0\%       \\
            \colorrowc 60.0\%         & 51            & 17.8\%        & 20.91    & 0.994      & 98.3\%       & 97.1\%       & 97.7\%       & 97.3\%       \\
            \colorrowa 70.0\%         & 74            & 25.9\%        & 23.67    & 0.993      & 98.1\%       & 97.5\%       & 97.8\%       & 97.4\%       \\
            \colorrowc 80.0\%         & 108           & 37.8\%        & 29.33    & 0.993      & 97.7\%       & 97.2\%       & 97.5\%       & 97.0\%       \\
            \colorrowa 90.0\%         & 162           & 56.6\%        & 38.11    & 0.991      & 98.1\%       & 97.3\%       & 97.8\%       & 97.3\%       \\
            \colorrowc 100.0\%        & 286           & 100.0\%       & 41.30    & 0.990      & 98.0\%       & 97.0\%       & 97.5\%       & 97.0\%       \\
      \end{longtable}
\end{center}

\begin{figure}[htbp]
      \centering
      \includegraphics[width=0.55\linewidth]{results/rf_ip_pulse_0.52}
      \caption[各特征对随机森林的相对贡献度(局部)]{\label{fig:rf_importance_pulse}各特征对随机森林的相对贡献度(局部)}
\end{figure}

\noindent
训练了PE识别模型,这一过程中的具体数值结果如\autoref{tab:rf_dr_2}所示。

从上述特征降维处理过程中可以得到以下结论:

1、本研究提出的PPG多维度时域特征集中包含了与PE相关性较强的特征。从数值上来看,特征集的286个特征的平均贡献度为0.34\%,而对由随机森林算法得到的贡献度最高的单特征CVALF\_9的贡献度为4.6\%,该值为平均值的13.5倍。
而从\autoref{tab:rf_dr_2}的特征贡献度来看,PPG多维度时域特征集中的286个特征中与PE相关性较强的特征较少,存在着无PE相关性较低的冗余特征。在保留原始特征集90\%的贡献度的条件下,就可以
在几乎不损失模型预测能力的条件下排除约43.4\%的无关特征。

2、从\autoref{tab:rf_dr_1}中特征种类来看,与下降支相关的特征要远多于上升支的特征。这也与PPG下降支是血液回流过程的反应、下降支包含了更多的血液循环中细节信息的事实相符合。同时,\autoref{tab:rf_dr_1}中
贡献度较高的特征主要集中在前端(1-3)与尾端(8-10)。这说明PPG下降支的前端与尾端可能包含了更多与PE相关的生理学信息。而从这些特征计算过程中使用的基准点确定策略来看,
左视类指标相较中视类指标与分层类指标也更少。这说明了较另外两种策略而言,左视类指标对下降支的表征效果最差。

3、从\autoref{tab:rf_dr_2}的特征降维过程来看,随着特征数量的减少,随机森林模型的训练速度大大加快,但随着特征数量的减少,新生成的随机森林模型在训练集与测试集上均出现了性能小幅提高再下降的整体趋势,
模型最佳性能出现在贡献度60\%-70\%之间。这说明筛选刚开始进行时排除了大量无关特征,从而使模型性能得到了提升;而随着筛选的进行,对模型生成有贡献度的特征数量的也在减少,导致随机森林的性能降低。

\subsection{所有PPG波形与PE之间关系}

本文第四章已经阐述过,研究所有PPG波形与PE之间关系并在此基础上得到的识别PE的机器学习模型具有更强泛化的能力及更广泛的适用性。
本研究将63名被试孕妇的全部PPG波形被划分为训练集,余下16名被试的全部波形被划分为测试集。
在此基础上,使用了在研究单个PPG波形与PE之间关系时综合性能较好的K近邻、决策树及随机森林等三种算法进行PE识别模型的训练。
\autoref{tab:model_screen2}给出了由这三种算法训练得到的模型的具体表现,其中,训练集上的AUC数值也是在测试集上进行5层交叉验证后求均值得到的。

横向对比\autoref{tab:model_screen}、\autoref{tab:rf_dr_2}及\autoref{tab:model_screen2}中结果可以发现,在研究被试所有PPG波形与PE之间关系时,K近邻、决策树及随机森林等三种算法在训练集与测试集上的均出现了
较为明显的性能下降,在测试集上性能下降得尤为明显。这一现象可能与测试集中的被试的任何PPG波形从未在训练集中出现有关。从\autoref{tab:model_screen2}可以发现,由随机森林算法与K近邻算法训练所得的模型的性能接近,在测试集上F1值可达到83\%以上,准确率也在78\%以上;而决策树算法生成的模型性能最差,其F1值与准确率分别仅为76.6\%与66.3\%。

注意到\autoref{tab:model_screen2}中测试集的混淆矩阵仍是按照单个PPG波形统计的,并不能完全反应被试所有PPG波形与PE之间关系。
将\autoref{tab:model_screen2}测试集中反应单个PPG波形的识别结果的混淆矩阵按其与测试集16名被试的隶属关系重新统计可得到\autoref{tab:model_detail}。
\autoref{tab:model_detail}统计了每名被试所对应的PPG波形总数与使用由三种算法得到的PE识别模型判断为PE的PPG波形数目,而预测比例一栏为这两个PPG波形数目的比值。
另外,每名被试所对应的真实PE状态也在\autoref{tab:model_detail}进行了标记。


若将\autoref{tab:model_detail}中经由各模型对应的预测比例作为判断识别PE的依据,则对PE的识别分析可以转换为寻找
能将这些预测比例进行分割的最佳阈值。\autoref{fig:model_detail}展示了测试集每名被试的预测比例所对应的散点图。
对特定的模型而言,依次将该模型下所有的预测比例数值作为分割阈值,可以得到在该数值下对PE识别的混淆矩阵,同时得到该值对应的敏感性与特异性。借助与每个预测比例数值对应的敏感性与特异性最终,最终可得到经由三种算法得到的预测比例
ROC曲线如\autoref{fig:model_roc}所示。其中,决策树、K近邻及随机森林三种算法对应的AUC数值分别为0.825、0.921及0.952。借助约登指数可以确定三种模
型的最佳分割阈值分别为0.969、0.907与0.688,各模型在最佳阈值下的准确率分别为81.3\%、81.3\%与93.8\%,其
具体混淆矩阵如\autoref{tab:cm_on_best}所示。

\begin{center}
      \zihao{-5}
      \begin{longtable}{m{2cm}<{\centering}m{1.1cm}<{\centering}m{1cm}<{\centering}m{1cm}<{\centering}m{1cm}<{\centering}m{1cm}<{\centering}m{1cm}<{\centering}m{1cm}<{\centering}m{1cm}<{\centering}m{1cm}<{\centering}m{1cm}<{\centering}}
            \caption{几种机器学习模型在被试人员分层抽样的数据集上的表现}\\
            \label{tab:model_screen2}\\
            \topline
            \colorhead &  & \multicolumn{5}{c}{\textbf{训练集}} & \multicolumn{4}{c}{\textbf{验证集}}                                                                                                                                                                                                      \\
            \colorhead \multirow{-2}{*}{\textbf{模型类型}} & \multirow{-2}{*}{\textbf{训练时间}}  &  \textbf{精确率} &  \textbf{召回率} &  \textbf{F1值} &  \textbf{准确率} &  \textbf{AUC}  &  \textbf{精确率} &  \textbf{召回率} &  \textbf{F1值} &  \textbf{准确率}    \\
            \midline
            \endfirsthead
            \caption[]{(续)}\\
            \midline
            \colorhead &  & \multicolumn{5}{c}{\textbf{训练集}} & \multicolumn{4}{c}{\textbf{验证集}}                                                                                                                                                                                                      \\
            \colorhead \multirow{-2}{*}{\textbf{模型类型}} & \multirow{-2}{*}{\textbf{训练时间}}  &  \textbf{精确率} &  \textbf{召回率} &  \textbf{F1值} &  \textbf{准确率} &  \textbf{AUC}  &  \textbf{精确率} &  \textbf{召回率} &  \textbf{F1值} &  \textbf{准确率}    \\
            \midline
            \endhead 
            \midline
            \endfoot
            \bottomline
            \endlastfoot
            \colorrowa K近邻算法      &   4.08 s   & 88.0\% & 76.0\% &81.6\% & 80.0\% & 0.870 & 81.5\% & 85.6\% & 83.5\% & 78.3\% \\
            \colorrowc 决策树算法      &   1.44 s  & 84.0\% & 80.9\% & 82.4\% & 79.8\% & 0.862 & 69.0\% & 86.1\% & 76.6\% & 66.3\% \\
            \colorrowa 随机森林算法      &   50.03 s  & 90.6\% & 80.6\% & 85.3\% & 83.8\% & 0.929 & 79.5\% & 90.9\% & 84.8\% & 79.1\% \\
      \end{longtable}
\end{center}

\begin{center}
      \zihao{-5}
      \begin{longtable}{m{1.2cm}<{\centering}m{1cm}<{\centering}m{1.2cm}<{\centering}m{1.5cm}<{\centering}m{1.5cm}<{\centering}m{1.5cm}<{\centering}m{1.5cm}<{\centering}m{1.5cm}<{\centering}m{1.5cm}<{\centering}m{1cm}<{\centering}}
            \caption{几种机器学习模型按被试统计后的性能表现}\\
            \label{tab:model_detail}\\
            \topline
            \colorhead  &  & & \multicolumn{2}{c}{\textbf{K近邻算法}} & \multicolumn{2}{c}{\textbf{决策树}} & \multicolumn{2}{c}{\textbf{随机森林}}  \\
            \colorhead \multirow{-2}{*}{\textbf{被试孕妇}}  &     \multirow{-2}{*}{\textbf{PE状态}} &   \multirow{-2}{*}{\textbf{波形总数}}   & \textbf{预测数目}     & \textbf{预测比例}       & \textbf{预测数目}     & \textbf{预测比例}       & \textbf{预测数目}     & \textbf{预测比例}             \\
            \midline
            \endfirsthead
            \caption[]{(续)}\\
            \midline
            \colorhead  &  & & \multicolumn{2}{c}{\textbf{K近邻算法}} & \multicolumn{2}{c}{\textbf{决策树}} & \multicolumn{2}{c}{\textbf{随机森林}}  \\
            \colorhead \multirow{-2}{*}{\textbf{被试孕妇}}  &     \multirow{-2}{*}{\textbf{PE状态}} &   \multirow{-2}{*}{\textbf{波形总数}}   & \textbf{预测数目}     & \textbf{预测比例}       & \textbf{预测数目}     & \textbf{预测比例}       & \textbf{预测数目}     & \textbf{预测比例}             \\
            \midline
            \endhead 
            \midline
            \endfoot
            \bottomline
            \endlastfoot
            \colorrowa cmf       & 0           & 88                    & 54         & 61.4\%     & 85         & 96.6\%     & 53         & 60.2\%                                                                            \\
            \colorrowc lxx       & 0           & 63                    & 32         & 50.8\%     & 53         & 84.1\%     & 34         & 54.0\%                                                                            \\
            \colorrowa shs       & 0           & 112                   & 37         & 33.0\%     & 105        & 93.8\%     & 55         & 49.1\%                                                                            \\
            \colorrowc sxh       & 0           & 95                    & 27         & 28.4\%     & 64         & 67.4\%     & 21         & 22.1\%                                                                            \\
            \colorrowa wdq       & 0           & 36                    & 0          & 0.0\%      & 0          & 0.0\%      & 0          & 0.0\%                                                                             \\
            \colorrowc wsj       & 0           & 78                    & 0          & 0.0\%      & 2          & 2.6\%      & 0          & 0.0\%                                                                             \\
            \colorrowa ygy       & 0           & 75                    & 40         & 53.3\%     & 70         & 93.3\%     & 67         & 89.3\%                                                                            \\
            \colorrowc gmn       & 1           & 139                   & 106        & 76.3\%     & 135        & 97.1\%     & 109        & 78.4\%                                                                            \\
            \colorrowa ty        & 1           & 98                    & 97         & 99.0\%     & 97         & 99.0\%     & 97         & 99.0\%                                                                            \\
            \colorrowc wjh       & 1           & 86                    & 86         & 100.0\%    & 86         & 100.0\%    & 86         & 100.0\%                                                                           \\
            \colorrowa xjf       & 1           & 106                   & 23         & 21.7\%     & 59         & 55.7\%     & 87         & 82.1\%                                                                            \\
            \colorrowc ywy       & 1           & 111                   & 110        & 99.1\%     & 111        & 100.0\%    & 111        & 100.0\%                                                                           \\
            \colorrowa yxl       & 1           & 110                   & 110        & 100.0\%    & 108        & 98.2\%     & 110        & 100.0\%                                                                           \\
            \colorrowc zdq       & 1           & 89                    & 81         & 91.0\%     & 84         & 94.4\%     & 82         & 92.1\%                                                                            \\
            \colorrowa zl        & 1           & 152                   & 137        & 90.1\%     & 75         & 49.3\%     & 120        & 78.9\%                                                                            \\
            \colorrowc zyy       & 1           & 89                    & 89         & 100.0\%    & 89         & 100.0\%    & 89         & 100.0\%                                                                            \\    
      \end{longtable}
\end{center}

\begin{figure}[htbp]
      \centering
      \includegraphics[width=.55\linewidth]{results/detail}
      \caption[基于三种算法得到的子痫前期预测比例散点图]{\label{fig:model_detail}基于三种算法得到的子痫前期预测比例散点图}
\end{figure}

\begin{figure}[htbp]
      \centering
      \includegraphics[width=.55\linewidth]{results/roc}
      \caption[由子痫前期预测比例得到的ROC曲线]{\label{fig:model_roc}由子痫前期预测比例得到的ROC曲线}
\end{figure}

\begin{center}
      \begin{longtable}{m{4cm}<{\centering}m{4cm}<{\centering}m{4cm}<{\centering}}
            \caption{三种模型在最佳分割阈值下的混淆矩阵}\\
            \label{tab:cm_on_best}\\
            \topline
            \textbf{决策树算法}&\textbf{K近邻算法}&\textbf{随机森林算法}\\
            \midline
            \endfirsthead
            \caption[]{(续)}\\ 
            \midline
            \textbf{决策树算法}&\textbf{K近邻算法}&\textbf{随机森林算法}\\
            \endhead 
            \midline
            \endfoot
            \bottomline
            \endlastfoot
            $\left[ \begin{array}{cc} 6 & 3 \\ 0 & 7 \end{array} \right]$ & $\left[ \begin{array}{cc} 6 & 3 \\ 0 & 7 \end{array} \right]$ & $\left[ \begin{array}{cc} 9 & 0 \\ 1 & 6 \end{array} \right]$ \\
      \end{longtable}
\end{center}


从上述研究结果中可以发现,在研究每名所有PPG波形与PE之间关系时,借助由具体算法得到的PE识别模型将判断为PE的PPG波形数目与同一被试的PPG波形总数的比例数值,可以较好的识别PE。通过AUC与ROC分析,三种算法中随机森林算法有着最佳的分类效果,其AUC数值最大(0.952)、识别准确率最高
(93.8\%),另外随机森林算法识别为假阴性的数目(0)也少于其他两种模型(均为3)。而从最佳分割阈值来看,决策树与K近邻算法的预测比例分割阈值(0.969与0.907)过于接近100\%,有泛化能力不足的潜在风险,而随机森林模型的分割阈值(0.688)则相对适中。

\section{基于PPG采样值时域特征集的机器学习分析}
本小节基于PPG采样值时域特征集,也分别研究了被试单个PPG波形、所有PPG波形与PE之间关系。

\subsection{单个PPG波形与PE之间关系}
与上部分类似,本部分也进行了机器学习算法初筛与调优及特征贡献度分析等研究工作。

一、模型初筛与调优

在上小节研究内容的基础上,本部分对初筛阶段使用的机器学习算法数量进行了缩减,只选用此前综合性能较好的决策树、K近邻及随机森林等三种算法。
在将采样值时域特征集按PPG波形划分数据集的基础上,这三种算法在的综合表现如\autoref{tab:model_screen3}所示。
其中,由决策树与K近邻算法训练得到的模型的统计数据是在全局最优超参数的条件下得到的,而各模型训练集的统计结果是经5层交叉验证处理后得到的。另外,\autoref{tab:model_screen3}中处理方式
一栏对应着第四章中介绍过的解决PPG波形时长差异的两种处理策略,即重采样策略与短端补齐策略(下同)。

从\autoref{tab:model_screen3}中可以得到以下结论:

1、从模型的训练时间方面来看,使用K近邻算法构建模型的速度最快,而使用随机森林算法速度最慢。这一结果也与各算法的时间复杂度一致。

\begin{center}
      \zihao{-5}
      \begin{longtable}{m{1.2cm}<{\centering}m{1.2cm}<{\centering}m{1.2cm}<{\centering}m{0.9cm}<{\centering}m{0.9cm}<{\centering}m{0.9cm}<{\centering}m{0.9cm}<{\centering}m{0.9cm}<{\centering}m{0.9cm}<{\centering}m{0.9cm}<{\centering}m{0.9cm}<{\centering}m{0.9cm}<{\centering}}
            \caption{基于脉搏波原始采样点的识别模型的初筛结果}\\
            \label{tab:model_screen3}\\
            \topline
            \colorhead       & \multicolumn{1}{c}{}   & \multicolumn{1}{c}{}  & \multicolumn{5}{c}{\textbf{训练集(5层交叉验证)}}   & \multicolumn{4}{c}{\textbf{验证集}}     \\
            \colorhead \multirow{-2}{*}{\textbf{处理方式}}  & \multirow{-2}{*}{\textbf{模型类型}} & \multirow{-2}{*}{\textbf{训练时间}} & \textbf{精确率} & \textbf{召回率}& \textbf{F1值} & \textbf{准确率}& \textbf{AUC} & \textbf{精确率} & \textbf{召回率} & \textbf{F1值}& \textbf{准确率} \\
            \midline
            \endfirsthead
            \caption[]{(续)}\\
            \midline
            \colorhead       & \multicolumn{1}{c}{}   & \multicolumn{1}{c}{}  & \multicolumn{5}{c}{\textbf{训练集(5层交叉验证)}}   & \multicolumn{4}{c}{\textbf{验证集}}                                                                                                                                                                                                    \\
            \colorhead \multirow{-2}{*}{\textbf{处理方式}}  & \multirow{-2}{*}{\textbf{模型类型}} & \multirow{-2}{*}{\textbf{训练时间}} & \textbf{精确率} & \textbf{召回率}& \textbf{F1值} & \textbf{准确率}& \textbf{AUC} & \textbf{精确率} & \textbf{召回率} & \textbf{F1值}& \textbf{准确率} \\
            \midline
            \endhead 
            \midline
            \endfoot
            \bottomline
            \endlastfoot
            \colorrowa & 决策树      & 4.13    & 83.9\%  & 88.6\%  & 86.1\% & 83.0\% & 0.914   & 86.6\%  & 90.9\%  & 88.7\% & 86.2\% \\
            \colorrowa & K近邻     & 2.81     & 93.4\%  & 90.0\%  & 91.4\% & 90.0\%   & 0.967  & 94.5\%   & 90.7\%   & 92.6\% & 91.4\% \\
            \colorrowa \multirow{-3}{*}{补零} & 随机森林    & 23.23   & 93.3\%  & 92.8\% & 93.0\% & 91.7\%  & 0.977 & 94.8\% & 94.5\%   & 94.6\% & 93.6\% \\
            \colorrowc & 决策树      & 6.77   & 85.6\%  & 85.2\%  & 85.4\% & 82.6\% & 0.902    & 88.3\%  & 84.1\%  & 86.2\% & 83.9\% \\
            \colorrowc & K近邻     & 3.90    & 91.5\%  & 89.5\%  & 90.5\% & 88.8\%   & 0.957  & 94.2\%   & 91.2\%   & 92.9\% & 91.6\% \\
            \colorrowc \multirow{-3}{*}{重采样} & 随机森林    & 51.57    & 91.4\%  & 91.9\% & 91.6\% & 90.0\%  & 0.967  & 93.4\% & 94.0\%   & 93.7\% & 92.5\% \\
      \end{longtable}
\end{center}

2、从模型在测试集上的表现来看,在重采样与短端补齐等两种处理PPG时长差异的策略下,由三种算法得到的模型均能对测试集数据进行较好的学习,各模型的AUC数值均超过了0.900,其中随机森林的AUC数值更是在0.967以上。
而从精度-召回率权衡方面来看,K近邻与随机森林算法明显优于决策树算法,使用这两种算法构建得到的模型的精确率、召回率及F1数值几乎全部在 90.0\% 以上(只有在重采样策略下使用K近邻算法得到的模型的召回率低于90.0\%,为89.5\%)。

3、从模型在验证集上的泛化能力方面来看,K近邻与随机森林算法也明显优于决策树算法。就准确率而言,由K近邻与随机森林算法的到的模型可达到91\%以上,而由决策树算法得到的模型准确率则不足87\%。
在精度­召回率权衡上,K近邻与随机森林算法也明显优于决策树算法,前两种算法训练得到的模型的精确率、召回率及F1等数值均在90.0\%以上,均明显高于决策树算法得到的模型表现。

4、综合2与3中结论,可以得到随机森林算法的综合性能表现最好,K近邻算法次之,决策树算法表现较差。这也验证了此前阐述过的集成学习算法较单一算法在训练模型时具有性能优势。

5、横向对比\autoref{tab:model_screen3}中重采样与短端补齐等两种处理PPG时长差异的策略可以发现,使用短端补齐策略会各模型在构建速度、训练集AUC及模型整体准确率、精确率等方面均略微优于使用重采样策略的对应结果。

二、特征贡献度分析

与上小节中处理类似,在PPG采样值时域特征集上使用随机森林算法构建模型后,同样可以得到特征集中各特征对最终模型的贡献度,也即
原始PPG中不同位置上的采样点对模型的贡献度。由于对这些原始采样点进行降维处理无实际意义,本部分未进行相应的降维分析与处理工作。

在\autoref{tab:model_screen3}中两种处理PPG时长差异的策略下,统计各采样点的在生成的随机森林中的平均深度后,得到各采样点的贡献度如\autoref{fig:rf_imp2}所示。
为直观显现各采样点贡献度与PPG波形的对应关系,\autoref{fig:rf_imp2}也同时绘制了在两种策略下的“平均”PPG波形,即所有PPG波形按采样点对齐再计算算术平均值的归一化结果。

\begin{figure}[htbp]
      \centering
      \includegraphics[width=.6\linewidth]{results/rf_imp2}
      \caption{\label{fig:rf_imp2}脉搏波采样点对随机森林的贡献度}
\end{figure}

从\autoref{fig:rf_imp2}可以得到以下结论:

1、两种策略下,各采样点的贡献度分布形态高度相似,均在平均PPG波形的波峰之后出现贡献度的主峰,在平均PPG波形的下降支尾端附近出现第二个次峰。平均PPG波形起点与终点处的采样点较
平均PPG波形波峰附近的采样点贡献度可以忽略不计。

2、对两种处理PPG时长差异的策略而言,尽管短端补齐的策略会得到更多的采样点数(120),但采样点贡献度的具体分布反而更为集中,有效特征点的贡献度数值也更大。
第四章中已经阐述过,本研究采集得到的绝大多数PPG波形采样点数均不超过80。而从\autoref{fig:rf_imp2}中也可以观察到,短端补齐处理后,序号在80以后的采样点贡献度几乎全部为0,也说明这部分采样点几乎没有参与到最终随机森林模型的构建之中。
但对重采样策略而言,重采样后所有PPG波形的采样点数均为100,且不完全为0,这也与\autoref{fig:rf_imp2}中该策略下所有采样点的贡献度均不为0的现象相匹配。

\subsection{所有PPG波形与PE之间关系}

与在PPG多维度时域特征集上的处理类似,本研究也基于PPG采样值时域特征集,研究了被试所有PPG波形与PE之间关系。
63名被试孕妇的全部PPG波形被划分为训练集,余下16名被试的全部波形被划分为测试集。而K近邻、决策树及随机森林等三种算法也被选作PE识别模型的训练算法。\autoref{tab:model_screen4}给出了由这三种算法训练得到
的模型的具体表现,其中,训练集上的 AUC 数值也是在测试集上进行 5 层交叉验证后
求均值得到的。

从\autoref{tab:model_screen4}中可以得到以下结论:

1、在研究被试所有PPG波形与PE之间关系时,K近邻、决策树及随机森林等三种

\begin{center}
      \zihao{6}
      \begin{longtable}{m{1.3cm}<{\centering}m{1.3cm}<{\centering}m{1.3cm}<{\centering}m{0.9cm}<{\centering}m{0.9cm}<{\centering}m{0.9cm}<{\centering}m{0.9cm}<{\centering}m{0.9cm}<{\centering}m{0.9cm}<{\centering}m{0.9cm}<{\centering}m{0.9cm}<{\centering}m{0.9cm}<{\centering}}
            \caption{几种机器学习模型在被试人员分层抽样的数据集上的表现}\\
            \label{tab:model_screen4}\\
            \topline
            \colorhead &     &  & \multicolumn{5}{c}{\textbf{训练集}} & \multicolumn{4}{c}{\textbf{验证集}}                                                                                                                                                                                                      \\
            \colorhead \multirow{-2}{*}{\textbf{处理方式}}&\multirow{-2}{*}{\textbf{模型类型}} & \multirow{-2}{*}{\textbf{训练时间}}  &  \textbf{精确率} &  \textbf{召回率} &  \textbf{F1值} &  \textbf{准确率} &  \textbf{AUC}  &  \textbf{精确率} &  \textbf{召回率} &  \textbf{F1值} &  \textbf{准确率}    \\
            \midline
            \endfirsthead
            \caption[]{(续)}\\
            \midline
            \colorhead &     &  & \multicolumn{5}{c}{\textbf{训练集}} & \multicolumn{4}{c}{\textbf{验证集}}                                                                                                                                                                                                      \\
            \colorhead \multirow{-2}{*}{\textbf{处理方式}}&\multirow{-2}{*}{\textbf{模型类型}} & \multirow{-2}{*}{\textbf{训练时间}}  &  \textbf{精确率} &  \textbf{召回率} &  \textbf{F1值} &  \textbf{准确率} &  \textbf{AUC}  &  \textbf{精确率} &  \textbf{召回率} &  \textbf{F1值} &  \textbf{准确率}    \\
            \midline
            \endhead 
            \midline
            \endfoot
            \bottomline
            \endlastfoot
            \colorrowa &     K近邻      &   3.12 s  & 86.0\% & 73.7\% &79.4\% & 77.6\% & 0.863 & 81.5\% & 85.6\% & 83.5\% & 78.3\% \\
            \colorrowa &     决策树      &   3.62 s  & 84.0\% & 82.2\% & 83.1\% & 80.5\% & 0.880 & 70.6\% & 84.6\% & 77.0\% & 67.5\% \\
            \colorrowa \multirow{-3}{*}{补零}  & 随机森林      &   23.18 s  & 87.0\% & 81.5\% & 84.2\% & 82.1\% & 0.909 & 76.5\% & 86.5\% & 81.2\% & 74.3\% \\
            \colorrowc &     K近邻      &   2.51 s   & 85.5\% & 76.9\% &81.0\% & 78.9\% & 0.862 & 76.7\% & 80.1\% & 78.4\% & 71.6\% \\
            \colorrowc &     决策树      &   1.49 s  & 82.7\% & 76.9\% & 79.7\% & 77.1\% & 0.847 & 76.2\% & 88.1\% & 81.7\% & 74.7\% \\
            \colorrowc \multirow{-3}{*}{重采样}  & 随机森林      &   36.36 s  & 85.1\% & 77.8\% & 81.2\% & 79.0\% & 0.884 & 78.9\% & 87.6\% & 83.0\% & 77.0\% \\
      \end{longtable}
\end{center}

\begin{center}
      \zihao{6}
      \begin{longtable}{m{1cm}<{\centering}m{1cm}<{\centering}m{1cm}<{\centering}m{1.5cm}<{\centering}m{2cm}<{\centering}m{1.5cm}<{\centering}m{2cm}<{\centering}m{1.5cm}<{\centering}m{2cm}<{\centering}}
            \caption[几种机器学习模型按被试统计后的性能表现]{几种机器学习模型按被试统计后的性能表现}\\
            \label{tab:model_detail2}\\
            \topline
            \colorhead  &  &  & \multicolumn{2}{c}{\textbf{K近邻算法}} & \multicolumn{2}{c}{\textbf{决策树}} & \multicolumn{2}{c}{\textbf{随机森林}}  \\
            \colorhead \multirow{-2}{*}{\textbf{被试孕妇}}  &     \multirow{-2}{*}{\textbf{PE状态}} &   \multirow{-2}{*}{\textbf{波形总数}}   & \textbf{预测数目}     & \textbf{预测比例}       & \textbf{预测数目}     & \textbf{预测比例}       & \textbf{预测数目}     & \textbf{预测比例}             \\
            \midline
            \endfirsthead
            \caption[]{(续)}\\
            \midline
            \colorhead  &  &  & \multicolumn{2}{c}{\textbf{K近邻算法}} & \multicolumn{2}{c}{\textbf{决策树}} & \multicolumn{2}{c}{\textbf{随机森林}}  \\
            \colorhead \multirow{-2}{*}{\textbf{被试孕妇}}  &     \multirow{-2}{*}{\textbf{PE状态}} &   \multirow{-2}{*}{\textbf{波形总数}}   & \textbf{预测数目}     & \textbf{预测比例}       & \textbf{预测数目}     & \textbf{预测比例}       & \textbf{预测数目}     & \textbf{预测比例}             \\
            \midline
            \endhead 
            \midline
            \endfoot
            \bottomline
            \endlastfoot
            \colorrowa cmf       & 0           & 88                    & 66 / 59         & 75.0\% / 67.0\%     & 75 / 60         & \textbf{85.2\% / 68.2\%}    & 61 / 49         & 69.3\% / 55.7\%                                     \\
            \colorrowc lxx       & 0           & 63                    & 32 / 35         & 50.8\% / 55.6\%    & 50 / 38         & \textbf{79.4\% / 60.3\%}     & 37 / 42         & 58.7\% / 66.7\%                                            \\
            \colorrowa shs       & 0           & 112                   & \textbf{32 / 58}         & \textbf{28.6\% / 51.8\%}     & 102 / 88        & 91.1\% / 78.6\%     & 57 / 47         & 50.1\% / 42.0\%                                                               \\
            \colorrowc sxh       & 0           & 95                    & 31 / 24         & 32.6\% / 25.3\%     & \textbf{52 / 27}         & \textbf{54.7\% / 28.4\%}     & 39 / 21         & \textbf{41.1\% / 22.1\%}                                                               \\
            \colorrowa wdq       & 0           & 36                    & 0 / 0          & 0.0\% / 0.0\%      & 2 / 0          & 5.6\% / 0.0\%      & 0 / 0          & 0.0\% / 0.0\%                                              \\
            \colorrowc wsj       & 0           & 78                    & 0 / 1          & 0.0\% / 0.0\%      & 1 / 0          & 1.3\% / 0.0\%     & 0 / 0          & 0.0\% / 0.0\%                                              \\
            \colorrowa ygy       & 0           & 75                    & 57 / 61         & 76.0\% / 81.3\%     & 63 / 56         & 84.0\% / 74.7\%    & 66 / 70         & 88.0\% / 93.3\%                                                      \\
            \colorrowc gmn       & 1           & 139                   & \textbf{51 / 82}        & \textbf{36.7\% / 59.0\%}     & 131 / 131        & 94.2\% /94.2\%    & 107 / 112        & 77.0\% / 80.6\%                                  \\
            \colorrowa ty        & 1           & 98                    & 98 / 98         & 100.0\% / 100.0\%     & 97 / 96         & 99.0\% / 98.0\%     & 97 / 97         & 99.0\% / 99.0\%                                                        \\
            \colorrowc wjh       & 1           & 86                    & 86 / 86         & 100.0\% / 100.0\%    & 83 / 86         & 96.5\% / 100.0\%    & 86 / 86         & 100.0\% / 100.0\%                            \\
            \colorrowa xjf       & 1           & 106                   & \textbf{13 / 71}         & \textbf{12.3\% / 67.0\%}     & \textbf{29 / 76}         & \textbf{27.4\% / 71.7\%}    & \textbf{36 / 72}         & \textbf{34.0\% / 67.9\%}                    \\
            \colorrowc ywy       & 1           & 111                   & \textbf{111 / 42}        & 100.0\% / 37.8\%     & \textbf{110 / 65}        & \textbf{99.1\% / 58.6\%}    & \textbf{110 / 66}        & \textbf{99.1\% / 59.5\%}                                 \\
            \colorrowa yxl       & 1           & 110                   & 107 / 107        & 97.3\% / 97.3\%   & 100 / 105        & 90.9\% / 95.5\%    & 108 / 110        & 98.2\% / 100.0\%                                      \\
            \colorrowc zdq       & 1           & 89                    & 77 / 70         & 86.5\% / 78.7\%     & 78 / 76         & 87.6\% / 85.4\%     & 79 / 83         & 88.8\% / 93.3\%                                       \\
            \colorrowa zl        & 1           & 152                   & \textbf{114 / 140}        & \textbf{75.0\% / 92.1\%}     & \textbf{112 / 139}         & \textbf{73.7\% / 91.4\%}     & 136 / 143        & 89.5\% / 94.1\%                                    \\
            \colorrowc zyy       & 1           & 89                    & 89 / 89         & 100.0\% / 100.0\%    & 89/89         & 100.0\% / 100.0\%    & 89/89         & 100.0\% / 100.0\%                                     \\    
      \end{longtable}
\end{center}
\noindent
算法在训练集与测试集上的均出现了较为明显的性能下降,在测试集上性能下降得尤为明显。
这一现象可能与测试集中的被试的任何PPG波形从未在训练集中出现有关。同时,三种模型在测试集上的性能并没有表现出明显的差异,在整体准确率、精确率等具体指标上均较为接近。

2、使用不同的处理PPG时长差异的策略并不会明显改变\autoref{tab:model_screen4}中各模型在PE识别问题上的性能,各模型在整体准确率、精确率等具体指标上变化幅度很小。

与在PPG多维度时域特征集上的处理类似,同样\autoref{tab:model_screen4}中按照单个PPG波形统计得到的混淆矩阵进行按是否隶属同一被试分类重新统计,结果如\autoref{tab:model_detail2}所示。
\autoref{tab:model_detail2}也统计了每名被试所对应的PPG波形总数与使用由三种算法得到的PE识别模型判断为PE的PPG波形数目,而预测比例一栏为这两个PPG波形数目的比值。另外,每名被试所对应的真
实PE状态也在\autoref{tab:model_detail2}进行了标记。

与在PPG多维度时域特征集上的处理类似,在\autoref{tab:model_detail2}的基础上,将三种算法训练所得的模型在测试集上的预测比例作为作为判断识别PE的依据,可以得到如\autoref{fig:model_detail2}所示的预测比例散点图。
也可以得到经由三种算法得到的预测比例的最佳阈值,以及在最佳阈值下的混淆矩阵、敏感性、特异性、预测准确率与AUC等具体数值,\autoref{tab:cm_on_best2}所示。在此过程中,三种模型的预测比例
所对应的ROC曲线如\autoref{fig:model_roc2}所示。
\begin{figure}[htbp]
      \centering
      \subfigure[\label{fig:detail_31}对脉搏波进行补零处理得到的预测比例散点图]{
      \includegraphics[width=7.5cm]{results/detail_31}
      }
      \quad
      \subfigure[\label{fig:detail_32}对脉搏波进行重采样处理得到的预测比例散点图]{
      \includegraphics[width=7.5cm]{results/detail_32}
      }
      \caption{\label{fig:model_detail2}基于三种算法得到的子痫前期预测比例散点图2}
\end{figure}

% \vspace{-0.5cm} 
\begin{center}
      \zihao{-5}
      % \setlength{\belowcaptionskip}{-0.5cm} 
      \begin{longtable}{m{2.8cm}<{\centering}m{2.8cm}<{\centering}m{2cm}<{\centering}m{1.5cm}<{\centering}m{1.8cm}<{\centering}m{1.8cm}<{\centering}}
            \caption{三种模型在最佳分割阈值下的混淆矩阵}\\
            \label{tab:cm_on_best2}\\
            \topline
            \colorhead \textbf{机器学习算法}&\textbf{PPG对齐策略}&\textbf{最佳阈值}&\textbf{混淆矩阵}&\textbf{准确率}&\textbf{AUC}\\
            \midline
            \endfirsthead
            \caption[]{(续)}\\ 
            \topline
            \colorhead \textbf{机器学习算法}&\textbf{PPG对齐策略}&\textbf{最佳阈值}&\textbf{混淆矩阵}&\textbf{准确率}&\textbf{AUC}\\
            \midline
            \endhead 
            \midline
            \endfoot
            \bottomline
            \endlastfoot
            \colorrowa                               &     短端补齐    & 0.864     &     $\left[ \begin{array}{cc} 7 & 2 \\ 1 & 6 \end{array} \right]$  & 81.3\% & 0.825 \\
            \colorrowa \multirow{-2}*{决策树}        &     重采样      & 0.573     &     $\left[ \begin{array}{cc} 7 & 2 \\ 0 & 7 \end{array} \right]$  & 87.5\% & 0.905  \\
            \colorrowc                               &     短端补齐    & 0.813     &     $\left[ \begin{array}{cc} 6 & 3 \\ 0 & 7 \end{array} \right]$  & 81.3\% & 0.849  \\
            \colorrowc \multirow{-2}*{K近邻}         &     重采样      & 0.820     &     $\left[ \begin{array}{cc} 8 & 1 \\ 2 & 5 \end{array} \right]$  & 81.3\% & 0.865 \\
            \colorrowa                               &     短端补齐    & 0.884     &     $\left[ \begin{array}{cc} 7 & 2 \\ 1 & 6 \end{array} \right]$  & 81.3\% & 0.905  \\
            \colorrowa \multirow{-2}*{随机森林}       &     重采样      & 0.673    &     $\left[ \begin{array}{cc} 8 & 1 \\ 1 & 6 \end{array} \right]$  & 87.5\% & 0.929  \\
      \end{longtable}
\end{center}
% \vspace{-1.0cm} 
\begin{figure}[htbp]
      \centering
      \subfigure[\label{fig:roc_31}按脉搏波进行补零处理得到的预测比例得到的ROC曲线]{
      \includegraphics[width=7.5cm]{results/roc_31}
      }
      \quad
      \subfigure[\label{fig:roc_32}按脉搏波进行重采样处理得到的预测比例得到的ROC曲线]{
      \includegraphics[width=7.5cm]{results/roc_32}
      }
      \caption{\label{fig:model_roc2}由子痫前期预测比例得到的 ROC 曲线2}
\end{figure}
% \vspace{1.5cm} 

从上述研究结果中可以得到以下结论:

1、在研究每名所有PPG波形与PE之间关系时,利用PPG采样值时域特征集,在两种PPG对齐处理策略下,借助机器学习算法可以得到性能较好的PE识别模型。

2、从识别准确率方面来看,由三种算法训练得到的模型均可达到81.3\%以上的识别准确率,其中,决策树与随机森林算法生成的模型性能接近,K近邻算法得到的模型准确率最低。从AUC数值方面来看,随机森林算法有着最佳的分类效果,AUC数值最大(0.905与0.929)。

3、对比两种PPG对齐处理策略可以发现,在两种策略下的测试集上的混淆矩阵高度相似,识别准确率也基本保持一致。这也说明了两种处理策略本身的合理性。

4、从最佳分割阈值来看,在对PPG按短端补齐处理后,三种算法对应的分割阈值较大,有泛化能力不足的潜在风险;而将PPG重采样处理后,决策树与随机森林算法得到的分割阈值则明显下降(0.573与0.673)。

\section{综合讨论与分析}
本小节在回顾前两节中的研究内容的基础上,进一步对以下问题进行讨论与分析。

一、数据集划分方式

第四章已经介绍过,在将新型时域波形描述特征集及脉搏波原始采样点按照全部波形与被试进行分层抽样、划分训练集与测试集的根本目的是为了弥补
本研究采集使用的实验数据较少。本章则在这两种数据集划分方式的基础上,分别进行了子痫前期识别模型的建立与研究。
结果表明\textbf{通过本研究提出的新型时域波形描述特征集及脉搏波原始采样点均能构建出具体一定子痫前期识别能力的机器学习模型,也即说明新型时域波形描述特征集及脉搏波原始采样点中
包含了能够表征子痫前期与正常妊娠孕妇形态差异的特征}。

\textbf{按照全部波形进行子痫前期识别模型的研究过程,实际上基于子痫前期导致的病生理变化可以在单个波形上得到体现的假设。}在这种假设中,单个脉搏波波形也包含了
识别子痫前期的全部信息。\textbf{按照被试人群进行子痫前期识别模型的研究过程,实际上基于子痫前期导致的病生理变化是通过被试全部脉搏波的多数波形的形态特征来反应的假设。}
在该假设中,同一被试的全部脉搏波波形的形态整体“群体决策”了该被试的子痫前期状态。

另一方面,在按全部波形进行训练集与测试集的划分时,同一被试的不同脉搏波波形可能会分别被划分至训练集与测试集。若这些波形高度相似,就可能最终的模型识别准确率较高。
因此,有必要按被试划分训练集与测试集的划分,确保同一被试的波形数据只出现在训练集或测试集。由于测试集对最终生成的模型而言是全新的数据,这种方式下对模型的泛化能力的评估更为有效。

最后从数值结果来看,本章的研究结果为上述两种猜想均提供了一定支撑。基于本研究提出的新型时域波形描述特征集及脉搏波原始采样点,按照两种数据集的划分方式,使用
多种机器学习训练得到的模型均有一定的子痫前期识别能力。

二、机器学习模型的影响

本研究在经过多种机器学习算法的初筛后,着重考察了两种单模型算法决策树与K近邻算法——与一种集成算法——随机森林——构建子痫前期识别模型的效果。
通过上述算法在本研究涉及了多种场景下的分析生成的子痫前期识别模型在测试集上性能表现(仅以准确率为例进行统计)如\autoref{tab:accuracy}所示。

\begin{table}[htbp]
      \zihao{-5}
      \centering
      \caption{\label{tab:accuracy}不同分析场景下通过三种算法模型在测试集上准确率对比}
      \begin{tabular}{cccccc}
      \topline
      \textbf{数据集}&\textbf{抽样方式}&\textbf{随机森林}&\textbf{K近邻}&\textbf{决策树}&\textbf{备注}\\
      \midline
      \multirow{3}{*}{新型时域波形描述特征集} & 全部波形 & 97.0\% & 93.3\% &  88.9\%& \\
            & 被试 & 79.1\% & 78.3\% & 66.3\% & 按波形统计\\
            & 被试 & 93.8\% & 81.3\% & 81.3\% & 计算预测比例后按被试统计\\
      \multirow{6}{*}{脉搏波采样点} & 全部波形 & 93.6\% & 91.4\% & 86.2\% & 脉搏波补零对齐\\
                  & 全部波形 & 92.5\% & 91.6\% & 83.9\% & 脉搏波重采样对齐\\
                  & 被试 & 74.3\% & 78.3\% & 67.5\% & 脉搏波补零对齐,按波形统计\\
                  & 被试& 77.0\% & 74.7\% & 71.6\% & 脉搏波重采样对齐,按波形统计\\
                  & 被试 & 81.3\% & 81.3\% & 81.3\% & 脉搏波补零对齐,计算预测比例后按被试统计\\
                  & 被试& 87.5\% & 81.3\% & 87.5\% & 脉搏波重采样对齐,计算预测比例后按被试统计\\      
      \bottomline
      \end{tabular}%
\end{table}%

从\autoref{tab:accuracy}可以发现,\textbf{随机森林算法充分体现了集成学习的优势,在本研究涉及的多种分析场景下,随机森林算法构建的模型性能均高于决策树算法与K近邻算法。
而对于决策树与K近邻算法,在大多数分析场景下,K近邻算法的性能表现优于决策树算法。}

三、特征集的影响

进一步考察\autoref{tab:accuracy}中使用的数据集对最终子痫前期识别模型在测试集上准确率可以发现,\textbf{本研究提出的新型时域波形描述特征集及脉搏波原始采样点在子痫前期的识别能力上具有一致性。}
在相同的研究场景中,使用相同算法在两类时域特征集构建的模型在测试集上的准确率数值极为接近,且均是在新型时域波形描述特征集上的构建模型性能优于在脉搏波原始采样点上构建的模型。
这也表明,\textbf{相较而言,新型时域波形描述特征集表征子痫前期的能力要强于脉搏波原始采样点。这也说明本研究提出的新型时域波形描述特征集设计合理,有应用于与本研究类似的基于脉搏波的其他相关研究的潜力}。

四、特征贡献与分布

本研究在新型时域波形描述特征集及脉搏波原始采样点上均按全部波形进行抽样后构建了子痫前期识别模型。特别地,当使用随机森林算法构建模型时,可以评估参与模型构建的特征的贡献度。
对比\autoref{tab:rf_dr_1}与\autoref{fig:rf_imp2},关于
新型时域波形描述特征集及脉搏波原始采样点在贡献度可以得到以下结论:

1、新型时域波形描述特征集与脉搏波原始采样点贡献度的最大值有一定差异,前者的最大值可以到达整体平均值的13.6倍以上,而后者仅为均值的6倍左右。这说明新型时域波形描述特征集中,与子痫前期相关
的特征更为明显。

2、对比两种脉搏波时域特征集上得到各特征的贡献度分布可以发现,两类时域特征集中具有子痫前期表征能力的特征出现的位置高度相似。贡献度高的特征多出现在下降支中,且集中在下降支的前端与尾端。这点在
\autoref{fig:rf_imp2}体现得更为明显,在特征贡献度分布中,脉搏波峰值之后及尾端位置出现了两个分布高峰。\textbf{这种现象也与脉搏波
下降支是血液回流过程的反应、下降支包含了更多的血液循环中细节信息的、子痫前期
会影响孕妇血液循环等事实相符合。说明这些位置的脉搏波形态可能是识别子痫前期的关键。}

\section{小结}
本小节通过新型时域波形描述特征集合与脉搏波原始采样点的基础上利用借助决策树、K近邻及随机森林等算法完成了不同分析场景下的子痫前期识别模型的构建与分析工作。
各模型在测试集上的性能表现证明生成的机器学习模型具有一定子痫前期识别能力的,也说明新型时域波形描述特征集及脉搏波原始采样点中
包含了能够表征子痫前期与正常妊娠孕妇形态差异的特征。而当横向对比三种机器学习算法的性能时,随机森林算法构建的子痫前期识别模型性能最好,K近邻次之,决策树最次。
而通过脉搏波新型时域波形描述特征集合构建的机器学习模型性能也要优于使用直接使用脉搏波原始采样点构建的模型。而从特征的角度而言,新型时域波形描述特征集合与脉搏波原始采样点中
具有子痫前期识别能力的特征出现的位置也高度相似,且与脉搏波的生理学基础有一定的对应关系。
