\chapter{基于多维度特征的PE识别结果与分析}
\section{引言}
此前的论文内容已经分别从机器学习的输入特征及机器学习的模型算法等方面进行铺垫介绍。本章将具体介绍利用前文提出的脉搏波的特征集合、使用多种机器学习算法
构建并评估子痫前期的一般识别模型。此外, 本章也对评估机器学习模型性能表现的一般指标进行了介绍。
\section{监督学习模型评估指标}
一、混淆矩阵

混淆矩阵(confusion matrix)是评估分类器分类效果优劣的常用工具\cite{Zhou2016,Aurélien2018}。其总体思路就是分别统计A类别实例被划分成B类别实例的数目。理论上混淆矩阵的行列没有上限,而在实际应用中,二分类任务的混淆矩阵是最常见的。
此时,将样例依据其真实所属类别与分类器预测类别进行组合可得到四种结果:真阳性(true positive,TP)、假阳性(false positive,FP)、真阴性(true negative,TN)及假阴性(false negative,TN),如\autoref{tab:cm}所示。此时显然有
$TP+FP+TN+FN=\text{样例总数}$。
\begin{table}[htbp]
      \centering
      \caption{\label{tab:cm}二分类任务的混淆矩阵}
      \begin{tabular}{ccc}
      \toprule
      \multicolumn{1}{c}{\multirow{2}[4]{*}{\textbf{真实情况}}} & \multicolumn{2}{c}{\textbf{预测结果}} \\
            \cmidrule{2-3}          & 阳性(1) & 阴性(0) \\
      \midrule
      阳性(1) & 真阳性(TP) & 假阴性(FN) \\
      阴性(0) & 假阳性(FP) & 真阴性(TN) \\
      \bottomrule
      \end{tabular}%
\end{table}%

为量化分类器的具体性能,人们在混淆矩阵的基础上衍生定义了一系列数字指标,包括查全率(recall)、查准率(precison)、准确率(accuracy)及特异性(specificity)等,如\autoref{equ:measures}所示。
\begin{equation}
      \label{equ:measures}
      \left \{
      \begin{aligned}
            Recall      &=\frac{TP}{TP+FN}         \\
            Precison    &=\frac{TP}{TP+FP}          \\
            Accuracy    &=\frac{TP+TN}{TP+FP+TN+FN} \\
            Specificity &=\frac{TN}{TN+FP}       \\
      \end{aligned}
      \right.
\end{equation}
其中,查全率亦称召回率、灵敏性(sensitivity)或真阳性率(true positive rate,TPR),查准率亦称精准率,特异性亦称真阴性率。查全率与查准率是应用的最广泛的两个指标\cite{Zhou2016,Aurélien2018}。
一般而言,查全率与查准率是对相互矛盾的度量指标,一个指标性能的提高意味着另一个指标性能的下降。通常只有在简单分类任务中,
才能同时获得较高的查准率与查全率。这称为精度-召回率权衡。为评估查全率与查准率均不相等的分类器性能,人们进一步定义了$F_1\text{分数}$,如\autoref{equ:f1}所示。
\begin{equation}
      \label{equ:f1}
      F_1=\frac{2}{\frac{1}{Precison}+\frac{1}{Recall}}=\frac{2\cdot Precison\cdot Recall}{Precison+Recall}=\frac{TP}{TP+\frac{FN+FP}{2}}
\end{equation}
$F_1\text{分数}$是召回率与精准率的谐波均值。召回率与精准率相近的分类器易获得更高的$F_1\text{分数}$。

在评估分类器性能时需要根据场景,从\autoref{equ:measures}与\autoref{equ:f1}中灵活选取恰当的评价指标。

二、ROC曲线、AUC与约登指数

受试者工作特征(Receiver Operating Characteristic,ROC)曲线是另一种常用于二分类问题的分析工具。ROC绘制的是真阳性率和假阳性率(false positive rate,FPR)之间的变化关系,其中
\begin{equation}
      \label{equ:fpr}
      FPR=\frac{TN}{TN+FP}=1-Specificity
\end{equation}
因此,ROC曲线也被称为灵敏度与1-特异性曲线。绘制曲线时,以分类器的预测结果对样例进行升序排列,依次将样本作为阳性进行预测,计算对应的TPR与FPR后,可得一坐标点$({FPR}_i,{TPR}_i)$,最后将所有坐标点连线即可,如\autoref{fig:roc}所示。
其中,虚线表示纯随机分类器的ROC曲线,理想性能的分类器应无限逼近左上角,即坐标点$(0,1)$。
\begin{figure}[htbp]
      \centering
      \includegraphics[width=.6\linewidth]{data_plan/roc}
      \caption[ROC曲线与AUC数值]{\label{fig:roc}ROC曲线与AUC数值。各分类器的AUC具体数值参见图例。}
\end{figure}

在衡量多个分类器性能优劣时,常将分类器对应的ROC曲线下面积作为判据,即为AUC(Area Under Curve)。纯随机分类器ROC的AUC数值为0.5,而理想分类器ROC的AUC数值为1,如\autoref{fig:roc}所示。

此外,约登指数(Youden Index)也是用来评价分类器效果的一个指标。若在评估分类器性能时,给予将分类器假阴性和假阳性以相同权重,即可应用约登指数
\begin{equation}
      \label{equ:yi}
      \begin{aligned}
            YI&=Sensitivity-(1-Specificity)\\
            &=Sensitivity+Specificity-1
      \end{aligned}
\end{equation}
一般认为,当YI取值最大时,此时对应的分类阈值为最佳阈值\cite{cwl}。
\section{监督学习算法的具体表现及分析}
\subsection{基于脉搏波时域特征集\Rnum{1}的结果及分析}
一、按照全部波形抽样

1. 模型初筛

为检验借助脉搏波时域特征集\Rnum{1}中的各项时域参数能否识别出孕妇是否患有子痫前期,本研究首先基于全部波形分层抽样的数据样本进行了监督学习算法的试探性研究\cite{scikit-learn}。
本研究共使用了随机梯度下降、决策树、K近邻、高斯朴素贝叶斯算法、逻辑回归、线性支持向量机、核支持向量机、C-支持向量机及多层感知机等9种基本分类算法进行模型探究。这些模型在进行初筛时,模型超参数均使用了默认设置\cite{scikit-learn}。
9种模型的初筛结果如\autoref{tab:model_screen}所示,其中训练集相关数据是对原始训练集数据经过5层交叉验证后得到的。

\begin{landscape}
      \zihao{-5}
      \begin{longtable}{m{3cm}<{\centering}m{1.7cm}<{\centering}m{2.3cm}<{\centering}m{1cm}<{\centering}m{1cm}<{\centering}m{1cm}<{\centering}m{1cm}<{\centering}m{1cm}<{\centering}m{2cm}<{\centering}m{1cm}<{\centering}m{1cm}<{\centering}m{1cm}<{\centering}m{1cm}<{\centering}}
            \caption{初筛结果}\\
            \label{tab:model_screen}\\
            \toprule
            &  & \multicolumn{6}{c}{\textbf{训练集(5层交叉验证)}} & \multicolumn{5}{c}{\textbf{验证集}}                                                                                                                                                                                                      \\
            \multirow{-2}{*}{\textbf{模型类型}} & \multirow{-2}{*}{\textbf{训练时间}} & \textbf{混淆矩阵} &  \textbf{精确率} &  \textbf{召回率} &  \textbf{F1值} &  \textbf{准确率} &  \textbf{AUC} &  \textbf{混淆矩阵} &  \textbf{精确率} &  \textbf{召回率} &  \textbf{F1值} &  \textbf{准确率}    \\
            \midrule
            \endfirsthead
            \caption[]{(续)}\\
            \midrule
            &  & \multicolumn{6}{c}{\textbf{训练集(5层交叉验证)}} & \multicolumn{5}{c}{\textbf{验证集}}                                                                                                                                                                                                      \\
            \multirow{-2}{*}{\textbf{模型类型}} & \multirow{-2}{*}{\textbf{训练时间}} & \textbf{混淆矩阵} &  \textbf{精确率} &  \textbf{召回率} &  \textbf{F1值} &  \textbf{准确率} &  \textbf{AUC} &  \textbf{混淆矩阵} &  \textbf{精确率} &  \textbf{召回率} &  \textbf{F1值} &  \textbf{准确率}    \\
            \midrule
            \endhead 
            \midrule
            \endfoot
            \bottomrule
            \endlastfoot
            随机梯度下降      &   6.17 s  &     $\left[ \begin{array}{cc} 2165 & 380 \\ 1164 & 2582 \end{array} \right]$ & 87.2\% & 68.9\% & 77.0\% & 75.5\% & 0.876 &
            $\left[ \begin{array}{cc} 328 & 308 \\ 22 & 915 \end{array} \right]$ & 74.8\% & 97.7\% & 84.7\% & 79.0\% \\
            决策树            &   5.24 s  &     $\left[ \begin{array}{cc} 2241 & 304 \\ 635 & 3111 \end{array} \right]$ & 91.1\% & 83.0\% & 86.9\% & 85.1\% & 0.907 &
            $\left[ \begin{array}{cc} 584 & 52 \\ 174 & 763 \end{array} \right]$ & 93.6\% & 81.4\% & 87.1\% & 85.6\% \\
            K近邻算法      &   3.08 s  &     $\left[ \begin{array}{cc} 2347 & 198 \\ 237 & 3509 \end{array} \right]$ & 94.7\% & 93.7\% & 94.2\% & 93.1\% & 0.974 &
            $\left[ \begin{array}{cc} 594 & 42 \\ 63 & 874 \end{array} \right]$ & 95.4\% & 93.3\% & 94.3\% & 93.3\% \\
            高斯朴素贝叶斯算法      &   1.22 s  &     $\left[ \begin{array}{cc} 2215 & 320 \\ 1354 & 2392 \end{array} \right]$ & 87.9\% & 63.9\% & 74.0\% & 73.2\% & 0.838 &
            $\left[ \begin{array}{cc} 569 & 67 \\ 328 & 609 \end{array} \right]$ & 90.1\% & 65.0\% & 75.5\% & 74.9\% \\
            逻辑回归算法      &   203.0 s  &     $\left[ \begin{array}{cc} 2174 & 371 \\ 347 & 3399 \end{array} \right]$ & 90.2\% & 90.7\% & 90.4\% & 88.6\% & 0.950 &
            $\left[ \begin{array}{cc} 573 & 63 \\ 66 & 871 \end{array} \right]$ & 93.3\% & 93.0\% & 93.1\% & 91.8\% \\
            线性支持向量机      &   47.22 s  &     $\left[ \begin{array}{cc} 1658 & 887 \\ 254 & 3492 \end{array} \right]$ & 79.7\% & 93.2\% & 86.0\% & 81.9\% & 0.917 &
            $\left[ \begin{array}{cc} 603 & 33 \\ 170 & 767 \end{array} \right]$ & 95.9\% & 81.9\% & 88.3\% & 87.1\% \\
            核支持向量机      &   60.28 s  &     $\left[ \begin{array}{cc} 1828 & 717 \\ 363 & 3383 \end{array} \right]$ & 82.5\% & 90.3\% & 86.2\% & 82.8\% & 0.916 &
            $\left[ \begin{array}{cc} 484 & 152 \\ 81 & 856 \end{array} \right]$ & 84.9\% & 91.4\% & 88.0\% & 85.2\% \\
            C-支持向量机      &   42.47 s  &     $\left[ \begin{array}{cc} 1914 & 631 \\ 354 & 3392 \end{array} \right]$ & 84.3\% & 90.5\% & 87.3\% & 84.3\% & 0.929 &
            $\left[ \begin{array}{cc} 510 & 126 \\ 86 & 851 \end{array} \right]$ & 87.1\% & 90.8\% & 88.9\% & 86.5\% \\
            多层感知机      &   26.8 s  &     $\left[ \begin{array}{cc} 1982 & 563 \\ 906 & 2840 \end{array} \right]$ & 83.5\% & 75.8\% & 79.5\% & 76.6\% & 0.905 &
            $\left[ \begin{array}{cc} 534 & 102 \\ 83 & 854 \end{array} \right]$ & 89.3\% & 91.1\% & 90.2\% & 88.2\% \\
      \end{longtable}
\end{landscape}

\begin{landscape}
      \zihao{-5}
      \begin{longtable}{m{3cm}<{\centering}m{5cm}<{\centering}m{1cm}<{\centering}m{2cm}<{\centering}m{1cm}<{\centering}m{3cm}<{\centering}m{1cm}<{\centering}m{2cm}<{\centering}m{1cm}<{\centering}}
            \caption{超参数优化前后模型性能对比}\\
            \label{tab:super_para}\\
            \toprule
            \multicolumn{1}{c}{\multirow{3}{*}{\textbf{模型类型}}} & \multicolumn{1}{c}{\multirow{3}{*}{\textbf{超参数组合值域}}}    & \multicolumn{3}{c}{\textbf{优化前}}   & \multicolumn{1}{c}{\multirow{3}{*}{\textbf{最优超参数}}}   & \multicolumn{3}{c}{\textbf{优化后}} \\
            \multicolumn{1}{c}{} & \multicolumn{1}{c}{}     & \textbf{训练集} & \multicolumn{2}{c}{\textbf{测试集}} & \multicolumn{1}{c}{}   & \textbf{训练集} & \multicolumn{2}{c}{\textbf{测试集}}     \\
            \multicolumn{1}{c}{} & \multicolumn{1}{c}{}     & \textbf{AUC} & \textbf{混淆矩阵}    & \textbf{准确率} & \multicolumn{1}{c}{}    & \textbf{AUC} & \multicolumn{1}{c}{\textbf{混淆矩阵}}     & \textbf{准确率} \\
            \midrule
            \endfirsthead
            \caption[]{(续)}\\
            \midrule
            \multicolumn{1}{c}{\multirow{3}{*}{\textbf{模型类型}}} & \multicolumn{1}{c}{\multirow{3}{*}{\textbf{超参数组合值域}}}    & \multicolumn{3}{c}{\textbf{优化前}}   & \multicolumn{1}{c}{\multirow{3}{*}{\textbf{最优超参数}}}   & \multicolumn{3}{c}{\textbf{优化后}} \\
            \multicolumn{1}{c}{} & \multicolumn{1}{c}{}     & \textbf{训练集} & \multicolumn{2}{c}{\textbf{测试集}} & \multicolumn{1}{c}{}   & \textbf{训练集} & \multicolumn{2}{c}{\textbf{测试集}}     \\
            \multicolumn{1}{c}{} & \multicolumn{1}{c}{}     & \textbf{AUC} & \textbf{混淆矩阵}    & \textbf{准确率} & \multicolumn{1}{c}{}    & \textbf{AUC} & \multicolumn{1}{c}{\textbf{混淆矩阵}}     & \textbf{准确率} \\
            \endhead 
            \midrule
            \endfoot
            \bottomrule
            \endlastfoot
            % \multirow{2}{*}{随机梯度下降}                            & \multirow{2}{*}{\begin{tabular}[c]{@{}l@{}}loss:{[}hinge, log\_loss,   \\ log, modified\_huber, \\ squared\_hinge, perceptron, \\ squared\_error,  huber,\\  epsilon\_insensitive, \\ squared\_epsilon\_insensitive{]},\\    penalty:{[}l2,l1,elasticnet{]},\\   alpha:{[}0.001,0.0001,0.00001{]}\end{tabular}} & \multirow{2}{*}{0.876} & \multirow{2}{*}{$\left[ \begin{array}{cc} 328 & 308 \\ 22 & 915 \end{array} \right]$ } & \multirow{2}{*}{79.0\%} & \multirow{2}{*}{\begin{tabular}[c]{@{}c@{}}alpha=0.001, \\ loss=squared\_hinge,   \\ penalty=elasticnet\end{tabular}} & \multirow{2}{*}{0.918} & \multirow{2}{*}{$\left[ \begin{array}{cc} 631 & 5 \\ 393 & 544 \end{array} \right]$} & \multirow{2}{*}{74.7\%} \\
            % &                      &      &       &    &    &    &      &     \\
            随机梯度下降    & \begin{tabular}[c]{@{}l@{}}loss:{[}\textbf{hinge}, log\_loss,   \\ log, modified\_huber, \\ squared\_hinge, perceptron, \\ squared\_error,  huber,\\  epsilon\_insensitive, \\ squared\_epsilon\_insensitive{]},\\    penalty:{[}\textbf{l2},l1,elasticnet{]},\\   alpha:{[}0.001,\textbf{0.0001},0.00001{]}\end{tabular} & 0.876        & $\left[ \begin{array}{cc} 328 & 308 \\ 22 & 915 \end{array} \right]$ & 79.0\%       & \begin{tabular}[c]{@{}l@{}}alpha=0.001, \\ loss=squared\_hinge,   \\ penalty=elasticnet\end{tabular} & 0.918        & $\left[ \begin{array}{cc} 631 & 5 \\ 393 & 544 \end{array} \right]$ & 74.7\%       \\
            高斯朴素贝叶斯算法   & var\_smoothing:{[}1e-5,1e-7,\textbf{1e-9},1e-11{]}       & 0.838        & $\left[ \begin{array}{cc} 569 & 67 \\ 328 & 609 \end{array} \right]$ & 74.9\%       & var\_smoothing=1e-7,                   & 0.842        & $\left[ \begin{array}{cc} 568 & 68 \\ 328 & 609 \end{array} \right]$ & 74.8\%      \\
            决策树          & \begin{tabular}[c]{@{}l@{}}criterion:{[}\textbf{gini},entropy,log\_loss{]},\\  splitter:{[}\textbf{best},random{]},\\     max\_depth:{[}\textbf{3},4,5{]},\\  max\_features:{[}sqrt,log2,\textbf{None}{]}\end{tabular}       & 0.907        & $\left[ \begin{array}{cc} 584 & 52 \\ 174 & 763 \end{array} \right]$ & 85.6\%       & \begin{tabular}[c]{@{}l@{}}criterion=entropy,\\  max\_depth=5, \\ max\_features=None\end{tabular}                             & 0.949        & $\left[ \begin{array}{cc} 621 & 15 \\ 159 & 778 \end{array} \right]$ & 88.9\%       \\
            K近邻算法           & \begin{tabular}[c]{@{}l@{}}n\_neighbors:{[}3,\textbf{5},7,9{]},\\    weights:{[}\textbf{uniform},distance{]}\end{tabular}     & 0.974        & $\left[ \begin{array}{cc} 594 & 42 \\ 63 & 874 \end{array} \right]$    & 93.3\%       & \begin{tabular}[c]{@{}l@{}}n\_neighbors=9,\\  weights=distance\end{tabular}      & 0.978        & $\left[ \begin{array}{cc} 598 & 38 \\ 67 & 870 \end{array} \right]$ & 93.3\%       \\
      \end{longtable}
\end{landscape}

从\autoref{tab:model_screen}中结果可以得到以下结论:

\Rnum{1} 在测试集上,除高斯朴素贝叶斯算法外剩余8种模型的AUC数值均在0.850以上,其中,K近邻算法的AUC数值更是高达0.974。
从各模型在训练集上得到的混淆矩阵来看,K近邻算法与逻辑回归算法在精度-召回率权衡上表现最好,精确率、召回率及F1数值均在90.0\%以上。决策树算法与三种支持向量机算法在精确率与召回率可以达到90.0\%+80.0\%
(或80.0\%+90.0\%)以上,4种算法模型的F1值也均在86.0\%以上。而剩下的随机梯度算法、高斯朴素贝叶斯算法与多层感知机算法在这些数值上表现较差。
另外,在各模型的训练时间方面,高斯朴素贝叶斯算法训练所需时间最短,仅需1.22s,而多层感知机、支持向量机模型所需时间较长、逻辑回归算法训练时间最长为203.0s。这些数值也与各算法模型的
复杂度对应,符合预期。

\Rnum{2} 在验证集上,随机梯度算法与高斯朴素贝叶斯算法的表现最差,出现精确率或召回率数值小于75\%的情况。剩余7种算法均在验证集上有较好的泛化能力,决策树算法、逻辑回归算法、三种支持向量机算法及多层感知机算法性能接近,精确率与召回率可以达到90.0\%+80.0\%
(或80.0\%+90.0\%),F1值也均在87.0\%以上。而K近邻算法与逻辑回归算法表现最为优秀,精确率、召回率与F1值三者数值更是均在93.0\%以上。

综上,上述结果初步说明了本研究提出的脉搏波时域特征集\Rnum{1}在子痫前期的识别问题上具有较强的表征能力,可以构建出具有较好泛化性能的子痫前期识别分类模型。
若不考虑模型训练所需时间,K近邻算法与逻辑回归算法的表现最为出色,随机梯度算法与高斯朴素贝叶斯算法表现最差。


2. 超参数优化

上小节已经在脉搏波时域特征集\Rnum{1}上初步得到了各机器学习模型的表现。由于上述模型的建立均采用默认超参数,上小节的结论并不完全严谨。
一般而言,超参数的调整与优化会使模型的性能得到提升。因此,本小节主要对模型超参数的设置进行了研究。
在综合考虑模型的训练时间及初筛时的性能表现,本研究从上述9种单一分类中选取了随机梯度下降算法、高斯朴素贝叶斯算法、决策树算法与K近邻算法等四种模型进行了超参数调优工作。
其中,前两类在初筛阶段表现较差的算法主要探索能否通过超参数的调整提升性能,后两种算法的超参数调整则是为了进一步的考察性能。
各模型超参数的最优数值组合通过网格搜索的方法来进行探索,而优化的评价标准是新生成模型在训练集上的AUC面积数值大小。

\autoref{tab:super_para}展示了超参数优化的过程与结果。其中,表格中超参数值域一栏给出了网格搜索时使用的具体超参数及其值域范围,加粗数值为默认超参数数值;优化前后,模型在训练集上的AUC数值均是在
进行了5层交叉验证后取得的;最优超参数一栏则是在各模型在训练集AUC取最大值时使用的超参数组合,此时模型在测试集上的混淆矩阵与准确率也在表格中一并给出。

从\autoref{tab:super_para}中的具体数值不难发现,四种模型的在超参数调整前后的AUC数值均有所提高。但只有决策树算法与K近邻算法
延续了之前的优秀表现,甚至整体性能有微幅上升。但对随机梯度下降算法与高斯朴素贝叶斯算法而言,超参数的调整并没有明显的提升模型性能。
因此,本小节的实验结果进一步佐证了上小节初筛时得到的各项结论。


3. 随机森林算法与特征降维

此前初筛过程中使用的均是单机器学习模型,本小节则使用了集成学习中的随机森林算法在脉搏波时域特征集\Rnum{1}进行了模型训练。
如\autoref{tab:rf_dr_2}第一行所示,使用默认超参数生成的随机森林模型在训练集与测试集上均有着优秀的表现,其在训练集上的AUC面积为0.990,在训练集与测试集上的准确率更是分别达到了95.2\%与97.0\%。
这些数值均高于之前的单机器模型中性能最好的K近邻算法与逻辑回归算法,充分体现了集成学习的优势。

\begin{figure}[htbp]
      \centering
      \includegraphics[width=\linewidth]{results/dt_clf}
      \caption{\label{fig:dt_clf}在训练集上的构建的一棵决策树示意}
\end{figure}

另一方面,本章之前的内容已经阐述过随机森林算法可以用来衡量原始数据样本各属性的贡献度,即随机森林算法可以用作特征降维处理。在使用脉搏波时域特征集\Rnum{1}的286个时域特征完成随机森林的构建之后,可以从中得到各特征对最终模型的贡献度的数值,
结果如\autoref{tab:rf_dr_1}所示。脉搏波时域特征集\Rnum{1}共有的286个特征中,这些特征的平均贡献度为0.34\%,贡献度最高的特征$CVALF\_9$的贡献度为平均值的13.5倍。
若以特征$CVALF\_9$为基准,可以得到所有特征对随机森林模型的相对贡献度如\autoref{fig:rf_importance_pulse}所示。

\begin{center}
      \zihao{5}
      \begin{longtable}{m{2cm}<{\centering}m{2cm}<{\centering}m{2cm}<{\centering}m{2cm}<{\centering}m{2cm}<{\centering}m{2cm}<{\centering}}
            \caption[参与构建随机森林的特征贡献度]{参与构建随机森林的特征贡献度。仅列举了对随机森林累计贡献度达51.9\%的前36个特征。}\\
            \label{tab:rf_dr_1}\\
            \toprule
            \textbf{特征名}&\textbf{贡献度}&\textbf{特征名}&\textbf{贡献度}&\textbf{特征名}&\textbf{贡献度}\\
            \midrule
            \endfirsthead
            \caption[]{(续)}\\
            \midrule
            \textbf{特征名}&\textbf{贡献度}&\textbf{特征名}&\textbf{贡献度}&\textbf{特征名}&\textbf{贡献度}\\
            \midrule
            \endhead 
            \midrule
            \endfoot
            \bottomrule
            \endlastfoot
            CVALF\_9                         & 4.6\%                            & LVRF\_9                          & 3.4\%                            & SVD\_10                          & 3.0\%                            \\
            SVAF\_10                         & 2.5\%                            & LVALF\_7                         & 2.4\%                            & SVAT\_10                         & 2.3\%                            \\
            CVRF\_11                         & 2.1\%                            & SVAR\_10                         & 1.8\%                            & STDZ\_3                          & 1.6\%                            \\
            LVRF\_8                          & 1.6\%                            & LVALF\_6                         & 1.6\%                            & CVD\_11                          & 1.6\%                            \\
            STDZ\_1                          & 1.5\%                            & CVALF\_8                         & 1.5\%                            & SVD\_9                           & 1.4\%                            \\
            SVAAR\_10                        & 1.4\%                            & CVALF\_1                         & 1.4\%                            & SVAAR\_8                         & 1.3\%                            \\
            SVD\_8                           & 1.1\%                            & SVAR\_9                          & 1.0\%                            & SVAF\_9                          & 1.0\%                            \\
            CVAAF\_1                         & 1.0\%                            & LVLF\_8                          & 0.9\%                            & SVAT\_8                          & 0.9\%                            \\
            CVRF\_1                          & 0.9\%                            & CVAAF\_2                         & 0.9\%                            & SVAR\_8                          & 0.8\%                            \\
            SVAF\_2                          & 0.8\%                            & CVRR\_11                         & 0.8\%                            & LVRF\_1                          & 0.7\%                            \\
            SVSR\_7                          & 0.7\%                            & CVRF\_3                          & 0.7\%                            & CVALR\_10                        & 0.7\%                            \\
            CVALF\_7                         & 0.7\%                            & CVAAR\_5                         & 0.7\%                            & LVRF\_2                          & 0.6\%                           
      \end{longtable}
\end{center}

\begin{figure}[htbp]
      \centering
      \includegraphics[width=0.6\linewidth]{results/rf_ip_pulse_0.52}
      \caption[各特征对随机森林的相对贡献度]{\label{fig:rf_importance_pulse}各特征对随机森林的相对贡献度。仅列举了对随机森林累计贡献度达51.9\%的前36个特征。}
\end{figure}

\begin{landscape}
      \zihao{-5}
      \begin{longtable}{m{1.8cm}<{\centering}m{1.8cm}<{\centering}m{1.8cm}<{\centering}m{1.8cm}<{\centering}m{1.8cm}<{\centering}m{2cm}<{\centering}m{1cm}<{\centering}m{2cm}<{\centering}m{1cm}<{\centering}m{1cm}<{\centering}m{1cm}<{\centering}m{1cm}<{\centering}}
            \caption{随机森林对脉搏波特征降维效果}\\
            \label{tab:rf_dr_2}\\
            \toprule
            \multicolumn{3}{c}{\textbf{随机森林特征输入}}              & \multicolumn{2}{c}{\textbf{训练时间}} & \multicolumn{2}{c}{\textbf{训练集}} & \multicolumn{5}{c}{\textbf{测试集}}                                          \\
            \textbf{贡献度比例} & \textbf{特征数量} & \textbf{数量比例} & \textbf{训练时间}  & \textbf{时间比例}  & \textbf{混淆矩阵}   & \textbf{AUC}   & \textbf{混淆矩阵} & \textbf{精确率} & \textbf{召回率} & \textbf{F1值} & \textbf{准确率} \\
            \midrule
            \endfirsthead
            \caption[]{(续)}\\
            \midrule
            \multicolumn{3}{c}{\textbf{随机森林特征输入}}              & \multicolumn{2}{c}{\textbf{训练时间}} & \multicolumn{2}{c}{\textbf{训练集}} & \multicolumn{5}{c}{\textbf{测试集}}                                          \\
            \textbf{贡献度比例} & \textbf{特征数量} & \textbf{数量比例} & \textbf{训练时间}  & \textbf{时间比例}  & \textbf{混淆矩阵}   & \textbf{AUC}   & \textbf{混淆矩阵} & \textbf{精确率} & \textbf{召回率} & \textbf{F1值} & \textbf{准确率} \\
            \midrule
            \endhead 
            \midrule
            \endfoot
            \bottomrule
            \endlastfoot
            100.0\%        & 286           & 100.0\%       & 41.30          & 100.0\%          & $\left[ \begin{array}{cc} 2404 & 141 \\ 160 & 3586 \end{array} \right]$  & 0.990        & $\left[ \begin{array}{cc} 617 & 19 \\ 28 & 909 \end{array} \right]$   & 98.0\%       & 97.0\%       & 97.5\%       & 97.0\%       \\
            90.0\%         & 162           & 56.6\%        & 38.11          & 92.3\%           & $\left[ \begin{array}{cc} 2421 & 124 \\ 134 & 3612 \end{array} \right]$  & 0.991        & $\left[ \begin{array}{cc} 619 & 17 \\ 25 & 912 \end{array} \right]$   & 98.1\%       & 97.3\%       & 97.8\%       & 97.3\%       \\
            80.0\%         & 108           & 37.8\%        & 29.33          & 71.0\%           & $\left[ \begin{array}{cc} 2431 & 114 \\ 130 & 3616 \end{array} \right]$  & 0.993        & $\left[ \begin{array}{cc} 615 & 21 \\ 26 & 911 \end{array} \right]$   & 97.7\%       & 97.2\%       & 97.5\%       & 97.0\%       \\
            70.0\%         & 74            & 25.9\%        & 23.67          & 57.3\%           & $\left[ \begin{array}{cc} 2431 & 114 \\ 125 & 3621 \end{array} \right]$  & 0.993        & $\left[ \begin{array}{cc} 618 & 18 \\ 23 & 914 \end{array} \right]$   & 98.1\%       & 97.5\%       & 97.8\%       & 97.4\%       \\
            60.0\%         & 51            & 17.8\%        & 20.91          & 50.6\%           & $\left[ \begin{array}{cc} 2442 & 103 \\ 115 & 3631 \end{array} \right]$  & 0.994        & $\left[ \begin{array}{cc} 620 & 16 \\ 27 & 910 \end{array} \right]$   & 98.3\%       & 97.1\%       & 97.7\%       & 97.3\%       \\
            50.0\%         & 34            & 11.9\%        & 15.17          & 36.7\%           & $\left[ \begin{array}{cc} 2433 & 112 \\ 146 & 3600 \end{array} \right]$  & 0.993        & $\left[ \begin{array}{cc} 616 & 20 \\ 28 & 909 \end{array} \right]$   & 97.8\%       & 97.0\%       & 97.4\%       & 97.0\%       \\
            40.0\%         & 21            & 7.3\%         & 11.90          & 28.8\%           & $\left[ \begin{array}{cc} 2432 & 113 \\ 145 & 3601 \end{array} \right]$  & 0.993        & $\left[ \begin{array}{cc} 615 & 21 \\ 30 & 907 \end{array} \right]$   & 97.7\%       & 96.8\%       & 97.3\%       & 96.8\%       \\
            30.0\%         & 13            & 4.5\%         & 9.79           & 23.7\%           & $\left[ \begin{array}{cc} 2422 & 123 \\ 200 & 3546 \end{array} \right]$  & 0.987        & $\left[ \begin{array}{cc} 612 & 24 \\ 34 & 903 \end{array} \right]$   & 97.4\%       & 96.4\%       & 96.9\%       & 96.3\%       \\
            20.0\%         & 7             & 2.4\%         & 7.82           & 18.9\%           & $\left[ \begin{array}{cc} 2277 & 268 \\ 382 & 3364 \end{array} \right]$  & 0.961        & $\left[ \begin{array}{cc} 587 & 49 \\ 92 & 845 \end{array} \right]$   & 94.5\%       & 90.1\%       & 92.3\%       & 91.0\%       \\
            10.0\%         & 3             & 1.0\%         & 5.31           & 12.9\%           & $\left[ \begin{array}{cc} 2065 & 480 \\ 484 & 3262 \end{array} \right]$  & 0.931        & $\left[ \begin{array}{cc} 539 & 97 \\ 124 & 813 \end{array} \right]$  & 89.3\%       & 86.8\%       & 88.0\%       & 86.0\%      
      \end{longtable}
\end{landscape}

\begin{landscape}
      \zihao{-5}
      \begin{longtable}{m{3cm}<{\centering}m{1.7cm}<{\centering}m{2.3cm}<{\centering}m{1cm}<{\centering}m{1cm}<{\centering}m{1cm}<{\centering}m{1cm}<{\centering}m{1cm}<{\centering}m{2cm}<{\centering}m{1cm}<{\centering}m{1cm}<{\centering}m{1cm}<{\centering}m{1cm}<{\centering}}
            \caption{几种机器学习模型在被试人员分层抽样的数据集上的表现}\\
            \label{tab:model_screen2}\\
            \toprule
            &  & \multicolumn{6}{c}{\textbf{训练集}} & \multicolumn{5}{c}{\textbf{验证集}}                                                                                                                                                                                                      \\
            \multirow{-2}{*}{\textbf{模型类型}} & \multirow{-2}{*}{\textbf{训练时间}} & \textbf{混淆矩阵} &  \textbf{精确率} &  \textbf{召回率} &  \textbf{F1值} &  \textbf{准确率} &  \textbf{AUC} &  \textbf{混淆矩阵} &  \textbf{精确率} &  \textbf{召回率} &  \textbf{F1值} &  \textbf{准确率}    \\
            \midrule
            \endfirsthead
            \caption[]{(续)}\\
            \midrule
            &  & \multicolumn{6}{c}{\textbf{训练集}} & \multicolumn{5}{c}{\textbf{验证集}}                                                                                                                                                                                                      \\
            \multirow{-2}{*}{\textbf{模型类型}} & \multirow{-2}{*}{\textbf{训练时间}} & \textbf{混淆矩阵} &  \textbf{精确率} &  \textbf{召回率} &  \textbf{F1值} &  \textbf{准确率} &  \textbf{AUC} &  \textbf{混淆矩阵} &  \textbf{精确率} &  \textbf{召回率} &  \textbf{F1值} &  \textbf{准确率}    \\
            \midrule
            \endhead 
            \midrule
            \endfoot
            \bottomrule
            \endlastfoot
            K近邻算法      &   4.08 s  &     $\left[ \begin{array}{cc} 2250 & 384 \\ 888 & 2815 \end{array} \right]$ & 88.0\% & 76.0\% &81.6\% & 80.0\% & 0.870 &
            $\left[ \begin{array}{cc} 357 & 190 \\ 141 & 839 \end{array} \right]$ & 81.5\% & 85.6\% & 83.5\% & 78.3\% \\
            决策树算法      &   1.44 s  &     $\left[ \begin{array}{cc} 2062 & 572 \\ 706 & 2997 \end{array} \right]$ & 84.0\% & 80.9\% & 82.4\% & 79.8\% & 0.862 &
            $\left[ \begin{array}{cc} 168 & 379 \\ 136 & 844 \end{array} \right]$ & 69.0\% & 86.1\% & 76.6\% & 66.3\% \\
            随机森林算法      &   50.03 s  &     $\left[ \begin{array}{cc} 2326 & 308 \\ 718 & 2985 \end{array} \right]$ & 90.6\% & 80.6\% & 85.3\% & 83.8\% & 0.929 &
            $\left[ \begin{array}{cc} 317 & 230 \\ 89 & 891 \end{array} \right]$ & 79.5\% & 90.9\% & 84.8\% & 79.1\% \\
      \end{longtable}
\end{landscape}

\begin{landscape}
      \zihao{-5}
      \begin{longtable}{m{2cm}<{\centering}m{2cm}<{\centering}m{2cm}<{\centering}m{2cm}<{\centering}m{2cm}<{\centering}m{2cm}<{\centering}m{2cm}<{\centering}m{2cm}<{\centering}m{2cm}<{\centering}}
            \caption{几种机器学习模型按被试统计后的性能表现}\\
            \label{tab:model_detail}\\
            \toprule
            \multirow{2}{*}{被试孕妇} & \multirow{2}{*}{波形总数} & \multicolumn{2}{c}{K近邻算法} & \multicolumn{2}{c}{决策树} & \multicolumn{2}{c}{随机森林} & \multirow{2}{*}{\begin{tabular}[c]{@{}l@{}}真实子痫前\\ 期患病状态\end{tabular}} \\
                        &                       & 预测阳性数目     & 预测比例       & 预测阳性数目     & 预测比例       & 预测阳性数目     & 预测比例        &                                                                        \\
            \midrule
            \endfirsthead
            \caption[]{(续)}\\
            \midrule
            \multirow{2}{*}{被试孕妇} & \multirow{2}{*}{波形总数} & \multicolumn{2}{c}{K近邻算法} & \multicolumn{2}{c}{决策树} & \multicolumn{2}{c}{随机森林} & \multirow{2}{*}{\begin{tabular}[c]{@{}l@{}}真实子痫前\\ 期患病状态\end{tabular}} \\
                        &                       & 预测阳性数目     & 预测比例       & 预测阳性数目     & 预测比例       & 预测阳性数目     & 预测比例        &                                                                        \\
            \midrule
            \endhead 
            \midrule
            \endfoot
            \bottomrule
            \endlastfoot
            cmf                   & 88                    & 54         & 61.4\%     & 85         & 96.6\%     & 53         & 60.2\%      & 0                                                                      \\
            lxx                   & 63                    & 32         & 50.8\%     & 53         & 84.1\%     & 34         & 54.0\%      & 0                                                                      \\
            shs                   & 112                   & 37         & 33.0\%     & 105        & 93.8\%     & 55         & 49.1\%      & 0                                                                      \\
            sxh                   & 95                    & 27         & 28.4\%     & 64         & 67.4\%     & 21         & 22.1\%      & 0                                                                      \\
            wdq                   & 36                    & 0          & 0.0\%      & 0          & 0.0\%      & 0          & 0.0\%       & 0                                                                      \\
            wsj                   & 78                    & 0          & 0.0\%      & 2          & 2.6\%      & 0          & 0.0\%       & 0                                                                      \\
            ygy                   & 75                    & 40         & 53.3\%     & 70         & 93.3\%     & 67         & 89.3\%      & 0                                                                      \\
            gmn                   & 139                   & 106        & 76.3\%     & 135        & 97.1\%     & 109        & 78.4\%      & 1                                                                      \\
            ty                    & 98                    & 97         & 99.0\%     & 97         & 99.0\%     & 97         & 99.0\%      & 1                                                                      \\
            wjh                   & 86                    & 86         & 100.0\%    & 86         & 100.0\%    & 86         & 100.0\%     & 1                                                                      \\
            xjf                   & 106                   & 23         & 21.7\%     & 59         & 55.7\%     & 87         & 82.1\%      & 1                                                                      \\
            ywy                   & 111                   & 110        & 99.1\%     & 111        & 100.0\%    & 111        & 100.0\%     & 1                                                                      \\
            yxl                   & 110                   & 110        & 100.0\%    & 108        & 98.2\%     & 110        & 100.0\%     & 1                                                                      \\
            zdq                   & 89                    & 81         & 91.0\%     & 84         & 94.4\%     & 82         & 92.1\%      & 1                                                                      \\
            zl                    & 152                   & 137        & 90.1\%     & 75         & 49.3\%     & 120        & 78.9\%      & 1                                                                      \\
            zyy                   & 89                    & 89         & 100.0\%    & 89         & 100.0\%    & 89         & 100.0\%     & 1                                                                       \\    
      \end{longtable}
\end{landscape}

在获取了所有特征对随机森林的贡献度后,此时即可按照贡献度按从高到低的顺序进行特征筛选,新筛选出的特征也被用于建立随机模型用于识别子痫,并用于评价筛选的效果。\autoref{tab:rf_dr_2}展示了这一过程,
其中筛选按所有特征对初始随机森林贡献度的百分比例以10\%为梯度进行,每栏给出了使用筛选出的特征上建立的随机森林的效果。

从\autoref{tab:rf_dr_2}可以发现以下现象。首先,随着特征数量的减少,随机森林模型的训练速度大大加快。其次,随着特征数量的减少,新生成的随机森林模型在训练集与测试集上均出现了性能小幅提高再下降的整体趋势,
模型最佳性能出现在贡献度60\%-70\%之间。这说明筛选刚开始进行时排除了大量无关特征,从而使模型性能得到了提升;而随着筛选的进行,对模型生成有贡献度的特征数量的也在减少,导致随机森林的性能降低。
第三,脉搏波时域特征集\Rnum{1}的286个原始特征中具有子痫前期表征能力的特征较少,存在大量不具任何子痫前期表征能力的冗余特征。在保留原始特征集90\%的贡献度的条件下,就可以
在几乎不损失模型预测能力的条件下排除约43.4\%的无关特征。甚至在仅保留原始特征集30\%的贡献度的13个特征上建立的随机森林模型也有着接近原始随机森林模型的性能(准确率96.3\% VS 97.0\%)。

4. 综合分析

从特征的角度而言,结合\autoref{tab:rf_dr_2}中结果,脉搏波时域特征集\Rnum{1}包含了一定的具有子痫前期表征能力的脉搏波时域形态特征。而研究\autoref{tab:rf_dr_1}中各特征具体的贡献度可以发现,
在这些有效特征中,与下降支相关的特征要远多于上升支的特征。这也与脉搏波下降支是血液回流过程的反应、下降支包含了更多的血液循环中细节信息的事实相符合。
从\autoref{tab:rf_dr_1}的指标种类来看,左视类指标相较中视类指标与分层类指标也更少,这也可能与左视类指标对下降支的表征效果不及另外两类有关。
此外,\autoref{tab:rf_dr_1}中的特征的下标也集中在前端(1-3)与尾端(8-10),这说明脉搏波下降支的前端
与尾端可能包含了更多与子痫前期相关的信息,如下降支在前端是否快速下降,在尾端是否平缓等。

本小节按照全部波形进行子痫前期识别模型的研究过程实际上基于子痫前期导致的病生理变化可以在单个波形上得到体现的假设。即单个波形也包含了
识别子痫前期的全部信息。本小节的研究结果为这种假设提供了一定的支撑。在脉搏波时域特征集\Rnum{1}上建立的多种机器学习模型的结果均能取得较好的
识别效果。但从另一个角度而言,由于本小节使用的训练集与测试集是基本脉搏波波形划分的,同一被试的不同脉搏波波形会分别被划分至训练集与测试集。若这些波形高度相似,
就可能导致基于脉搏波波形的各类机器学习模型具有较高的识别效果。

二、按照被试人员分层抽样

为保证在进行机器学习模型训练时同一被试的不同脉搏波数据只出现在训练集或测试集,本小节按照被试人员进行了分层抽样。通过这种方式训练得到的模型,在测试集上的表现可以用来
更加客观评估模型的泛化能力。本小节将63名被试孕妇的全部脉搏波波形被划分为训练集,余下16名被试的全波波形被划分为测试集。使用了在上小节中表现较好的K近邻、决策树及随机森林等三种算法在新划分的数据集上进行子痫前期识别模型的研究。
这三种模型的具体表现如\autoref{tab:model_screen2}所示,其中,训练集上的AUC数值是在进行了 5 层交叉验证后取得的。

横向对比\autoref{tab:model_screen}、\autoref{tab:rf_dr_2}及\autoref{tab:model_screen2}中结果可以发现,在将原始数据按被试人员分层抽样之后,K近邻、决策树及随机森林等三种算法在训练集与测试集上的均出现了
较为明显的性能下降,在测试集上性能下降得尤为明显。而单就\autoref{tab:model_screen2}而言,随机森林算法与K近邻算法的性能接近,在测试集上F1值可达到83\%以上,准确率也在78\%以上;决策树算法表现最差F1值仅有76.6\%,准确率也仅有66.3\%。
但以上数据仍旧是按照波形进行统计,并没有体现出新的分层抽样的结果。在将\autoref{tab:model_screen2}各模型对测试集上的各波形的预测结果按被试进行统计的结果如\autoref{tab:model_detail}所示。
\autoref{tab:model_detail}给出了测试集上被试人员所对应的脉搏波波形总数与其子痫前期患病状态,对三种算法分别统计了预测为子痫前期阳性的脉搏波波形数目,而预测比例则为相应模型下预测为阳性的脉搏波数目与该被人脉搏波波形总数之比。

\begin{figure}[htbp]
      \centering
      \includegraphics[width=.55\linewidth]{results/detail}
      \caption[几种机器学习的detail]{\label{fig:model_detail}几种机器学习的detail}
\end{figure}
\begin{figure}[htbp]
      \centering
      \includegraphics[width=.6\linewidth]{results/roc}
      \caption[几种机器学习ROC曲线]{\label{fig:model_roc}几种机器学习ROC曲线}
\end{figure}

若将\autoref{tab:model_detail}中三种模型的预测比例作为该模型表征子痫前期患病状态的输出,则可以得到\autoref{fig:model_detail}所示的散点图。此时,对子痫前期的识别分析可以转换为寻找
能将\autoref{fig:model_detail}进行最佳分割的阈值。遍历各模型的预测比例数值并将该数值作为分割阈值,可以得到在该数值下的混淆矩阵,并进一步得到该数值对应的敏感性与特异性。最终,三种模型的预测比例
所对应的ROC曲线如\autoref{fig:model_roc}所示。其中,决策树、K近邻及随机森林三种模型对应的AUC数值分别为0.825、0.921及0.952。同时,借助约登指数可以确定三种模型的最佳分割阈值分别为0.969、0.907与0.688,各模型在最佳阈值下的
混淆矩阵如\autoref{tab:cm_on_best}所示。三种模型的整体识别准确率分别为81.3\%、81.3\%与93.8\%。

\begin{table}[htbp]
      \zihao{5}
      \centering
      \caption{\label{tab:cm_on_best}三种模型在最佳分割阈值下的混淆矩阵}
      \begin{tabular}{ccc}
      \toprule
      \textbf{决策树算法}&\textbf{K近邻算法}&\textbf{随机森林算法}\\
      \midrule
      $\left[ \begin{array}{cc} 6 & 3 \\ 0 & 7 \end{array} \right]$ & $\left[ \begin{array}{cc} 6 & 3 \\ 0 & 7 \end{array} \right]$ & $\left[ \begin{array}{cc} 9 & 0 \\ 1 & 6 \end{array} \right]$ \\
      \bottomrule
      \end{tabular}%
\end{table}%

在上述三种分类算法中,随机森林模型无疑有着最佳的分类效果。从AUC数值而言,随机森林模型的AUC数值最大。从\autoref{tab:cm_on_best}来看,随机森林算法识别的总体准确率高于决策树与K近邻算法;
识别为假阴性的数目(0)也少于其他两种模型(3与3)。从最佳分割阈值来看,决策树与K近邻算法的预测比例分割阈值过于接近100\%,有泛化能力不足的潜在风险,而随机森林模型的分割阈值(0.688)则相对适中。

综上,本小节按照被试孕妇进行子痫前期识别模型的研究过程实际上基于子痫前期导致的病生理变化可以通过具体的脉搏波波形上,并可以通过评估该被试全部
脉搏波波形的形态特征进行群体决策的假设。本小节通过决策树算法、K近邻算法与随机森林算法分别构建了子痫前期的识别模型,并得到了各模型被测试集被试孕妇的预测比例数值。
借助AUC与ROC分析,可以分别达到81.3\%、81.3\%与93.8\%的整体准确率。

\subsection{基于脉搏波时域特征集\Rnum{2}的结果及分析}

脉搏波时域特征集\Rnum{2}实质上是将脉搏波波形各原始采样值看成是相应的输入特征进行处理。在采样率相同的情况下,脉搏波波形时长的差异会导致波形所对应的“特征”维数的差异。
参考心电分析中心率图的定义,将同一被试的所有脉搏波波形按起点进行对齐后描述在同一张图里则可以得到该被试的脉率图,如\autoref{fig:no_pe}所示。
\begin{figure}[htbp]
      \centering
      \subfigure[\label{fig:pe_hdy}患有子痫前期的被试hdy的脉率图]{
      \includegraphics[width=7.5cm]{results/hdy in group PE}
      }
      \quad
      \subfigure[\label{fig:pe_wjh}患有子痫前期的被试wjh的脉率图]{
      \includegraphics[width=7.5cm]{results/wjh in group PE}
      }
      \quad
      \subfigure[\label{fig:no_cmf}正常被试cmf的脉率图]{
      \includegraphics[width=7.5cm]{results/cmf in group No}
      }
      \quad
      \subfigure[\label{fig:no_lh}正常被试lh的脉率图]{
      \includegraphics[width=7.5cm]{results/lh in group No}
      }
      \caption{\label{fig:no_pe}被试孕妇的原始脉搏波脉率图对照}
\end{figure}

从\autoref{fig:no_pe}可以看到,在本研究采集得到的被试数据中,正常组与实验组在脉搏波波形形态及脉搏波整体分布上均有一定的差异。同时,可以看到不同脉搏波波形的时长差异较为明显。因此,本研究采取了两种
处理策略:一是在原始采样值的尾端进行补零,将所有脉搏波波形的采样点数均调整为120(该值已经可以确保涵盖本次研究采集得到的所有脉搏波波形时长的最大值);二是
对原始采样值进行重采样,使采样点数均调整至100(该值可以保证脉搏波波形的描述的分辨率与精度),如\autoref{fig:no_pe2}所示。
在此基础上,本小节也进行了与上小节类似的研究工作,分别按波形与被试对脉搏波波形进行了处理与建模。

\begin{figure}[htbp]
      \centering
      \subfigure[\label{fig:pe_hdy2}患有子痫前期的被试hdy的脉率图]{
      \includegraphics[width=7.5cm]{results/contrast/hdy in group PE}
      }
      \quad
      \subfigure[\label{fig:pe_wjh2}患有子痫前期的被试wjh的脉率图]{
      \includegraphics[width=7.5cm]{results/contrast/wjh in group PE}
      }
      \quad
      \subfigure[\label{fig:no_cmf2}正常被试cmf的脉率图]{
      \includegraphics[width=7.5cm]{results/contrast/cmf in group No}
      }
      \quad
      \subfigure[\label{fig:no_lh2}正常被试lh的脉率图]{
      \includegraphics[width=7.5cm]{results/contrast/lh in group No}
      }
      \caption{\label{fig:no_pe2}重采样后的被试孕妇的脉搏波脉率图对照}
\end{figure}

一、按照全波波形抽样

1. 模型初筛

\begin{landscape}
      \zihao{-5}
      \begin{longtable}{m{1.5cm}<{\centering}m{1.5cm}<{\centering}m{1.5cm}<{\centering}m{2cm}<{\centering}m{1cm}<{\centering}m{1cm}<{\centering}m{1cm}<{\centering}m{1cm}<{\centering}m{1cm}<{\centering}m{2cm}<{\centering}m{1cm}<{\centering}m{1cm}<{\centering}m{1cm}<{\centering}m{1cm}<{\centering}}
            \caption{初筛结果}\\
            \label{tab:model_screen3}\\
            \toprule
                  & \multicolumn{1}{c}{}   & \multicolumn{1}{c}{}  & \multicolumn{6}{c}{\textbf{训练集(5层交叉验证)}}   & \multicolumn{5}{c}{\textbf{验证集}}     \\
            \multirow{-2}{*}{\textbf{处理方式}}  & \multicolumn{1}{c}{\multirow{-2}{*}{\textbf{模型类型}}} & \multicolumn{1}{c}{\multirow{-2}{*}{\textbf{训练时间}}} & \textbf{混淆矩阵} & \textbf{精确率} & \textbf{召回率}& \textbf{F1值} & \textbf{准确率}& \textbf{AUC} & \textbf{混淆矩阵}& \textbf{精确率} & \textbf{召回率} & \textbf{F1值}& \textbf{准确率} \\
            \midrule
            \endfirsthead
            \caption[]{(续)}\\
            \midrule
                  & \multicolumn{1}{c}{}   & \multicolumn{1}{c}{}  & \multicolumn{6}{c}{\textbf{训练集(5层交叉验证)}}   & \multicolumn{5}{c}{\textbf{验证集}}                                                                                                                                                                                                    \\
            \multirow{-2}{*}{\textbf{处理方式}}  & \multicolumn{1}{c}{\multirow{-2}{*}{\textbf{模型类型}}} & \multicolumn{1}{c}{\multirow{-2}{*}{\textbf{训练时间}}} & \textbf{混淆矩阵} & \textbf{精确率} & \textbf{召回率}& \textbf{F1值} & \textbf{准确率}& \textbf{AUC} & \textbf{混淆矩阵}& \textbf{精确率} & \textbf{召回率} & \textbf{F1值}& \textbf{准确率} \\
            \midrule
            \endhead 
            \midrule
            \endfoot
            \bottomrule
            \endlastfoot
            & 决策树      & 4.13    & $\left[ \begin{array}{cc} 1906 & 639 \\ 428 & 3318 \end{array} \right]$ & 83.9\%  & 88.6\%  & 86.1\% & 83.0\% & 0.914    & $\left[ \begin{array}{cc} 504 & 132 \\ 85 & 852 \end{array} \right]$ & 86.6\%  & 90.9\%  & 88.7\% & 86.2\% \\
            & K近邻     & 2.81    & $\left[ \begin{array}{cc} 2309 & 236 \\ 392 & 3354 \end{array} \right]$ & 93.4\%  & 90.0\%  & 91.4\% & 90.0\%   & 0.967  & $\left[ \begin{array}{cc} 587 & 49 \\ 87 & 850 \end{array} \right]$ & 94.5\%   & 90.7\%   & 92.6\% & 91.4\% \\
            \multirow{-3}{*}{补零} & 随机森林    & 23.23    & $\left[ \begin{array}{cc} 2295 & 250 \\ 271 & 3475 \end{array} \right]$ & 93.3\%  & 92.8\% & 93.0\% & 91.7\%  & 0.977 & $\left[ \begin{array}{cc} 587 & 49 \\ 52 & 885 \end{array} \right]$  & 94.8\% & 94.5\%   & 94.6\% & 93.6\% \\
            & 决策树      & 6.77    & $\left[ \begin{array}{cc} 2206 & 539 \\ 554 & 3192 \end{array} \right]$ & 85.6\%  & 85.2\%  & 85.4\% & 82.6\% & 0.902    & $\left[ \begin{array}{cc} 532 & 104 \\ 149 & 788 \end{array} \right]$ & 88.3\%  & 84.1\%  & 86.2\% & 83.9\% \\
            & K近邻     & 3.90    & $\left[ \begin{array}{cc} 2234 & 311 \\ 392 & 3354 \end{array} \right]$ & 91.5\%  & 89.5\%  & 90.5\% & 88.8\%   & 0.957 & $\left[ \begin{array}{cc} 583 & 53 \\ 79 & 858 \end{array} \right]$ & 94.2\%   & 91.2\%   & 92.9\% & 91.6\% \\
            \multirow{-3}{*}{重采样} & 随机森林    & 51.57    & $\left[ \begin{array}{cc} 2222 & 323 \\ 305 & 3441 \end{array} \right]$ & 91.4\%  & 91.9\% & 91.6\% & 90.0\%  & 0.967 & $\left[ \begin{array}{cc} 574 & 62 \\ 56 & 881 \end{array} \right]$  & 93.4\% & 94.0\%   & 93.7\% & 92.5\% \\
      \end{longtable}
\end{landscape}

本研究选取了在脉搏波时域特征集\Rnum{1}具有代表性的决策树、K近邻及随机森林等三种算法来进行本小节相关内容的研究。基于脉搏波时域特征集\Rnum{2}三种算法模型的具体表现如
\autoref{tab:model_screen3}所示。其中,决策树与K近邻模型的数据已经是经行过超参数调优后的结果;三种模型在训练集相关数据是对原始训练集数据经过5层交叉验证后得到的。

从\autoref{tab:model_screen3}可以得到以下结论:

\Rnum{1}、在测试集上,从模型方面来看,K近邻模型构建速度最快,随机森林处理速度最慢。而从模型分类效果来看,三种模型在测试集上的AUC数值均超过了0.900,其中随机森林的AUC数值更是在0.967以上。
在精度­召回率权衡上,K近邻与随机森林算法明显优于决策树算法,精确率、召回率及F1 数值均在 90.0\% 附近或以上。

\Rnum{2}、在验证集上,K近邻与随机森林算法的泛化能力也同样优于决策树算法。就准确率而言,K近邻与随机森林算法均在91\%以上,而决策树算法则不足87\%。
在精度­召回率权衡上,K近邻与随机森林算法明显优于决策树算法,精确率、召回率及F1 数值均在 90.0\% 。

\Rnum{3}、而横向对比两种对脉搏波波形进行对齐的处理方式可以发现,直接在脉搏波尾端进行补零处理会使各模型在构建速度、训练集AUC及模型整体准确率、精确率等方面均略微优于对脉搏波重采样处理的相应数据。

另外,对比\autoref{tab:super_para}、\autoref{tab:rf_dr_2}及\autoref{tab:model_screen3}可以发现,在脉搏波时域特征集\Rnum{1}及脉搏波时域特征集\Rnum{2}上分别构建的经过超参数调优的三种模型
在整体性能方面具有一致性。使用相同算法在两类时域特征集构建的模型在测试集上的混淆矩阵及准确率等数值接近,但整体而言,基于脉搏波时域特征集\Rnum{1}构建的模型在测试集上的泛化性能更好。这也从
侧面证明了两类时域特征集对于子痫前期的识别判断均有一定的表征能力,并且相较而言脉搏波时域特征集\Rnum{1}的表征能力更好。

2. 采样点贡献度

与上小节中处理类似,在脉搏波时域特征集\Rnum{2}上使用随机森林构建模型后,可以得到脉搏波时域特征集\Rnum{2}中各特征(即不同位置上的采样点)对最终模型的贡献度。由于基于贡献度对采样点的降维处理
缺乏实际处理的意义,本研究未进行相应的处理分析工作。\autoref{fig:rf_imp2}展示了\autoref{tab:model_screen3}中基于两种脉搏波对齐方式得到随机森林模型中各采样点的贡献度,此外,\autoref{fig:rf_imp2}
也给出了所有脉搏波波形两种对齐方式得到的“平均脉搏波”波形示意。

\begin{figure}[htbp]
      \centering
      \includegraphics[width=.6\linewidth]{results/rf_imp2}
      \caption{\label{fig:rf_imp2}脉搏波采样点对随机森林的贡献度}
\end{figure}

从\autoref{fig:rf_imp2}可以发现,按照两种方式处理得到的采样点贡献度分布形态上高度相似,均是在平均脉搏波的波峰之后出现贡献度的主峰,在平均脉搏波的下降支尾端附近出现第二个次峰。
此外,尽管补零处理的方式具有更多的采样点数(120个),但采样点贡献度的分布更为集中,各有效特征点的贡献度数值更大。在\autoref{fig:rf_imp2}中进行补零处理的80以后的采样点几乎完全没有参与
最终随机森林模型的构建。而采样点为80则对应着本研究中得到的大多数脉搏波波形时间的最大值。因此,在逻辑上这些新值及原始采样值中采样点80以后的数据均属于冗余项。
因此,尽管在数据预处理阶段进行的是对脉搏波波形对齐进行的是补零处理插值处理,但在模型真正构建时,对待脉搏波波形对齐进行的是截断处理。

另外,结合\autoref{tab:rf_dr_1}及\autoref{fig:rf_imp2}对比两种脉搏波时域特征集上得到各特征的贡献度分布可以发现,两类时域特征集中具有子痫前期表征能力的特征出现的位置高度相似。这说明脉搏波
在波峰之后及下降支结束之前的具体形态可能是识别子痫前期的关键。

二、按照被试人员分层抽样








\subsection{基于脉搏波时域特征集\Rnum{3}的结果及分析}

\section{非监督学习算法的具体表现及分析}
聚类分析是无监督学习的一种,旨在发现数据间是否有潜在的相似性\cite{Liu2018,Li2017}。
因此,我们做了额外的探究——在不给定脉搏波对应的子痫前期的数据标签的基础上,探究能否有效的将数据分成两类,并考察验证分类的效果。

本文进行了以下探究:
1.	分别使用基于ppg 特征的$ppg_feature$ 数据与基于ppg波形数据的$ppg_points $数据,使用sklearn的kmeans 方法进行了聚类分析,其中超参数$n_clusters$被设置为2。
此时,此时若我们将聚类结果作为其学习的分类结果,与数据对应的数据标签进行对照,也可以得到混淆矩阵分别如下

之所以每种分析会得到两个混淆矩阵,是因为聚类分析的簇是不带标签的。两个混淆矩阵是给予簇不同的标签(健康或子痫)。为分析方便,我们选取准确率高的一种聚类结果,即上表中的2、3。并在此基础上进行后续分析。

2.	从表格中可以看到两种数据的聚类分析的准确率分别达到了61.6\%与51.5\%。
基于特征的聚类效果优于直接使用脉搏波波形数据的。
其次,乍一看,数据分类效果并不太好。特别是后者的准确率堪堪超过50\%(典型的随机分类的效果)
但是,我们需要注意到聚类分析的本质是根据数据特征的相似度,也即数据波形的相似度。
因此,让我们来考察下划分出来的波形究竟如何。
于是,我们把图画出来。

此外,我们还需要注意到另外一个变量因素,在PE的影响下,所有患病孕妇的数据波形是会受到影响的,而与正常波形形态有异。因此,所有PE患者都受到了一定程序的医学干预(包括不限于降血压等治疗),目的是使PE患者能够恢复正常水平。因此,会出现假阴性远高于假阳性的现象。可以总结为,在PE确实能改变孕妇脉搏波波形的前提下,聚类分析中出现的假阴性高于假阳性证明是医学干预的必然结果。分析数据与理论分析保持一致。
3.	到具体波形

\begin{figure}[htbp]
    \centering
    \includegraphics[width=.6\linewidth]{unsupervised/cluster using points_2d}
    \caption[]{\label{fig:cluster2d}111}
\end{figure}
\begin{figure}[htbp]
    \centering
    \includegraphics[width=.7\linewidth]{unsupervised/cluster using points_3d}
    \caption[]{\label{fig:cluster3d}222}
\end{figure}
\section{小结}