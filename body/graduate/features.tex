\chapter{数据特征工程}
\section{引言}


特征的处理

* 分布特性

  * 有无差异性,SPSS统计,已用python实现

  * 特征相关性,heatmap

* 标准化

* 处理缺失值

* 构建新特征(char参数)

* 分析数据集准备:两种方式

  * A. by pulse

  * B. by person

* 降维与特征贡献度

  * 部分工作需要下一章节完成后才能确认

\subsection{PPG时域描述特征集构建}
本小节对本研究实际采用的多种PPG时域描述特征进行汇总,对各参数符号及前置计算条件也进行了统一说明,如\autoref{tab:allfeatures}所示。
TODO
\begin{center}
    \fontsize{10}{4}
    \begin{longtable}{p{3cm}<{\centering}p{1cm}<{\centering}p{2cm}<{\centering}p{6cm}<{\centering}p{1cm}<{\centering}}
        \caption{本研究使用的所有PPG时域指标一览}\\
        \label{tab:allfeatures}\\
        \hline\hline
            \textbf{研究者}&\textbf{时间}&\textbf{脉搏波参数}&\textbf{研究结果}&\textbf{备注}\\
        \hline
        \endfirsthead
        \caption[]{(续)}\\
        \hline
            \textbf{研究者}&\textbf{时间}&\textbf{脉搏波参数}&\textbf{研究结果}&\textbf{备注}\\
        \hline
        \endhead 
        \hline
        \endfoot
        \hline\hline
        \endlastfoot
        &       &       &       &  \\
        &       &       &       &  \\
        &       &       &       &  \\
        &       &       &       &  \\
        &       &       &       &  \\
    \end{longtable}
\end{center}
\section{数据清洗}
\section{新特征的创建}
\section{相关性验证}
\section{特征缩放}
\section{特征降维}
\section{数据集构建}
\subsection{两种划分方式}
\subsection{数据集的处理}
\section{小结}