\chapter{脉搏波时域特征集的构建与处理}
\section{引言}
第三章中介绍了多种描述PPG的时域特征,本章在其基础构建了基于时长、角度、斜率与面积等的PPG多维度时域特征集
同时,也将PPG信号的原始采样值作为特殊的描述特征,构建了PPG采样值时域特征集。本章详细介绍了这两类特征集的处理构建方式,为利用ML相关算法为PE的识别分析这一具体问题做好了数据准备。
另外,本章也明确了利用PPG数据通过ML方法对PE识别这一问题进行研究的具体目标,并按ML模型构建的一般要求,对两个特征集进行了预处理工作。
本章研究内容的框架图如\autoref{fig:frameworks4}所示。

\begin{figure}[htbp]
  \centering
  \includegraphics[width=\linewidth]{features/frameworks4}
  \caption{\label{fig:frameworks4}第四章研究内容框架}
\end{figure}
% \vspace{-0.5cm} 
\section{脉搏波时域特征集的构建}
作为构建能够识别PE的ML模型的数据准备,本章在第三章介绍过的PPG波形的新型时域特征的设计基础上,确定了这些新型参数的具体计算标准与方法。得到的基于时长、角度、斜率与面积等维度的新型特征构成了
\textbf{脉搏波多维度时域特征集}(photoplethysmographic multidimensional time-domain feature set,PPGMTFS)。
本章也将PPG的原始采样值视为一种特殊的描述特征,将一个完整的PPG波形的全部采样值视为\textbf{脉搏波采样值时域特征集}(photoplethysmographic sample-value-based time-domain feature set,PPGSTFS)。
由于设计理念不同,本论文并未将以上两个特征集进行合并或交叉处理,这两个特征集合各自被\textbf{独立地}用作后续PE识别分析阶段的输入数据。

\subsection{脉搏波多维度时域特征集}

依据第三章中PPG波形的新型时域特征的设计基础,只要在PPG波形上的确定任意点$Q$,都可以得到与之唯一关联的一系列基于线段、曲线长度、斜率、弧度及面积等多维度时域特征,如\autoref{fig:point}所示。
从采样的角度而言,当选取了足够多的这样的基准参考点,就可以用与这些基准点关联的多维度时域特征描述PPG波形。此时,这些多维度时域特征也必然是以向量的形式存在的,
也即第三章介绍过的脉搏波特征描述向量(photoplethysmographic feature vector,PPGFV)。

\begin{figure}[htbp]
  \centering
  \subfigure[\label{fig:lv}左视类指标示意]{
  \includegraphics[width=6.5cm]{features/lv}
  }
  \quad
  \subfigure[\label{fig:cv}中视类指标示意]{
  \includegraphics[width=6.5cm]{features/cv}
  }
  \quad
  \subfigure[\label{fig:sv}分层类指标示意]{
  \includegraphics[width=6.5cm]{features/sv}
  }
  \caption{\label{fig:all_views}确定基准点的三种策略示意}
\end{figure}

在PPG波形上选取并确定基准点的过程,是通过PPGFV对PPG进行描述的关键。
根据选取基准点的方式,本研究共使用了三种策略,分别为左视策略(left view strategy,LVS)、中视策略(center view strategy,CVS)与分层策略(scaled view strategy,SVS)。
这三种策略的概念图如\autoref{fig:all_views}所示。

一、LVS

LVS以PPG上升支起点为原点$T$,同时将PPG波峰设为$P$,将线段$TP$与水平基线$TT'$所构成的夹角$\angle PTT'$等分成若干份。这些等分线与PPG的交点${Q'}_i$被确定为参考基准点,这一过程如图\autoref{fig:lv}所示。

二、CVS

CVS以PPG波峰$P$在水平线$TT'$上的映射点$O$为中心,将线段$OP$与水平基线构成的两个夹角$\angle POT$与$\angle POT'$等分成若干份。将这些等分线与PPG的上升支、下降支的交点计为参考基准点${Q'}_i$,这一过程如图\autoref{fig:cv}所示。

三、SVS

SVS将PPG波峰$P$在水平线$TT'$上的映射点计为$O$,作线段$OP$的水平垂线将$OP$等分为若干份。将这些等分线与PPG的上升支、下降支的交点计为参考基准点${Q'}_i$,这一过程如图\autoref{fig:sv}所示。

基准点的维数在本研究中也被统一设置为10,该值是在完整描述PPG的前提下避免出现信息冗余的折中选项。
经由这些策略得到的基准点最终确定得到的多维度时域特征如\autoref{tab:allfeatures}所示。
在计算这些特征的具体数值前,PPG波形全部进行了标准化设置,波形幅值被调整缩放至[0,1000]区间内,其中PPG波形的上升支与下降支分别按第三章的介绍过的线性变换思想进行了处理。
该标准化过程中涉及的斜率与截距参数也在\autoref{tab:allfeatures}进行了补充。

\begin{center}
  \zihao{5}
  \begin{longtable}{m{1.5cm}<{\centering}m{3.5cm}<{\centering}m{2cm}<{\centering}m{8cm}<{\centering}}
    \caption{本研究使用的PPG多维度时域特征指标}\\
    \label{tab:allfeatures}\\
        \topline
        \colorhead \textbf{类别}&\textbf{特征名称}&\textbf{缩写符号}&\textbf{物理意义}\\
        \midline
        \endfirsthead
        \caption[]{(续)}\\
        \midline
        \colorhead \textbf{类别}&\textbf{特征名称}&\textbf{缩写符号}&\textbf{物理意义}\\
        \midline
        \endhead 
        \midline
        \endfoot
        \bottomline
        \endlastfoot
        \colorrowa &     左视斜率    &   LVS    &   各等分线所对应的直线斜率   \\
        \colorrowa &     左视上升支交点坐标 & LVLR & 等分线与PPG波形上升支交点横坐标 \\
        \colorrowa &     左视下降支交点坐标 & LVLF & 等分线与PPG波形下降支交点横坐标 \\
        \colorrowa \multirow{-4}*{左视策略}&     左视上升支交点距离 & LVRR & 等分线与PPG波形上升支交点与波形起点距离 \\
        \colorrowa &     左视下降支交点距离 & LVRF & 等分线与PPG波形下降支交点与波形起点距离 \\
        \colorrowa &     左视交点坐标差 & LVD & 左视上升支交点坐标与左视下降支交点坐标之差 \\
        \colorrowa &     左视上升支弧长 & LVALR & PPG波形上升支被等分线分割的各区间弧长 \\
        \colorrowa \multirow{-4}*{左视策略} & 左视下降支弧长 & LVALF & PPG波形下降支被等分线分割的各区间弧长 \\
        \colorrowc &     中视上升支交点坐标 & CVLR & 等分线与PPG波形上升支交点横坐标 \\
        \colorrowc &     中视下降支交点坐标 & CVLF & 等分线与PPG波形下降支交点横坐标 \\
        \colorrowc &     中视上升支交点距离 & CVRR & 等分线与PPG波形上升支交点与峰值水平映射点距离 \\
        \colorrowc &     中视下降支交点距离 & CVRF & 等分线与PPG波形下降支交点与峰值水平映射点距离 \\
        \colorrowc &     中视交点坐标差 & CVD & 中视上升支交点坐标与中视下降支交点坐标之差 \\
        \colorrowc &     中视上升支弧长 & CVALR & PPG波形上升支被等分线分割的各区间弧长 \\
        \colorrowc &     中视下降支弧长 & CVALF & PPG波形下降支被等分线分割的各区间弧长 \\
        \colorrowc &     中视上升支面积 & CVAR & PPG波形上升支被等分线分割的各区域面积 \\
        \colorrowc \multirow{-12}*{中视策略} &     中视下降支面积 & CVAF & PPG波形下降支被等分线分割的各区域面积 \\
        \colorrowa &     分层上升支交点坐标 & SVLR & 等分线与PPG波形上升支交点横坐标 \\
        \colorrowa &     分层下降支交点坐标 & SVLF & 等分线与PPG波形下降支交点横坐标 \\
        \colorrowa &     分层上升支面积 & SVAR & PPG波形上升支被等分线分割的各区域面积 \\
        \colorrowa &     分层下降支面积 & SVAF & PPG波形下降支被等分线分割的各区域面积 \\
        \colorrowa &     分层面积 & SVAT & PPG波形整体被等分线分割的各区域面积 \\
        \colorrowa &     分层上升支交点距离 & SVRR & 等分线与PPG波形上升支交点与峰值水平映射点距离 \\
        \colorrowa &     分层下降支交点距离 & SVRF & 等分线与PPG波形下降支交点与峰值水平映射点距离 \\
        \colorrowa &     分层交点坐标差 & SVD &  分层上升支交点坐标与分层下降支交点坐标之差\\
        \colorrowa &     分层上升支弧长 & SVALR & PPG波形上升支被等分线分割的各区间弧长 \\
        \colorrowa &     分层下降支弧长 & SVALF & PPG波形下降支被等分线分割的各区间弧长 \\
        \colorrowa \multirow{-11}*{分层策略} &     分层上升支斜率 & SVSR & 等分线与PPG波形上升支交点与PPG峰值水平映射点所形成直线的斜率\\
        \colorrowa \multirow{-1}*{分层策略} & 分层下降支斜率 & SVSF & 等分线与PPG波形下降支交点与PPG峰值水平映射点所形成直线的斜率 \\
        \colorrowc &     上升支标准化斜率 & STDKR & 标准化PPG上升支波形至[0-1000]时使用的斜率 \\
        \colorrowc &     上升支标准化截距 & STDBR & 标准化PPG上升支波形至[0-1000]时使用的截距 \\
        \colorrowc &     下降支标准化斜率& STDKF & 标准化PPG下降支波形至[0-1000]时使用的斜率 \\
        \colorrowc \multirow{-4}*{标准化指标}   &  下降支支标准化截距 & STDBF & 标准化PPG上升支波形至[0-1000]时使用的截距 \\
  \end{longtable}
\end{center}
\vspace{-1.2cm} 
\subsection{脉搏波采样值时域特征集}

前文中介绍的所有PPG特征均是在原始采样值的基础上,经一定的数学处理后,间接对PPG波形进行描述的。由于PPG信号的原始采样值本身就是一种量化描述,
故本文将其作为一种特殊的直接描述PPG波形的特征。
从这个角度而言,一个完整波形的所有采样值可视为向量维数即为采样数的特殊PPGFV。

\autoref{fig:no_pe}对比展示了四例被试的PPG波形的采样值差异,其中,各子图在同一坐标系下绘制了同一被试的所有PPG波形,这些波形均按起点进行了对齐,波形的采样值经标准化后被调整至[0,1]区间。

\begin{figure}[htbp]
  \centering
  \includegraphics[width=0.75\linewidth]{features/ppgs}
  \caption{\label{fig:no_pe}被试孕妇PPG经标准化处理后的波形对照}
\end{figure}

从\autoref{fig:no_pe}可以看到,实验组与对照组的被试在PPG波形形态及整体分布上均呈现出一定的差异,这说明通过采样值描述PPG波形的可行性。
此外,从\autoref{fig:no_pe}也可以看到,即使对同一被试而言,不同PPG波形之间也存在这较为明显的时长差异。若不加任何处理,这种差异会直接导致PPGSTFS的维数差异。
为消除这种维数差异可能带来的影响,本文提出了三种处理策略进行PPG波形对齐操作,分别为重采样策略(resampling strategy, RS)、补齐策略(complementary strategies,CS)
与截取策略(interception strategy,IS)。为说明三种策略处理思路,下面以两个PPG波形的对齐处理过程为例,记这两个波形的原始采样数分别记为$m$、$n$且$m \neq n$,不失一般性,设$m>n$。

一、RS

RS主要通过信号的插值与抽取调整数据采样率,随后将两个PPG波形的采样点数调整至相同数值$n_r$。
若按RS将采样长度为$m$的波形调整至$n_r$,记$m$与$n_r$的最小公倍数为$p$,只需将原始数据均匀插值至$p$点,再按每$p/n_r$个点进行抽取即可。

由于采集得到的绝大多数PPG波形采样点数不足80,本研究最终将$n_r$值设置为100。
该值保证了PPG波形的描述的分辨率与精度,同时可保证只有极少数波形需要进行采样点的压缩。
将\autoref{fig:no_pe}中的被试数据进行重采样处理,所得的最终结果如\autoref{fig:no_pe2}展示。

\begin{figure}[htbp]
  \centering
  \includegraphics[width=0.75\linewidth]{features/ppgs1}
  \caption{\label{fig:no_pe2}被试孕妇PPG经标准化、重采样处理后的波形对照}
\end{figure}

二、CS

CS则是以采样点数为$m$的PPG波形为基准,对两一个波形在尾端直接进行补零处理,使两者的采样点均达到$n_c$($n_c=m$)。

由于本文得到的实验数据采样率为100$Hz$,$n_c$值被设为120。与这一数值对应的PPG波形周期也高达1.2$s$,该值已经超过了本研究采集得到的任一有效的PPG波形周期。

三、IS

与CS相反,IS会以采样点数为$n$的PPG波形为基准,舍弃掉另一个波形在$n$点以后的采样值,使两者的采样点均达到$n_i$($n_i=n$)。此过程不可避免地出现了数据损失,故这种策略被暂时弃用。

\section{脉搏波时域特征集的处理}
数据是为具体的ML研究问题而准备的,故在对数据进行下一步处理前,有必要明确本文需要研究的具体ML的问题与任务。本小节对本文要开展的具体ML研究问题进行了介绍。在此基础上,
为保证后续章节ML相关研究可顺利开展,对PPGMTFS与PPGSTFS等两个数据集的预处理工作进行了说明。此部分工作主要包括数据集的划分、特征缩放及特征降维等。

\subsection{关于子痫前期的猜想与具体机器学习方向}
绪论中已经阐述过,本文的研究目标是探寻通过PPG识别判断PE病发的可能,而该目标本质上可以划分为ML领域内的二分类任务。
而本文对PPG与PE之间的可能潜在的关系,提出了以下两种猜想。

一、\textbf{单个PPG波形可以反映PE的病发状态}

这种猜想认为PPG的波形是孕妇PE病发状态的稳定“表达”,单个PPG波形包含了可以识别区分PE的全部信息。换言之,该猜想认为由PE导致的病生理变化可以在单个波形上得到全部体现,
单个PPG波形能够作为分析识别PE的最小分析对象。

二、\textbf{被试的全部PPG波形才可以反映PE病发状态}

这种猜想认为PPG的单个波形是孕妇PE病发状态的不稳定“表达”,但孕妇的一段时间内采集得到的所有PPG波形包含了可以识别区分PE的全部信息。换言之,该猜想认为由PE导致的病生理变化可以在孕妇
的多数PPG波形的形态特征上得到全部体现,需要被试的全部PPG波形进行一个类似“群体决策”的过程。

针对上述两个猜想,本研究拟从以下两个方向展开了具体研究工作:

一、基于波形的研究

基于波形的研究(pulse-based research,PR)将单个PPG波形作为分析识别PE的最小分析对象,并进行ML算法模型的研究。此研究方向下,同一被试的PPG数据可能对应ML模型训练与验证阶段的多个数据样本,这种方式相对地增加了
数据集的样本容量。

二、基于被试的研究

基于被试的研究(subject-based research,SR)将单个被试的所有PPG波形作为一个整体进行分析处理,并进行ML算法模型的研究。此研究方向下,同一被试的PPG波形只会出现在训练集或测试集之中,
测试集中的数据样本对于训练得到的模型是完全陌生的未见示例,因此更考ML模型的泛化能力。

\subsection{数据集的划分}
数据集的划分是监督学习在开始模型训练前必不可少的重要步骤,是将原始数据集划分为训练集与测试集的过程。其中,训练集作为训练模型的输入数据,而测试集则是用来评估该模型在未见示例上的泛化能力。

第三章已经介绍过,本文共采集得到79例被试孕妇的有效PPG数据波形共计7864个,其中,实验组44名被试包含有效波形4683个,对照组35名被试包含波形3181个。
由于本文使用了PPGMTFS与PPGSTFS等两个数据集,并提出了PR与SR等两个具体研究方向,且后续章节在进行研究时也使用了多种ML算法进行模型的构建,数据集的合理划分显得尤为重要。
本文从以下方面进行了处理工作。

一、按合理比例划分

前人的研究已经证实,在进行数据划分时,测试集包含的样本数量与全部样本的比例的最优区间在[20\%, 30\%]\cite{Gholamy2018Why7O}。
本文将该划分比例数值设为20\%,即按照训练集与测试集4:1的比例分别按被试与波形对原始数据抽样。

二、分层抽样

由于本文实验得到的PPG数据在是否患有PE的这一问题上存在分布不平衡现象,直接使用纯随机方法抽样有可能导致抽样偏差,最终影响准确性、稳定性与鲁棒性等模型效果\cite{Aurélien2018}。
为了避免此情况发生,本研究额外应用了分层抽样的策略划分数据集,将数据样本按其是否属于实验组分成两组,对新得到的两组数据
分别按照4:1比例进行抽样,两组对应的抽样结果在合并之后才形成最终的训练集与测试集。

三、固化抽样结果

为使后续多种ML模型的性能具有可比行,必须保持数据集的划分在整个研究过程中的一致性。因此,前两步中的分层抽样过程只进行一次,随即被固化保存。该划分结果在之后的ML过程中不再进行任何其他调整,
供所有ML算法训练模型或进行测试使用。

最终的数据集划分结果如\autoref{tab:dataset}所示。
\begin{center}
  \zihao{-5}
  \begin{longtable}{m{1.5cm}<{\centering}m{1.5cm}<{\centering}m{1.5cm}<{\centering}m{1.5cm}<{\centering}m{1.5cm}<{\centering}m{1.5cm}<{\centering}}
    \caption{数据集划分结果}\\
    \label{tab:dataset}\\
        \topline
        \colorhead & & \multicolumn{2}{c}{\textbf{训练集}} & \multicolumn{2}{c}{\textbf{训练集}}\\
        \colorhead \multirow{-2}{*}{\textbf{研究方向}}& \multirow{-2}{*}{\textbf{研究方向}}& \textbf{实验组} & \textbf{对照组} & \textbf{实验组} & \textbf{对照组} \\
        \midline
        \endfirsthead
        \caption[]{(续)}\\
        \midline
        \colorhead & & \multicolumn{2}{c}{\textbf{训练集}} & \multicolumn{2}{c}{\textbf{训练集}}\\
        \colorhead \multirow{-2}{*}{\textbf{研究方向}}& \multirow{-2}{*}{\textbf{研究方向}}& \textbf{实验组} & \textbf{对照组} & \textbf{实验组} & \textbf{对照组} \\
        \midline
        \endhead 
        \midline
        \endfoot
        \bottomline
        \endlastfoot
        \colorrowa  PR  & 7864  & 3746 & 2545 & 937 & 636 \\
        \colorrowa  PR  & 79  & 35 & 28 & 9 & 7 \\           
  \end{longtable}
\end{center}
\vspace{-0.8cm} 

\subsection{特征缩放}
一般而言,原始数据的输入特征在数值属性出现较大的差异会导致机器学习模型的性能下降、表现欠佳\cite{Aurélien2018}。为保证这些输入特征能满足特定的机器学习算法的输入要求,通常还要对这些特征的数值分布进行一定的调整,这也就是特征缩放操作。

常见的特征缩放处理原则是同比例缩放所有属性,使用的方法有归一化与标准化等两类方法。归一化方法又称为最小-最大缩放,可将所有数据的特征属性值同比例映射至[0,1]区间内
\begin{equation}
  \label{equ:maxmin}
  z = \frac{x - x_{min}}{x_{max}-x_{min}}
\end{equation}
其中,$z$为缩放后的数据,$x$需要标准化的数据,$x_{min}$与$x_{max}$分别对应这批数据的最小值与最大值。

而标准化的过程可将所有数据的特征属性调整至符合正态分布
\begin{equation}
  \label{equ:maxmin}
  z = \frac{x - \mu}{\epsilon}
\end{equation}
其中,$z$为缩放后的数据,$x$需要标准化的数据,$\mu$与$\epsilon$分别对应这批数据的平均值与样本方差。

\subsection{特征降维}
机器学习模型的训练过程花费的时间成本会随着输入特征维数的增加而成非线性增加,这也就是通常而言的维数诅咒或维数灾难。
为加快模型训练速度,一种可行的策略在构建模型时尽可能只使用“与预期结果最相关的”、“最重要的”输入特征,即按照特征的贡献度对
原始数据集进行降维处理。需要注意的是,数据降维在加速训练的同时,通常也会导致模型性能的下降。因此,一般认为特征降维是机器学习过程中的一个可选项而非必选项。


特征降维在训练模型前后均可进行。在训练模型前的特征降维处理
主要依赖于特征数据属性值的分布特性进行筛选;在模型训练完成后的特征降维主要依赖于特征属性对模型的贡献程度进行筛选。
本小节在训练机器学习模型前,使用U检验通过评估PPG多维度时域特征集中各特征参数与PE的相关性完成了筛选工作,如\autoref{tab:utest}所示。

\begin{center}
  \zihao{-5}
  \begin{longtable}{m{2.5cm}<{\centering}m{2cm}<{\centering}m{2.5cm}<{\centering}m{2cm}<{\centering}m{2.5cm}<{\centering}m{2cm}<{\centering}}
    \caption{脉搏波时域特征集\Rnum{1}数据特征的U检验结果}\\
    \label{tab:utest}\\
        \topline
        \colorhead \textbf{特征}&\textbf{p值}&\textbf{特征}&\textbf{p值}&\textbf{特征}&\textbf{p值}\\
        \midline
        \endfirsthead
        \caption[]{(续)}\\
        \midline
        \colorhead \textbf{特征}&\textbf{p值}&\textbf{特征}&\textbf{p值}&\textbf{特征}&\textbf{p值}\\
        \midline
        \endhead 
        \midline
        \endfoot
        \bottomline
        \endlastfoot
        \colorrowa  LVLR\_9  &  0.004 &  CVLF\_1  & \cellcolor{pink} \textbf{0.068} &  SVRR\_2  &  0.027 \\
        \colorrowc  LVLF\_1  &  \cellcolor{pink}\textbf{0.27}  &  CVLF\_2  &  0.038 &  SVRR\_3  &  0.001 \\
        \colorrowa  LVRR\_6  &  0.002 &  CVRF\_4  & \cellcolor{pink} \textbf{0.44}  &  SVRR\_4  &  0.001 \\
        \colorrowc  LVRR\_7  &  \cellcolor{pink}\textbf{0.387} &  CVD\_1   &  \cellcolor{pink}\textbf{0.159} &  SVD\_2   &  0.009 \\
        \colorrowa  LVD\_1   &  0.022 &  CVD\_2   &  0.024 &  SVD\_3   & \cellcolor{pink} \textbf{0.65}  \\
        \colorrowc  LVD\_2   &  0.006 &  CVALF\_6 &  0.02  &  SVALR\_2 & \cellcolor{pink} \textbf{0.078} \\
        \colorrowa  LVALR\_4 &  0.013 &           &        &  SVALR\_3 &  0.001 \\
        \colorrowc  LVALR\_5 &  \cellcolor{pink}\textbf{0.063} &           &        &           &               
  \end{longtable}
\end{center}

\autoref{tab:utest}给出了经U检验得到$p$值$>10^{-4}$的特征参数,$p$值$> 0.05$的特征未参与后续模型的训练过程,这部分特征参数在\autoref{tab:utest}进行了突出显示。
;在模型训练后的特征筛选处理可参见下一章节相关内容。


\section{小结}
本小节在已经脉搏波预处理过程的基础上完成了脉搏波时域特征集合的构建,包括基于脉搏波波形特征描述向量、原始采样值及脉搏波波形差异值等三大类。
此外,本小节也完成了机器学习的数据集划分、特征缩放及特征降维等部分的工作。