\chapter{脉搏波描述特征集的构建及分析}
\section{引言}
第三章已经对脉搏波的描述特征参数进行了介绍,本章选取了其中的部分参数构成了基本机器学习的输入数据集。
本章对这部分特征数据按机器学习的一般要求进行了相关处理及准备工作,同时也对这些参数进行了基本分析。

\section{数据集的构建}
本文在第三章已经对脉搏波的描述特征参数进行了介绍,这里选取了其中的部分参数构成了基本机器学习的输入数据集。实际上,本研究共构建了三类相互独立的脉搏波描述特征集分别用作特定模型的训练输入数据。

一、波形描述特征集合



二、波形间差异描述特征集合

三、原始波形采样值


本小节对本研究实际采用的多种PPG时域描述特征进行汇总,对各参数符号及前置计算条件也进行了统一说明,如\autoref{tab:allfeatures}所示。
\begin{center}
    \fontsize{10}{4}
    \begin{longtable}{p{3cm}<{\centering}p{1cm}<{\centering}p{2cm}<{\centering}p{6cm}<{\centering}p{1cm}<{\centering}}
        \caption{本研究使用的所有PPG时域指标一览}\\
        \label{tab:allfeatures}\\
        \hline\hline
            \textbf{研究者}&\textbf{时间}&\textbf{脉搏波参数}&\textbf{研究结果}&\textbf{备注}\\
        \hline
        \endfirsthead
        \caption[]{(续)}\\
        \hline
            \textbf{研究者}&\textbf{时间}&\textbf{脉搏波参数}&\textbf{研究结果}&\textbf{备注}\\
        \hline
        \endhead 
        \hline
        \endfoot
        \hline\hline
        \endlastfoot
        &       &       &       &  \\
        &       &       &       &  \\
        &       &       &       &  \\
        &       &       &       &  \\
        &       &       &       &  \\
    \end{longtable}
\end{center}
\subsection{数据集的划分}
在大多数机器学习的案例里,将原始数据集重新划分为训练集与测试集是必不可少的一个步骤。

* 分析数据集准备:两种方式

  * A. by pulse

  * B. by person

\subsection{数据清洗}
* 处理缺失值

\subsection{新特征的创建}
* 构建新特征(char参数)
\subsection{特征相关性验证}
* 分布特性

  * 有无差异性,SPSS统计,已用python实现

  * 特征相关性,heatmap
\subsection{特征缩放}
在大多数情况下,原始数据的特征在数值属性出现较大的比例差异会导致机器学习模型的性能下降、表现欠佳\cite{Aurélien2018}。因此,需要对特征数值的分布进行一定的调整使其称为能够满足具体模型算法的输入要求。
一般的处理原则是同比例缩放所有属性,常见的方法有归一化与标准化两类。

一、归一化

归一化亦称为最小-最大缩放,

二、标准化

balaba
\subsection{特征降维}
事实证明,机器学习模型的训练速度随着训练数据的特征数量增加而降低,这也就是通常而言的维数诅咒或维数灾难。因此,一种可行的策略在构建模型时尽可能只使用“最重要的”特征,即按照特征的贡献度对
原始数据集进行降维处理。但需要注意的是,数据降维在加速训练的同时,通常也会导致系统性能的下降。从这个角度说,特征降维是机器学习过程中的一个可选项而非必选项。

由于特征降维在划分上是数据特征工程的一部分,但在逻辑上的处理又往往与具体的机器学习模型绑定。故本小节暂时只进行概念,真正的降维分析请参见本文后续章节内容。
\subsection{数据集的分析}
\section{小结}