\chapter{结论}
近年来随着智能手机的普及与信息通信技术的高速发展,移动医疗的概念逐渐兴起,基于移动式、便携式的电子医疗分析仪受到广泛的关注。
在此背景下,本文主要研究了基于安卓智能手机的心电监护系统的应用前景并在调研了相关技术的前提下完成了一定的开发及实现工作。本文的主要工作有: 

(1)在 Matlab 上完成了心电信号检测的所有程序。从心电信号数据的提取、心电信号的滤波去噪到心电信号的特征提取等过程均完成了相应的计算及仿真。 

(2)在 Android 下开发了心电监护的程序,这是本文进行的主要工作。针对通过 Matlab 对 MIT-BIH 提取得到的原始数据进行了分析,在智能手机上了完成
了数据的滤波去噪及去基线漂移过程,同时动态对比显示了滤波前后的心电波形,针对滤波后的心电数据进行了相关特征值提取及图像标注显示工作。在程序的最后,
对特征提取的结果进行了直接显示。 

通过系统开发及对算法的优化,作者对基于智能手机的心电监护系统的研究开发技术及医学信号分析处理都有了较为深刻的理解与认识并取得了一定研究成果,
但由于作者系统开发经验较为欠缺和时间不足,本系统还有许多值得完善提高改进的地方,下一步工作的目标是: 

(1)本文将心电信号的去噪算法、特征点识别算法、特征参数提取算法、分析算法通过在 Android 上的实现,对一定的心电数据进行自动分析有着较好的检测结果,
但是针对不同心电图、不同心脏疾病的心电数据,本文的相关算法仍然有着一定改进优化的空间。 

(2)针对 Java 下检测的数据进行简单辅助诊断的功能有待进一步加强。 

(3)本文进行数据分析的来源是本地数据,但目前智能手机的各项通信技术日臻完善,基于蓝牙、WIFI 等无线通信技术已经较为成熟,
下步工作目标可以选定一种通信技术可以接收硬件电路发送的心电监测数据,实现心电实时数据的分析。 

总之,在 ECG 的分析处理处理与 Android 开发领域,新的问题不断出现,新的算法不断涌现。随着科技的进步,基于 Android 的心电监护系统将会不断成熟完善。 
