\usepackage{graphicx}
\usepackage{geometry}
\usepackage{tabularx}
\usepackage{multido}
\usepackage{fancyhdr}
\usepackage{fontspec}
\usepackage{titlesec}
\usepackage{tocloft}
\usepackage{multirow}
\usepackage{makecell}
\usepackage{hyperref}
\usepackage{ulem}
\usepackage{pdfpages}
\usepackage[
    style=gb7714-2015,
    % gbpub=false,         % Uncomment if you do NOT want '[S.l. : s.n.]' in reference entries, GitHub Issue (#47)
    % gbnamefmt=lowercase, % Uncomment if you do NOT want uppercase author names in reference entries, GitHub Issue (#23)
]{biblatex}
\usepackage{siunitx}
\usepackage{caption}
\DeclareCaptionLabelSeparator{twospace}{\ ~} 
\captionsetup{labelsep=twospace}
\usepackage{color}
\usepackage{chngcntr}
\usepackage{enumitem}
\usepackage{float}
\usepackage{listings}
\usepackage{amssymb}
\usepackage{etoolbox}
\usepackage{xparse}
\usepackage{bookmark}
\usepackage{calc}
\usepackage{booktabs}
\usepackage{threeparttable}
\usepackage{amsmath}
\makeatletter
\let\c@lofdepth\relax
\let\c@lotdepth\relax
\makeatother
\usepackage{subfigure}
\usepackage{rotating}
\usepackage{longtable}
\usepackage{lscape}
\usepackage{listings}
\usepackage{xcolor}
\usepackage{colortbl}
\definecolor{dkgreen}{rgb}{0,0.6,0}
\definecolor{gray}{rgb}{0.5,0.5,0.5}
\definecolor{mauve}{rgb}{0.58,0,0.82}
\lstdefinestyle{myJava}{
    frame=none,
    language=Java,
    aboveskip=3mm,
    belowskip=3mm,
    showstringspaces=false,
    columns=flexible,
    basicstyle = \ttfamily\small,
    numbers=none,
    numberstyle=\tiny\color{gray},
    keywordstyle=\color{blue},
    commentstyle=\color{dkgreen},
    stringstyle=\color{mauve},
    backgroundcolor=\color{gray!10!},
    captionpos=t,
    breaklines=true,
    breakatwhitespace=true,
    tabsize=3
}
\lstdefinestyle{myPython}{
	frame=none,
	language=Python,
	aboveskip=3mm,
	belowskip=3mm,
	showstringspaces=false,
	columns=flexible,
	numberstyle=\small\color{red},
	basicstyle={\small\ttfamily},
	keywordstyle=\color{blue},
	commentstyle=\color{dkgreen},
	stringstyle=\color{mauve},
	breaklines=true,
	breakatwhitespace=true,
	tabsize=3
}
\usepackage{algorithm}
\usepackage{algorithmicx}
\usepackage{algpseudocode}
\usepackage{amsmath}
\renewcommand{\algorithmicrequire}{\textbf{输入:}}
\renewcommand{\algorithmicensure}{\textbf{输出:}}
\makeatletter
\newenvironment{breakablealgorithm}
{% \begin{breakablealgorithm}
\begin{center}
    \refstepcounter{algorithm}% New algorithm
    \hrule height.8pt depth0pt \kern2pt% \@fs@pre for \@fs@ruled 画线
    \renewcommand{\caption}[2][\relax]{% Make a new \caption
    {\raggedright\textbf{\ALG@name~\thealgorithm} ##2\par}%
    \ifx\relax##1\relax % #1 is \relax
        \addcontentsline{loa}{algorithm}{\protect\numberline{\thealgorithm}##2}%
    \else % #1 is not \relax
        \addcontentsline{loa}{algorithm}{\protect\numberline{\thealgorithm}##1}%
    \fi
    \kern2pt\hrule\kern2pt
    }
}{% \end{breakablealgorithm}
    \kern2pt\hrule\relax% \@fs@post for \@fs@ruled 画线
\end{center}
}
\makeatother

\usepackage{ulem}